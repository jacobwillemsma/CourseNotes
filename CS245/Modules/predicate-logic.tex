\newpagesection{First Order Predicate Logic}
\subsection{Introduction/Motivation}

In propositional logic, only the logical forms of the propositions are analyzed.  We never actually \emph{talk} about the objects we are working with.  This is important because we need some way to properly address what kind of qualities or properties are attributed to an object, some objects or all objects, without enumerating all the objects in a set.\\

\exbox{Motivating Example}{Consider the statement:\\
\emph{Every student is younger than some instructor.}\\

This statement is certainly true, but how can we show this?  Certainly not with propositional logic, we need some way to assign being \emph{younger} than someone, some way to differentiate instructor from student, and some way to show that all of something is younger than some of another thing.}

\defnbox{First-Order Logic}{\textbf{First-order logic (FOL)} gives us a language to express statements about objects.\\

FOL is expressed with the following ingredients:
\begin{itemize}
\item Domain of objects (What kind of object they are)
\item Variables
\item Designated individuals (E.g. constants)
\item Functions (E.g. +)
\item Relations (E.g. =)
\item Quantifiers
\item Propositional connectives
\end{itemize}
}

For example, we could write $S(andy)$ to denote that Andy is a student and $I(paul)$ to denote that Paul is an instructor.  Likewise, $Y(andy,paul)$ could mean that Andy is younger than Paul.  In oder to make predicates like these more useful and easier to write, we use \emph{variables} or \emph{placeholders}.  These assigned relations to variables are called \textbf{predicates}.\\

In the previous case would would have something as follows:\\
\begin{itemize}
\item $S(x)$:	$x$ is a student.
\item $I(x)$:	$x$ is an instructor.
\item $Y(x,y)$:	$x$ is younger than $y$.
\end{itemize}

Quantifiers are the next big change brought into FOL.  Quantifiers are used to make the difference between \emph{every} and \emph{some}.  We will use what we learned in MATH135 to describe these:\\
\begin{itemize}
\item $\forall$:	For all (every).
\item $\exists$:	There exists (some).
\end{itemize}

For all is called the \emph{universal quantifier} and there exists is called the \emph{existential quantifier}.\\

Just with these two additions and our knowledge of propositional logic, we can right the previous statement (although a little paraphrased) completely in FOL.

\exbox{First FOL Statement}{The Statement \emph{Every student is younger than some instructor.} can be translated into FOL as the following:\\

$\forall$x.(S(x)$\to$($\exists$y.(I(y) $\wedge$ Y(x,y))))}

\defnbox{Title}{Definition}
\exbox{Example number}{Example text.}
\newpagesubsection{Logic}