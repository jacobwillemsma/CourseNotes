\newpagesection{Propositional logic}

\subsection{Introduction/Background}

Welcome to CS245 - Logic and Computation.\\

To begin, we will review a couple topics covered in MATH135 (a prerequisite course) to warm up and then jump straight into proposition logic.\\

I will not be covering the brief introduction to \textbf{set theory} at the beginning of this course because at this point, most of you should have a strong sense of that material.  If however you do not feel confident with this, please review the slides 1-12 on propositional logic.\\

\defnbox{Induction}{\textbf{Induction} is typically a scary word, however the logic behind it is fairly simple.  All \textbf{induction} serves to accomplish is to \emph{prove} a theorem is true by showing that if it is true for any \emph{particular} case and it is also true for the next case in a series, then it must be true for all particular cases.}

\exbox{1}{We can use an inductive process to define how to find \emph{natural numbers}.}

\begin{itemize}
\item 0 $\in$ N.
\item For any \emph{n}, if \emph{n} $\in$ N, then \emph{n'} in N, where \emph{n'} is the successor of \emph{n}.
\item \emph{n} $\in$ N only if \emph{n} has been generated by step 1 and step 2.
\end{itemize}

There are plenty of examples of induction online.  I will let you look those up as well.\\

\newpagesubsection{Logic}

\defnbox{Logic}{Logic is the study of the principles of valid reasoning and inference.  Computer science and math are not the only two major schools that study logic, in fact philosophy is obsessed with it.  The act of being able to properly reason is at the foundation for debate and life itself.}

\exbox{1}{Is the following argument valid?  Which is to say, does it have a sound bases in logic? \vspace{10 mm}

If the train arrives late and there are no taxis at the station, then John is late for his meeting. John is not late for his meeting. The train did arrive late. Therefore, there were taxis in the station.}

To be able to show that this sentence is valid, it is much easier to break it down into the form of symbols (atoms) and relations (connectives).\\

For instance, the example sentence can be broken down into the following line of reasoning:\\

\exbox{2}{If \emph{p} and not \emph{q}, then \emph{r}.  Not \emph{r}. \emph{p}. Therefore \emph{q}.}

In order to obtain this string of logic, we have broken down the statement as follows.\\

\begin{itemize}
\item \emph{p} means that the train arrives late.
\item \emph{q} means that there are taxis at the station. \emph{n.b. we negate this in the string of logic}.
\item \emph{r} means John is late for his meeting.
\end{itemize}

\defnbox{Proposition}{We begin our quest to understand logic by defining a formal language where we can express sentences and logical structures. we will call this a \textbf{proposition}.  A proposition is a declarative sentence that is either \emph{true} or \emph{false}.}

Every proposition can only ever have the result true, or the result false.  It can also not simultaneously be both true and false.\\

We will focus on building \textbf{symbolic} logics, therefore we will use strings of symbols to state propositions and build arguments using connectives.\\

For instance, the train argumentation can be summed up into the following:\\

\exbox{3}{((p $\land$ $\neg$ q) $\to$ r)}

A logic is formalized by the following:\\
\begin{itemize}
\item Syntax.
\item Semantics.
\item Proof procedures.
\end{itemize}

\newpagesubsection{Syntax of propositional logic}
As briefly stated before, in propositional logic, we will refer to the simple propositions that are the building blocks to create \textbf{compound propositions} as simply \textbf{atoms}.\\

Right then, let's get to defining the propositional language.\\

\defnbox{Propositional Language}{The \textbf{propositional language} (denoted L\textsuperscript{p}) consists of the:
\begin{itemize}
\item \textbf{proposition symbols} (denoted with small Latin letters like \emph{p}, \emph{q}).  This set is know as \emph{Atom(L\textsuperscript{p})} and is the basic building blocks of proposition language.
\item 2 \textbf{punctuation} symbols "(" and ")".
\item 5 \textbf{connectives}:
	\begin{itemize}
	\item $\lnot$ (not or negation).
	\item $\land$ (and).
	\item $\lor$ (or).
	\item $\to$ (implies).
	\item $\leftrightarrow$ (equivalence).
	\end{itemize}
\end{itemize}}

\defnbox{Expression}{An \textbf{expression} is a finite string of symbols where the length of the expression is the number of symbols in the expression.}

At this time, the course notes and slides go into a great deal about terminology for expressions in L\textsuperscript{p}.  I don't particularly find these useful to go into detail about, so I will quickly bullet point them:\\

\begin{itemize}
\item The \textbf{empty expression} is denoted: $\emptyset$.
\item Two expressions are \textbf{equal} if and only if they are the same length and every symbol is the same at each index [1 ... n].
\item \emph{UV} is the \textbf{concatenation} of two expressions \emph{U} and \emph{V}.  It has length \emph{U} + \emph{V}.
\item If \emph{U} is part of a expression \emph{V}, then it is a \textbf{segment} of V.
\item If \emph{U} is part of a expression \emph{V} and \emph{U} $\neq$ \emph{V}, then it is a \textbf{proper segment} of V.
\item We may use \emph{*} to refer to an arbitrary \textbf{binary connective}.
\end{itemize}

\defnbox{Formation Rules}{The set of \textbf{formulas} (denoted \emph{Form(L\textsuperscript{p})}) is inductively defined as follows:
\begin{itemize}
\item \emph{Atom(L\textsuperscript{p})} $\subseteq$ \emph{Form(L\textsuperscript{p})}.
\item If A $\in$ \emph{Form(L\textsuperscript{p})}, then ($\lnot$A) $\in$ \emph{Form(L\textsuperscript{p})}.
\item If A, B $\in$ \emph{Form(L\textsuperscript{p})}, then (A * B) $\in$ \emph{Form(L\textsuperscript{p})}.
\end{itemize}}

\defnbox{\emph{Form(L\textsuperscript{p})}}{\emph{Form(L\textsuperscript{p})} is the smallest class of expressions of \emph{L\textsuperscript{p}} which is \textbf{closed} under the formation rules of \emph{L\textsuperscript{p}}.  We say that it is a \textbf{well formed formula}.}

How then do we test if a expression is a well formed formula?  There are two methods, an algorithm or drawing a parse tree.\\

\defnbox{Algorithm for testing WFF}{To test whether or not an expression \emph{U} is a well formed formula in \emph{Form(L\textsuperscript{p})} we will follow this algorithm:
\begin{enumerate}
\item If the formula is empty, it is \textbf{not} a well formed formula.
\item If \emph{U} $\in$ \emph{Atom(L\textsuperscript{p})}, then it \textbf{is} a well formed formula.  If \emph{U} is any other single symbol, then it is \textbf{not} a well formed formula.
\item If \emph{U} contains more than one symbol and it does not start with "(", then it is \textbf{not} a well formed formula.
\item If the second symbol is $\lnot$, \emph{U} must be ($\lnot$\emph{V}) where \emph{V} is an expression.  Otherwise it is \textbf{not} a well formed formula.  Apply the same algorithm to \emph{V}.
\item If \emph{U} begins with "(" but the second symbol is not $\lnot$, scan from left to right until a \emph{V} segment is found where \emph{V} is a proper sub-expression.  \emph{U} must be of the form (\emph{V} * \emph{W}) where \emph{W} is also a proper sub-expression.  Otherwise it is \textbf{not} a well formed formula.  Apply the same algorithm recursively to \emph{V} and \emph{W}.
\end{enumerate}}

Feel free to look up how to draw a parse tree in the course slides, it is not very difficult.\\

Let's now introduce a new type of induction which will come in handy for proving theorems related to formulas in \emph{Form(L\textsuperscript{p})}.\\

\defnbox{Structural Induction}{In order to prove properties of propositional formulas, we apply induction on the \textbf{height} of the parse tree.  This proof technique is called \textbf{structural induction}.}

I will not provide a worked out example for this here, but the general idea for proving this is quite simple, and once you understand you understand.\\

\begin{enumerate}
\item The first step is to prove a base case.  Generally a small one.
\item The next step is to assume it works for a tree of height \emph{n}.
\item Finally in the inductive step, we prove that it works for a tree of height \emph{n} + 1.  To do this, we will show that if the \emph{n} + 1 connective is a unary or binary, the inductive step will still hold.
\end{enumerate}

\defnbox{Scopes}{We have two types of scopes, one for unary and one for binary connectives.
\begin{itemize}
\item If ($\lnot$\emph{A}) is a segment of \emph{C}, then \emph{A} is called the \textbf{scope} in \emph{C} of the $\lnot$ on the left of \emph{A}.
\item If (\emph{A} * \emph{B}) is a segment of \emph{C}, then \emph{A} and \emph{B} are called the left and right \textbf{scopes} in \emph{C} of the * between \emph{A} and \emph{B}.
\end{itemize}}

\newpagesubsection{Semantics of propositional logic}