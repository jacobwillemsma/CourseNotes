\documentclass[12pt, twoside]{article}
\usepackage[left=28mm, top=24mm, right=28mm, bottom=24mm, asymmetric, reversemarginpar]{geometry}
\usepackage{titlesec}
\usepackage{marginnote}
\usepackage{listings}
\usepackage{color}
\usepackage{xkeyval}
\usepackage{varwidth}
\usepackage{microtype}
\usepackage{hyperref}
\usepackage{enumerate}
\title{\textbf{CLAS202 Review Notes}}
% Code box.
\lstnewenvironment{code}[1][]
	{\begingroup
		%\vfil\penalty-9999\vfilneg\lstset{language=#1}
		\lstset{language=#1}
	}
	{\endgroup}

% Definition box.
\newcommand{\defnbox}[2] {
	\setlength{\fboxsep}{8pt}
	\marginpar {
		\vspace{0.9em}
		\begin{center}
		\footnotesize{\textbf{\color{brown}DEFINITION}}
		\footnotesize{\textbf{#1}}
		\end{center}
	}
	\colorbox{lightyellow}{
		\begin{minipage}{\dimexpr\linewidth-2\fboxsep}
		#2
		\end{minipage}
	}
	~\\
}

% Example box.
\newcommand{\exbox}[2] {
	\setlength{\fboxsep}{8pt}
	\marginpar {
		\vspace{0.9em}
		\footnotesize{\textbf{\color{darkpurple}EXAMPLE #1}}
	}
	\colorbox{lightpurple}{
		\begin{minipage}{\dimexpr\linewidth-2\fboxsep}
		#2
		\end{minipage}
	}
	~\\
}

% Exercise box.
\newcommand{\exerbox}[1] {
	\setlength{\fboxsep}{8pt}
	\marginpar {
		\vspace{0.9em}
		\footnotesize{\textbf{\color{darkred}EXERCISE}}
	}
	\colorbox{lightred}{
		\begin{minipage}{\dimexpr\linewidth-2\fboxsep}
		#1
		\end{minipage}
	}
	~\\
}

% Used on the side for definitions.
\definecolor{brown}{RGB}{101, 91, 71}
\definecolor{lightyellow}{RGB}{228, 224, 128}

% Used for the code block itself.
\definecolor{codebg}{RGB}{255, 255, 238}
\definecolor{codeborder}{RGB}{243, 242, 222}

% Used for exercises.
\definecolor{darkred}{RGB}{203, 20, 20}
\definecolor{lightred}{RGB}{229, 130, 130}

% Used for examples.
\definecolor{darkpurple}{RGB}{76, 60, 189}
\definecolor{lightpurple}{RGB}{184, 183, 255}

% Used for C and Lisp Syntax.
\definecolor{purple}{RGB}{174, 19, 198}
\definecolor{darkblue}{RGB}{0, 0, 102}
\definecolor{lightblue}{RGB}{50, 155, 171}
\definecolor{lightgreen}{RGB}{29, 131, 43}

% Document formatting for headings.
\pagestyle{myheadings}
\setcounter{secnumdepth}{4} % 4 being sub sections.

% Removes indentation of paragraphs.
\setlength{\parindent}{0cm}

% Sets page numbering to roman.
\pagenumbering{roman}

% Declaring the default listing style.
\lstdefinestyle{default_style} {
	backgroundcolor=\color{codebg},
	rulecolor=\color{codeborder},
	stringstyle=\color{purple},
	keywordstyle=\color{darkblue},
	identifierstyle=\color{lightblue},
	commentstyle=\color{lightgreen},
	basicstyle=\footnotesize\sffamily,
	xleftmargin=10pt,
	xrightmargin=10pt,
	belowcaptionskip=10pt,
	belowskip=20pt,
	framesep=10pt,
	frame=single,
	%numbers=left,
	%numbersep=8pt,
	showspaces=false,
	showstringspaces=false,
	tabsize=2
}

% Sets the default style for all code blocks.
\lstset {
	style=default_style
}

% Module section shortcut commands.
\newcommand{\newpagesection}[1] {
	\clearpage
	\section{#1}
}

\newcommand{\newpagesubsection}[1] {
	\clearpage
	\subsection{#1}
}
\begin{document}
\makeatletter
\hfil\parbox[t]{0.7\textwidth}{\centering\LARGE\bfseries\@title}\par
\kern0.5cm \hrule\kern0.5cm
\makeatother

% Table of contents
\renewcommand{\contentsname}{Table of Contents}
\tableofcontents
\clearpage

% Content
\pagenumbering{arabic}
\setlength{\oddsidemargin}{1.6cm}
\setlength{\evensidemargin}{\oddsidemargin}
\setlength{\marginparwidth}{2.6cm}
\setlength{\marginparsep}{0.25cm}

Welcome to CLAS202 - Ancient Roman Society.  This course will have a particular focus on the earlier empire of Rome.  This course also has something for everyone.  It will touch on the architechture, culture, emperors, mathematicians, artists, the art the entertainment the decadence and everything in between of Roman society. \\

We will begin with how we have found out so much about this ancient society. \\

\section{Sources for Roman Society}

There is truly a staggering amount of content from ancient Rome.  So much so that when Rome built their subway they had to continuously push back the completion date because as soon as they dug down they found all sorts of neat, Roman, artifacts. \\

\subsection{Literature}
\begin{itemize}
\item Massive amount of literature
	\begin{itemize}
	\item on papyrus (in Egypt and Herculaneum Italy).
	\item on parchment (Dead Sea Scrolls)
	\item recopied by monks in the Middle Ages.
	\end{itemize}
\item Includes historians, philosophers, geographers, poets, politician's speeches and propaganda, letters, biographies and even encyclopedias.
\item Lots was burned. lost or changed by Christians.
\item Lots of Manuscripts.
\item These guys literally wrote everything.
\end{itemize}

\subsection{Art and Artifacts}
\begin{itemize}
\item Sculptures (thousands).
\item Paintings.
\item Architecture.
\item Daily life (buildings).
\item Roman toys.
\item Pottery
\item So much content here we have not finished getting through it all.
\end{itemize}

\subsection{Inscriptions}
\begin{itemize}
\item On stone or metal.
\item Basically invented Graffiti (\emph{graffito} = a message scratched or painted on a wall).
\item A house was literally a blank canvas, people wrote everything, everywhere. 
\end{itemize}

\subsection{Administrative}
\begin{itemize}
\item Records on papyrus.
\item Censuses.
\item Coins
	\begin{itemize}
	\item Excellent help for finding out who was the emperor and when.
	\item Coin dyes are an awesome find.
	\end{itemize}
\end{itemize}

\subsection{What we don't find}
\begin{itemize}
\item We don't find graves.  Does likely to the fact that the Romans cremated their dead.
\item Occasionally we find a body and get excited, we pull conclusions (is this a good idea? Probably not).
\end{itemize}

\section{Geographical Background}
\begin{itemize}
\item The Roman world is the Mediterranean World.
	\begin{itemize}
	\item 7600km of coastline.
	\item 4 times the size of Greece (but still smaller than Newfoundland - very small).
	\item Mediterranean triad (grain, olives and grapes)... incredibly fertile land.
	\end{itemize}
\item Italian mountain ranges and rivers:
	\begin{itemize}
	\item Alps to the north of Italy, forming a natural border.
	\item Apennines down the center, making east/west travel a little difficult.
	\item Po river in the north.
	\item Tiber river at Rome.  Rome has a natural crossing making is an ideal place for travel and merchanting.
	\end{itemize}
\item Neighbors:
	\begin{itemize}
	\item Celts north of the Po (barbarians).
	\item Greeks settling in southern Italy.
		\begin{itemize}
		\item $6^{th}$ century BC Greeks move to the ``toe'' of Italy due to civil wars and over population.
		\end{itemize}
	\item Latium (the plains surrounding Rome)
	\item Rome built on 7 hills.  Capitoline is the highest and is where the stronghold is built.
	\end{itemize}
\end{itemize}

\section{Early Italy}
\subsection{Beginnings}
\begin{itemize}
\item Urnfield culture in 1800BC (put their dead in urns).
\item Villanovans in 1000-750BC.
\item Mostly wattle and daub houses (twigs covered with mud).
\item The center-most hill of Rome is settled (Palatine).
\item Later on the Fossa People (buried their dead in trenches).
\item Magna Graecia (southern Italy settled by Greeks).
\item No need to fight, plenty of resources to go around.
\end{itemize}

\subsection{The Etruscans}
\begin{itemize}
\item 900-800BC Etruscans.
	\begin{itemize}
	\item Herodotus says from the Near East.
	\item Lived in North-West Italy.
	\item Language unknown.
	\item Famed as town planners.
		\begin{itemize}
		\item Built towns in rectangle shape with roads crossing North/South and East/West.
		\end{itemize}
	\item Devoutly religious.
		\begin{itemize}
		\item Three gods, Jupiter, Juno and Minerva.
		\end{itemize}
	\item Mudbrick houses.
	\item \textbf{Necropolis}, burial in decorated tombs arranged,
	\item Bucchero (black pottery), sold all around the Mediterranean, excellent merchants.
	\item Fine metal worker and craftsmen in terracotta.
	\end{itemize}
\item 550BC expansion into the land around them, beginning of Hellenistic (after death of Alexander the Great, formation of Roman Empire) phase.
\item Emperor Claudius (41-54AD) was the last Etruscan speaker.
	\begin{itemize}
	\item Historian.
	\item Wrote 20 books on Etruscan History.
	\item All Etruscan literature lost.
	\end{itemize}
\end{itemize}

\section{The Monarchy (753-510BC)}
\subsection{The Founding of Rome}

\begin{itemize}
\item Founding of Rome was likely very boring, probably just farmers on a hilltop who eventually began a town and then a city.
\item However, to assert the divine creation of Rome, myth is invented.
\item So the story goes:
	\begin{itemize}
	\item A Vestal Virgin is impregnated by Mars and gives birth to \textbf{Romulus and Remus}.
	\item Her brother is upset and wants to kill the children.
	\item Like any good mother, she puts the children in a wicker basket and sends them down the Tiber.
	\item They are taken in by a \emph{She-wolf} who raises them as her own.
	\item Romulus eventually in his adulthood kills Remus.
	\item Romulus becomes the first Etruscan king of Rome (7 total - divine number).
	\item Rome officially founded April 21, 8:05 AM, \textbf{753 BCE}.
	\item \textbf{Pomoerium} is the ``sacred'' boundary.
	\item At first there is only men in Rome, so the Romans arrange a party for their neighbors and once they are drunk, steal and rape their women.
		\begin{itemize} \item Raped women love Rome so much, they stay. \end{itemize}
	\end{itemize}
\end{itemize}

\subsection{Kings}
\begin{itemize}
\item Each king (\textbf{Rex}) has two \textbf{Lictors} which are attendants of the King.  Later they become magistrates (judges).
\item The Lictors carried \textbf{Fasces}, bundles of rods and axes.  Often seen during victory parades, which happened pretty often.
\end{itemize}

\subsection{Classes}
\begin{itemize}
\item The people of early Rome had a very specific class system, broken down into two categories:
	\begin{itemize}
	\item Patricians
		\begin{itemize}
		\item Social upper class.
		\item Make up 10\% - 20\%
		\end{itemize}
	\item Plebeians
		\begin{itemize}
		\item Social lower class.
		\item Make up 80\% - 90\%
		\end{itemize}
	\end{itemize}
\item \textbf{Gentes} (the family clan) became very important.
	\begin{itemize}
	\item Your name was a compound name.
	\item Given Name + Clan Name + Family Name
	\end{itemize}
\item \textbf{Curiate Assembly} was formed, 10 for every tribe (10 x 3).  In charge of voting ``democratically'' (only Patricians could vote).
\item Each tribe provided \textbf{Centuries} for Rome.


\end{itemize}







\end{document}