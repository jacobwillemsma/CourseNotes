\documentclass[12pt, twoside]{article}
\usepackage[left=28mm, top=24mm, right=28mm, bottom=24mm, asymmetric, reversemarginpar]{geometry}
\usepackage{titlesec}
\usepackage{marginnote}
\usepackage{listings}
\usepackage{color}
\usepackage{xkeyval}
\usepackage{varwidth}
\usepackage{microtype}
\usepackage{hyperref}
\usepackage{enumerate}
\title{\textbf{CLAS202 Review Notes}}
% Code box.
\lstnewenvironment{code}[1][]
	{\begingroup
		%\vfil\penalty-9999\vfilneg\lstset{language=#1}
		\lstset{language=#1}
	}
	{\endgroup}

% Definition box.
\newcommand{\defnbox}[2] {
	\setlength{\fboxsep}{8pt}
	\marginpar {
		\vspace{0.9em}
		\begin{center}
		\footnotesize{\textbf{\color{brown}DEFINITION}}
		\footnotesize{\textbf{#1}}
		\end{center}
	}
	\colorbox{lightyellow}{
		\begin{minipage}{\dimexpr\linewidth-2\fboxsep}
		#2
		\end{minipage}
	}
	~\\
}

% Example box.
\newcommand{\exbox}[2] {
	\setlength{\fboxsep}{8pt}
	\marginpar {
		\vspace{0.9em}
		\footnotesize{\textbf{\color{darkpurple}EXAMPLE #1}}
	}
	\colorbox{lightpurple}{
		\begin{minipage}{\dimexpr\linewidth-2\fboxsep}
		#2
		\end{minipage}
	}
	~\\
}

% Exercise box.
\newcommand{\exerbox}[1] {
	\setlength{\fboxsep}{8pt}
	\marginpar {
		\vspace{0.9em}
		\footnotesize{\textbf{\color{darkred}EXERCISE}}
	}
	\colorbox{lightred}{
		\begin{minipage}{\dimexpr\linewidth-2\fboxsep}
		#1
		\end{minipage}
	}
	~\\
}

% Used on the side for definitions.
\definecolor{brown}{RGB}{101, 91, 71}
\definecolor{lightyellow}{RGB}{228, 224, 128}

% Used for the code block itself.
\definecolor{codebg}{RGB}{255, 255, 238}
\definecolor{codeborder}{RGB}{243, 242, 222}

% Used for exercises.
\definecolor{darkred}{RGB}{203, 20, 20}
\definecolor{lightred}{RGB}{229, 130, 130}

% Used for examples.
\definecolor{darkpurple}{RGB}{76, 60, 189}
\definecolor{lightpurple}{RGB}{184, 183, 255}

% Used for C and Lisp Syntax.
\definecolor{purple}{RGB}{174, 19, 198}
\definecolor{darkblue}{RGB}{0, 0, 102}
\definecolor{lightblue}{RGB}{50, 155, 171}
\definecolor{lightgreen}{RGB}{29, 131, 43}

% Document formatting for headings.
\pagestyle{myheadings}
\setcounter{secnumdepth}{4} % 4 being sub sections.

% Removes indentation of paragraphs.
\setlength{\parindent}{0cm}

% Sets page numbering to roman.
\pagenumbering{roman}

% Declaring the default listing style.
\lstdefinestyle{default_style} {
	backgroundcolor=\color{codebg},
	rulecolor=\color{codeborder},
	stringstyle=\color{purple},
	keywordstyle=\color{darkblue},
	identifierstyle=\color{lightblue},
	commentstyle=\color{lightgreen},
	basicstyle=\footnotesize\sffamily,
	xleftmargin=10pt,
	xrightmargin=10pt,
	belowcaptionskip=10pt,
	belowskip=20pt,
	framesep=10pt,
	frame=single,
	%numbers=left,
	%numbersep=8pt,
	showspaces=false,
	showstringspaces=false,
	tabsize=2
}

% Sets the default style for all code blocks.
\lstset {
	style=default_style
}

% Module section shortcut commands.
\newcommand{\newpagesection}[1] {
	\clearpage
	\section{#1}
}

\newcommand{\newpagesubsection}[1] {
	\clearpage
	\subsection{#1}
}
\begin{document}
\makeatletter
\hfil\parbox[t]{0.7\textwidth}{\centering\LARGE\bfseries\@title}\par
\kern0.5cm \hrule\kern0.5cm
\makeatother

% Table of contents
\renewcommand{\contentsname}{Table of Contents}
\tableofcontents
\clearpage

% Content
\pagenumbering{arabic}
\setlength{\oddsidemargin}{1.6cm}
\setlength{\evensidemargin}{\oddsidemargin}
\setlength{\marginparwidth}{2.6cm}
\setlength{\marginparsep}{0.25cm}

Welcome to CLAS202 - Ancient Roman Society.  This course will have a particular focus on the earlier empire of Rome.  This course also has something for everyone.  It will touch on the architechture, culture, emperors, mathematicians, artists, the art the entertainment the decadence and everything in between of Roman society. \\

We will begin with how we have found out so much about this ancient society. \\

\section{Sources for Roman Society}

There is truly a staggering amount of content from ancient Rome.  So much so that when Rome built their subway they had to continuously push back the completion date because as soon as they dug down they found all sorts of neat, Roman, artifacts. \\

\subsection{Literature}
\begin{itemize}
\item Massive amount of literature
	\begin{itemize}
	\item on papyrus (in Egypt and Herculaneum Italy).
	\item on parchment (Dead Sea Scrolls)
	\item recopied by monks in the Middle Ages.
	\end{itemize}
\item Includes historians, philosophers, geographers, poets, politician's speeches and propaganda, letters, biographies and even encyclopedias.
\item Lots was burned. lost or changed by Christians.
\item Lots of Manuscripts.
\item These guys literally wrote everything.
\end{itemize}

\subsection{Art and Artifacts}
\begin{itemize}
\item Sculptures (thousands).
\item Paintings.
\item Architecture.
\item Daily life (buildings).
\item Roman toys.
\item Pottery
\item So much content here we have not finished getting through it all.
\end{itemize}

\subsection{Inscriptions}
\begin{itemize}
\item On stone or metal.
\item Basically invented Graffiti (\emph{graffito} = a message scratched or painted on a wall).
\item A house was literally a blank canvas, people wrote everything, everywhere. 
\end{itemize}

\subsection{Administrative}
\begin{itemize}
\item Records on papyrus.
\item Censuses.
\item Coins
	\begin{itemize}
	\item Excellent help for finding out who was the emperor and when.
	\item Coin dyes are an awesome find.
	\end{itemize}
\end{itemize}

\subsection{What we don't find}
\begin{itemize}
\item We don't find graves.  Does likely to the fact that the Romans cremated their dead.
\item Occasionally we find a body and get excited, we pull conclusions (is this a good idea? Probably not).
\end{itemize}

\section{Geographical Background}
\begin{itemize}
\item The Roman world is the Mediterranean World.
	\begin{itemize}
	\item 7600km of coastline.
	\item 4 times the size of Greece (but still smaller than Newfoundland - very small).
	\item Mediterranean triad (grain, olives and grapes)... incredibly fertile land.
	\end{itemize}
\item Italian mountain ranges and rivers:
	\begin{itemize}
	\item Alps to the north of Italy, forming a natural border.
	\item Apennines down the center, making east/west travel a little difficult.
	\item Po river in the north.
	\item Tiber river at Rome.  Rome has a natural crossing making is an ideal place for travel and merchanting.
	\end{itemize}
\item Neighbors:
	\begin{itemize}
	\item Celts north of the Po (barbarians).
	\item Greeks settling in southern Italy.
		\begin{itemize}
		\item $6^{th}$ century BC Greeks move to the ``toe'' of Italy due to civil wars and over population.
		\end{itemize}
	\item Latium (the plains surrounding Rome)
	\item Rome built on 7 hills.  Capitoline is the highest and is where the stronghold is built.
	\end{itemize}
\end{itemize}

\section{Early Italy}
\subsection{Beginnings}
\begin{itemize}
\item Urnfield culture in 1800BC (put their dead in urns).
\item Villanovans in 1000-750BC.
\item Mostly wattle and daub houses (twigs covered with mud).
\item The center-most hill of Rome is settled (Palatine).
\item Later on the Fossa People (buried their dead in trenches).
\item Magna Graecia (southern Italy settled by Greeks).
\item No need to fight, plenty of resources to go around.
\end{itemize}

\subsection{The Etruscans}
\begin{itemize}
\item 900-800BC Etruscans.
	\begin{itemize}
	\item Herodotus says from the Near East.
	\item Lived in North-West Italy.
	\item Language unknown.
	\item Famed as town planners.
		\begin{itemize}
		\item Built towns in rectangle shape with roads crossing North/South and East/West.
		\end{itemize}
	\item Devoutly religious.
		\begin{itemize}
		\item Three gods, Jupiter, Juno and Minerva.
		\end{itemize}
	\item Mudbrick houses.
	\item \textbf{Necropolis}, burial in decorated tombs arranged,
	\item Bucchero (black pottery), sold all around the Mediterranean, excellent merchants.
	\item Fine metal worker and craftsmen in terracotta.
	\end{itemize}
\item 550BC expansion into the land around them, beginning of Hellenistic (after death of Alexander the Great, formation of Roman Empire) phase.
\item Emperor Claudius (41-54AD) was the last Etruscan speaker.
	\begin{itemize}
	\item Historian.
	\item Wrote 20 books on Etruscan History.
	\item All Etruscan literature lost.
	\end{itemize}
\end{itemize}

\section{The Monarchy (753-510BC)}
\subsection{The Founding of Rome}

\begin{itemize}
\item Founding of Rome was likely very boring, probably just farmers on a hilltop who eventually began a town and then a city.
\item However, to assert the divine creation of Rome, myth is invented.
\item So the story goes:
	\begin{itemize}
	\item A Vestal Virgin is impregnated by Mars and gives birth to \textbf{Romulus and Remus}.
	\item Her brother is upset and wants to kill the children.
	\item Like any good mother, she puts the children in a wicker basket and sends them down the Tiber.
	\item They are taken in by a \emph{She-wolf} who raises them as her own.
	\item Romulus eventually in his adulthood kills Remus.
	\item Romulus becomes the first Etruscan king of Rome (7 total - divine number).
	\item Rome officially founded April 21, 8:05 AM, \textbf{753 BCE}.
	\item \textbf{Pomoerium} is the ``sacred'' boundary.
	\item At first there is only men in Rome, so the Romans arrange a party for their neighbors and once they are drunk, steal and rape their women.
		\begin{itemize} \item Raped women love Rome so much, they stay. \end{itemize}
	\end{itemize}
\end{itemize}

\subsection{Kings}
\begin{itemize}
\item Each king (\textbf{Rex}) has two \textbf{Lictors} which are attendants of the King.  Later they become magistrates (judges).
\item The Lictors carried \textbf{Fasces}, bundles of rods and axes.  Often seen during victory parades, which happened pretty often.
\end{itemize}

\subsection{Classes}
\begin{itemize}
\item The people of early Rome had a very specific class system, broken down into two categories:
	\begin{itemize}
	\item Patricians
		\begin{itemize}
		\item Social upper class.
		\item Make up 10\% - 20\%
		\end{itemize}
	\item Plebeians
		\begin{itemize}
		\item Social lower class.
		\item Make up 80\% - 90\%
		\end{itemize}
	\end{itemize}
\item \textbf{Gentes} (the family clan) became very important.
	\begin{itemize}
	\item Your name was a compound name.
	\item Given Name + Clan Name + Family Name
	\end{itemize}
\item \textbf{Curiate Assembly} was formed, 10 for every tribe (10 x 3).  In charge of voting ``democratically'' (only Patricians could vote).
\item Each tribe provided \textbf{Centuries} for Rome.
\end{itemize}

\section{The Early Republic (509-264BC)}
\begin{itemize}
\item 510/509 BC expulsion of Etruscan kings.
\item Romans date this as 244 a.u.c. (\emph{ab urbe condita = the the foundation of the city}).
	\begin{itemize}
	\item 244 + 509 = 753BC
	\end{itemize}
\item \textbf{Res publica}, republic, for the people.
\item 2 consuls (cheif magistrates)
	\begin{itemize}
	\item Replace the Rex
	\end{itemize}
\item Dictator - 6 months maximum.  Only when issues arose and decisions had to be made, often used less than 6 months (too much power, hand it away quickly).
\item Patricians run the Senate.
\item 471BC Plebian Council
	\begin{itemize}
	\item Tribunes, representatives of the plebs.
	\end{itemize}
\item Twelve Tables
	\begin{itemize}
	\item Laws posted clearly on two bronze tablets.
	\item Speaks to the literacy level of the republic.
	\end{itemize}
\end{itemize}
\subsection{The Legend of Horatius Cocles}
\begin{itemize}
\item 509BC the Etruscan king Lars Porsemma of Clusium attacked Rome.
\item Horatius defended the pons Sublicius bridge.
	\begin{itemize}
	\item Cocles - one eyed.  Oddly, a good thing in Roman culture (blessed by gods)
	\end{itemize}
\item Defends the bridge singlehandedly as his friends cut down the bridge behind him.
\item At the last second jumps over and survives.
\end{itemize}
\subsection{A New City Defends Itself}
\begin{itemize}
\item 493BC Latin League
	\begin{itemize}
	\item Allies with Latin tribes around them to protect against the Etruscans.
	\item Rome gets between the fights of the Latin tribes, help in fights and defeat other tribes and makes them allies and eventually Roman.
	\item Slowly Rome grows and has no enemies. (Divide and conquer)
	\end{itemize}
\item 480-396BC Veii, closest Etruscan city to Rome.
	\begin{itemize}
	\item After defeating these guys though, they kill everyone.
	\end{itemize}
\item Gaul: A territory north of the Apennine mountains in modern day France.
	\begin{itemize}
	\item Taller on average, blonde or red-haired.
	\item Huge populations are armies.
	\item Heroic warfare still important.
	\item Fanatics would fight naked.
	\end{itemize}
\end{itemize}

\subsection{The Sack of Rome}
\begin{itemize}
\item 390/387 sace of Rome.  Brennus, Cheiftain of the Senones.
\item \textbf{Vae victis}.  Woe to the Vanquished.  Sucks to lose.
	\begin{itemize}
	\item 1000 pounds of gold ransom 
	\end{itemize}
\item Capitol (Capitoline Hill) is not taken.
\item Romans take advantage of the Barbarians sack of Etruscan villages on the way to Rome, following and finishing the job.
\end{itemize}

\subsection{Expanding North}
\begin{itemize}
\item Rome expands North following the retreat of the Gauls.
\item “Servian” Wall (really dates to 380's, not Servius Tullius) built
\item ager publicus ( land belonging to the state)
\item colonies (veteran settlement in captured territories)
	\begin{itemize}
	\item Keep an eye on things (well trained military)
	\end{itemize}
\end{itemize}

\subsection{Samnites}
\begin{itemize}
\item Italic herdsmen, lived in mountains.
\item Huge families, bred like rabbits, threaten to swamp Italy.
\item Mobile experts at mountain and rough ground fighting.
\item Samnite Wars (343-290BC)
\item 321BC Caudine Forks: Colossal loss for Rome.
	\begin{itemize}
	\item Pass beneath the Yolk - insult and embarrass the whole army and Romans.
	\item Refuse peace treaty, give the two generals instead.  Bad luck for the Samnites to accept the gift.
	\item Angers Rome more and more and they decided they need to work harder on defeating these guys.
	\end{itemize} 
\item Via Appia: Fortified road from Rome to Campania.
	\begin{itemize}
	\item Speed, communication and supplies.
	\end{itemize}
\item Eventually absorb the Samnites into the Roman empire.
\end{itemize}

\subsection{The Pyrrhic Wars}
\begin{itemize}
\item Tarentum
	\begin{itemize}
	\item Major Greek city state in southern Italy
	\item Threatened by Italic Sabelline trines to their north. 
	\end{itemize}
\item Tarentum calls on King Pyrrhic (Greek - Alexander the Great's Cousin) for aid.
\item Sabines call on Rome for aid.
\item 280-275BC Wars
	\begin{itemize}
	\item Pyrrhus brings 25000 pikemen and war elephants.
	\item First time Romans see Elephants.
	\item Wins three battles and leaves. Was not expecting to see the Romans, could not afford to lose key soldiers to them.
	\end{itemize}
\item 264BC Rome is the \textbf{Domina} of central and southern Italy.  Can call up to 700000 troops if needed.
\end{itemize}

\section{Government}
\begin{itemize}
\item S.P.Q.R (\emph{The Senate and Roman People})
\item Senate (Aristocratic, old Patrician families)
	\begin{itemize}
	\item Major legislation and advise consults.
	\item Foreign policy
	\item Senatus consultum (\emph{decree of the Senate})
	\item Should be of strong moral character. 
	\end{itemize}
\end{itemize}

\subsection{Three popular Asemblies}
\begin{itemize}
\item Curiate Assembly
\item Centuriate Assembly
	\begin{itemize}
	\item Contains Plebs and Patricians
	\item Majority voting power is in the patricians favor. 
	\end{itemize}
\item Tribal Assembly
	\begin{itemize}
	\item 35 tribes, 4 in Rome, 31 in country.
	\item Elect lower magistrates (Quaestor and Aediles) and the 10 Tribunes of the Plebs.
	\item Plebian Council (471BC)
	\item 287 BCE the Lex Hortensia made the \textbf{plebiscite} (decision of the plebs) law.
	\item Magistrates:
		\begin{itemize}
		\item Cursus Honorum (starts at age 30, senatorial career pattern)
		\item Quaestor (4, eventually 20), financial, inluding provincial treasurer.
		\item Aediles (4) - in charge of streets, markets, festivals and public works. 
		\end{itemize}
	\item Praetor (8)
		\begin{itemize}
		\item In charge of public law courts or governors.
		\item Held the power of a lesser Consul.
		\item Should be at least 39 years old and have serves as a Quaestor
		\end{itemize}
	\item Consul(2)
		\begin{itemize}
		\item Chief magistrate, with legal and military power.
		\item replaced the Etruscan kings
		\item commanded the armies of Rome
		\item must be at least 42 years old
		\item each could veto (meaning “I forbid”) the other
		\item 367 BCE law requires one of the Consuls to be a Plebeian
		\end{itemize}
	\item Censor (2, every 5 years for an 18 month term)
	\item Tribune (10)
		\begin{itemize}
		\item represent plebs
		\item sacrosanct man of god.  Cannot be persecuted by anyone.
		\item veto
		\end{itemize}
	\item Dictator (1)
		\begin{itemize}
		\item dictator re gerundae causa (‘dictator to do what needs to be done’)
		\item only in emergency
		\item only for 6 months maximum
		\item limitless power to safeguard the state 
		\end{itemize}
	\item Lictors (2 - same as before)
	\item Triumph
		\begin{itemize}
		\item legal wars that were won and resulted in at least 5,000 enemy dead required a Triumph
		\item Victorious General, in his best clothes and armour, with his face painted purple, was paraded through Rome
		\item accompanied by soldiers, captives and spoils of war
		\item procession ended at the Temple of Jupiter Optimus Maximus on the Capitoline Hill
		\item general made sacrifices to Jupiter for the victory
		\end{itemize}
	\item Ovation
		\begin{itemize}
		\item lesser victories (fewer enemy dead or against lesser enemies, such as slaves) received an Ovation
		\item an honour, but less impressive procession and celebration
		\end{itemize}
	\end{itemize}	
\end{itemize}

\section{Republican Ideals}
\begin{itemize}
\item mos maiorum (ancestral customs, respect and emilate ancestral traditions)
\item gravitas (seriousness - self control) 
\item pietas (respect for authority to the gods, state and family)
\item religio (being “bound” to the gods by acting the way you should)
\item virtus (manliness, courage)
\item fides (loyalty, faithfulness, honesty, integrity)
\item simplicitas (plain lifestyle)
\item clementia (calculated mercy)
\item frugalitas (frugality)
\end{itemize}

\section{Family Life}
\begin{itemize}
\item familia (family)
\item Differences between Roman and “modern” families
	\begin{itemize}
	\item extended family, including dependent children and slaves
	\item many children lost at least one parent by age 15
	\end{itemize}
\end{itemize}

\subsection{Paterfamilias}
\begin{itemize}
\item paterfamilias (male head of the family)
\item patria potestas (authority of the paterfamilias)
	\begin{itemize}
	\item can expose unwanted children, or give away/abandon to others
	\item adultery laws of 18 BCE allows father to kill daughter and seducer if “caught in the act” in his own home
	\end{itemize}
\item genius (protective spirit)
\end{itemize}

\subsection{Matrona}
\begin{itemize}
\item matrona (wife of the paterfamilias)
	\begin{itemize}
	\item virtuous \& strong
	\item devoted to the education and advancement of her family
	\item self sacrificing
	\item run household and slaves
	\item make and craft with wool
	\item many wives and stepmothers due to high mortality (and divorce among Patrician class)
	\end{itemize}
\end{itemize}

\subsection{Women}
\begin{itemize}
\item bias of our evidence (written by men for men)
\item role of women:
	\begin{itemize}	
	\item biological (childbirth, sex)
	\item economic (dowry, household management, labour, wool-working)
	\item supervise slaves, children
	\end{itemize}
\item high moral standard expected (otherwise could be killed)
\item little involvement in public life (service to emperor or deity)
\item demonstration against Oppian Law on luxury (195 BC)
\item Notable women:
	\begin{itemize}
	\item Cornelia (mother of the Gracchi)
	\item Laelia, Hortensia (orators, great public speakers)
	\item Iaia of Cyzicus (painter)
	\item Theophila (philosopher-poet, compared with Sappho)
	\item Hypatia (philosopher-mathematician, in Alexandria until bishop thought was pagan and she was killed)
	\item Demo (commentator on Homer)
	\item criticism of women: Juvenal's 6th satire
	\item praise of women: Quintilian; eulogy of Turia
	\end{itemize}
\item legal dependency: male control (father, husband, guardian)
	\begin{itemize}
	\item incl. exposure, arranged marriages
	\end{itemize}
\item double standard re. adultery, citizenship
\item home bodies, or party animals? e.g. Livy vs. Ovid; Sabine women;
\item Lucretia; Good Goddess; Papirius (all role models)
\item Patrician women do not work!
\item Most Plebeian women (low class) do work
\item women in work force (jobs attested in inscriptions, reliefs)
	\begin{itemize}
	\item dress maker
	\item hair dresser
	\item fish monger
	\item farmer
	\item taberna (bar) maid
	\item cottage industrie
	\item ”comfort girl” for shepherds
	\end{itemize}
\end{itemize}

\subsection{Children}
\begin{itemize}
\item (sources: Pliny the Elder, Lucretius, Soranus, Quintilian, Martial, Cicero, Plutarch)
\item Augustus' legislation to encourage children
	\begin{itemize}
	\item 9AD law giving priority to Consul with the most children
	\item women remarry within 1 year if widowed, or 6 months if divorced
	\item financial rewards for marriage \& children
	\item bachelor’s cannot inherit until they marry
	\item short engagements
	\end{itemize}
\item use of contraceptives, actually did.
\item strange ideas on mechanics of birth - didn't understand cycles - women are simply greenhouses for birth (plant the seed).
\item Miscarriages (common \& due to hysteria/pressure to have children)
\item Death from childbirth common
\item abortion (e.g. Domitian's niece) (not against not having children, but against the idea of getting an abortion to prevent stretch marks)
\item exposure by paterfamilias
\item Adoption (common and often necessary to provide an heir)
\item size of families (e.g. Germanicus, Marcus Aurelius)
\item illegitimate children
\item “Posthumous” (who's the father, add Posthumous at the end of a name)
\item treatment of children
\item alimenta (relief scheme for farmers and needy children) started by the Emperor Nerva - baby bonus.
\end{itemize}

\section{Republican Literature}
\begin{itemize}
\item no Latin literature until 3rd c. BC (too busy trying to live)
\item earliest forms are just copies of Greek originals translated into Latin
\item "Captive Greece captured her rude conqueror" (Horace)
\item Romans enjoyed many and variety forms of literature
\item “Golden Age” of Roman literature begins in the 1st Century BCE
\end{itemize}

\subsection{Lucius Livius Andronicus (284-204BC)}
\begin{itemize}
\item Greek from Tarentum
\item Greco-Roman dramatist and epic poet
\item Translated many Greek works into Latin
\item \textbf{“The “Father of Latin Literature”}
\item Most famous for his plays, and translation of Homer’s Odyssey into Latin
\end{itemize}

\subsection{Quintus Ennius (239-169BC)}
\begin{itemize}
\item \textbf{“The Father of Latin Poetry”}
\item Only fragments of his work survive, but his influence is very significant 
\item The Epicharmus discusses the nature of the gods, the universe, and heavenly enlightenment. 
\item The Annals is an epic poem of the history of Rome in verse, written in 18 books, covering the period from the fall of Troy in 1180 CE, to the Censorship of Cato the Elder in 184 BC
\item Writes history as poetry.
\end{itemize}

\subsection{Polybius (203-120 BC)}
\begin{itemize}
\item \textbf{Greek Historian}, soldier, general, statesman, and political hostage of Rome
\item Wrote a prose History of Rome, The Histories, covering the period 220-146 BCE
\item A bit biased
\item Believed that Historians must write from experience
\item First person accounts.
\end{itemize}

\subsection{Titus Maccius Plautus (254-184BC)}
\begin{itemize}
\item \textbf{Roman comedic playwright}
\item 21 of 130 plays survive (high rate!)
\item Rude, crude, low class and populist comedian
\item One of the first writers of musical theatre
\end{itemize}

\subsection{Publius Terentius (Terence) Aper (195-159BC)}
\begin{itemize}
\item \textbf{Comedic playwright}
\item Subtle humor.
\item Was brought to Rome as a slave by Terentius Lucanus, a  senator, was educated by him and then freed when his talent was recognized
\item All 6 of his plays survive
\item More refined than Plautus, but less funny (more intellectual)
\item Plagiarized others?
\item \textbf{“Fortune favors the brave”}
\item \textbf{“Where there is life there is hope”}
\item \textbf{“Each man to his own opinion”}
\end{itemize}

\subsection{Marcus Porcius Cato (234-149BC)}
\begin{itemize}
\item a Roman statesman, surnamed the Censor (Censorius), the Wise (Sapiens), the Ancient (Priscus), or the Elder (Maior)
\item \textbf{“Father of Latin Prose”}
\item wrote artistic prose
\item wrote on History, politics, agriculture and technical subjects
\item disliked aristocrats
\end{itemize}

\subsection{Gaius Lucilius (160’s-103/2BE)}
\begin{itemize}
\item Roman Equestrian
\item One of the earliest Roman \textbf{satirists} (the only literary form invented by the Romans)
\item Harsh critic of people, politicians and “foreigners”
\item Few fragments survive of his work
\end{itemize}

\subsection{Titus Lucretius Carus(c. 99-55BC)}
\begin{itemize}
\item Roman poet and Epicurean philosopher
\item Only known work is the epic poem \textbf{De Rerum Natura}, (“On the Nature of Things”)
	\begin{itemize}
	\item outlines his views on Epicurean philosophy in order to free people of the fear of the supernatural and death
	\end{itemize}
\end{itemize}

\subsection{Marcus Tullius Cicero (106–43BC)}
\begin{itemize}
\item Roman Equestrian, statesman,  Consul, philosopher, lawyer, orator and constitutionalist
\item Brilliant orator and prose writer
\item \textbf{De Re Publica} (“On The Republic”) and \textbf{De Legibus} (“On The Laws”)
\item Proponent of  “rights”, based on ancient law and custom. 
\item 6 books six on rhetoric, parts of eight on philosophy, and 58 speeches survive.
\end{itemize}

\subsection{Gaius Julius Caesar (100-44BC) (all important)}
\begin{itemize}
\item Roman General and statesman 
	\begin{itemize}
	\item considered one of the best orators and writers of Latin prose
	\item historical commentaries on Gallic Wars and Civil Wars
	\end{itemize}
\end{itemize}

\subsection{Gaius Sallustius (Sallust) Crispus (86-35BC)}
\begin{itemize}
\item \textbf{historian}, politician, and Novus Homo 
	\begin{itemize}
	\item supporter of Julius Caesar and opponent of Cicero
	\end{itemize}
\item \textbf{The Jugurthine War, Catiline Conspiracy and  Histories}  (fragments)
\item tried to show the connection and meaning of events, not just record them
\end{itemize}

\subsection{Gaius Valerius Catullus (84–54BC)}
\begin{itemize}
\item A  rich Equestrian from Cisalpine Gaul
\item Alexandrian school of lyric poetry
\item very explicit style 
	\begin{itemize}
	\item very popular with some, and despised by others , for being rude and amoral
	\end{itemize}
\item Influenced Ovid, Horace and Virgil
\item Lesbia poetry
\end{itemize}
\end{document}