\documentclass[12pt, twoside]{article}
\usepackage[left=28mm, top=24mm, right=28mm, bottom=24mm, asymmetric, reversemarginpar]{geometry}
\usepackage{titlesec}
\usepackage{marginnote}
\usepackage{listings}
\usepackage{color}
\usepackage{xkeyval}
\usepackage{varwidth}
\usepackage{microtype}
\usepackage{hyperref}
\usepackage{enumerate}
\title{\textbf{CLAS202 Review Notes}}
% Code box.
\lstnewenvironment{code}[1][]
	{\begingroup
		%\vfil\penalty-9999\vfilneg\lstset{language=#1}
		\lstset{language=#1}
	}
	{\endgroup}

% Definition box.
\newcommand{\defnbox}[2] {
	\setlength{\fboxsep}{8pt}
	\marginpar {
		\vspace{0.9em}
		\begin{center}
		\footnotesize{\textbf{\color{brown}DEFINITION}}
		\footnotesize{\textbf{#1}}
		\end{center}
	}
	\colorbox{lightyellow}{
		\begin{minipage}{\dimexpr\linewidth-2\fboxsep}
		#2
		\end{minipage}
	}
	~\\
}

% Example box.
\newcommand{\exbox}[2] {
	\setlength{\fboxsep}{8pt}
	\marginpar {
		\vspace{0.9em}
		\footnotesize{\textbf{\color{darkpurple}EXAMPLE #1}}
	}
	\colorbox{lightpurple}{
		\begin{minipage}{\dimexpr\linewidth-2\fboxsep}
		#2
		\end{minipage}
	}
	~\\
}

% Exercise box.
\newcommand{\exerbox}[1] {
	\setlength{\fboxsep}{8pt}
	\marginpar {
		\vspace{0.9em}
		\footnotesize{\textbf{\color{darkred}EXERCISE}}
	}
	\colorbox{lightred}{
		\begin{minipage}{\dimexpr\linewidth-2\fboxsep}
		#1
		\end{minipage}
	}
	~\\
}

% Used on the side for definitions.
\definecolor{brown}{RGB}{101, 91, 71}
\definecolor{lightyellow}{RGB}{228, 224, 128}

% Used for the code block itself.
\definecolor{codebg}{RGB}{255, 255, 238}
\definecolor{codeborder}{RGB}{243, 242, 222}

% Used for exercises.
\definecolor{darkred}{RGB}{203, 20, 20}
\definecolor{lightred}{RGB}{229, 130, 130}

% Used for examples.
\definecolor{darkpurple}{RGB}{76, 60, 189}
\definecolor{lightpurple}{RGB}{184, 183, 255}

% Used for C and Lisp Syntax.
\definecolor{purple}{RGB}{174, 19, 198}
\definecolor{darkblue}{RGB}{0, 0, 102}
\definecolor{lightblue}{RGB}{50, 155, 171}
\definecolor{lightgreen}{RGB}{29, 131, 43}

% Document formatting for headings.
\pagestyle{myheadings}
\setcounter{secnumdepth}{4} % 4 being sub sections.

% Removes indentation of paragraphs.
\setlength{\parindent}{0cm}

% Sets page numbering to roman.
\pagenumbering{roman}

% Declaring the default listing style.
\lstdefinestyle{default_style} {
	backgroundcolor=\color{codebg},
	rulecolor=\color{codeborder},
	stringstyle=\color{purple},
	keywordstyle=\color{darkblue},
	identifierstyle=\color{lightblue},
	commentstyle=\color{lightgreen},
	basicstyle=\footnotesize\sffamily,
	xleftmargin=10pt,
	xrightmargin=10pt,
	belowcaptionskip=10pt,
	belowskip=20pt,
	framesep=10pt,
	frame=single,
	%numbers=left,
	%numbersep=8pt,
	showspaces=false,
	showstringspaces=false,
	tabsize=2
}

% Sets the default style for all code blocks.
\lstset {
	style=default_style
}

% Module section shortcut commands.
\newcommand{\newpagesection}[1] {
	\clearpage
	\section{#1}
}

\newcommand{\newpagesubsection}[1] {
	\clearpage
	\subsection{#1}
}
\begin{document}
\makeatletter
\hfil\parbox[t]{0.7\textwidth}{\centering\LARGE\bfseries\@title}\par
\kern0.5cm \hrule\kern0.5cm
\makeatother

% Table of contents
\renewcommand{\contentsname}{Table of Contents}
\tableofcontents
\clearpage

% Content
\pagenumbering{arabic}
\setlength{\oddsidemargin}{1.6cm}
\setlength{\evensidemargin}{\oddsidemargin}
\setlength{\marginparwidth}{2.6cm}
\setlength{\marginparsep}{0.25cm}

Welcome to CLAS202 - Ancient Roman Society.  This course will have a particular focus on the earlier empire of Rome.  This course also has something for everyone.  It will touch on the architechture, culture, emperors, mathematicians, artists, the art the entertainment the decadence and everything in between of Roman society. \\

We will begin with how we have found out so much about this ancient society. \\

\section{Sources for Roman Society}

There is truly a staggering amount of content from ancient Rome.  So much so that when Rome built their subway they had to continuously push back the completion date because as soon as they dug down they found all sorts of neat, Roman, artifacts. \\

\subsection{Literature}
\begin{itemize}
\item Massive amount of literature
	\begin{itemize}
	\item on papyrus (in Egypt and Herculaneum Italy).
	\item on parchment (Dead Sea Scrolls)
	\item recopied by monks in the Middle Ages.
	\end{itemize}
\item Includes historians, philosophers, geographers, poets, politician's speeches and propaganda, letters, biographies and even encyclopedias.
\item Lots was burned. lost or changed by Christians.
\item Lots of Manuscripts.
\item These guys literally wrote everything.
\end{itemize}

\subsection{Art and Artifacts}
\begin{itemize}
\item Sculptures (thousands).
\item Paintings.
\item Architecture.
\item Daily life (buildings).
\item Roman toys.
\item Pottery
\item So much content here we have not finished getting through it all.
\end{itemize}

\subsection{Inscriptions}
\begin{itemize}
\item On stone or metal.
\item Basically invented Graffiti (\emph{graffito} = a message scratched or painted on a wall).
\item A house was literally a blank canvas, people wrote everything, everywhere. 
\end{itemize}

\subsection{Administrative}
\begin{itemize}
\item Records on papyrus.
\item Censuses.
\item Coins
	\begin{itemize}
	\item Excellent help for finding out who was the emperor and when.
	\item Coin dyes are an awesome find.
	\end{itemize}
\end{itemize}

\subsection{What we don't find}
\begin{itemize}
\item We don't find graves.  Does likely to the fact that the Romans cremated their dead.
\item Occasionally we find a body and get excited, we pull conclusions (is this a good idea? Probably not).
\end{itemize}

\section{Geographical Background}
\begin{itemize}
\item The Roman world is the Mediterranean World.
	\begin{itemize}
	\item 7600km of coastline.
	\item 4 times the size of Greece (but still smaller than Newfoundland - very small).
	\item Mediterranean triad (grain, olives and grapes)... incredibly fertile land.
	\end{itemize}
\item Italian mountain ranges and rivers:
	\begin{itemize}
	\item Alps to the north of Italy, forming a natural border.
	\item Apennines down the center, making east/west travel a little difficult.
	\item Po river in the north.
	\item Tiber river at Rome.  Rome has a natural crossing making is an ideal place for travel and merchanting.
	\end{itemize}
\item Neighbors:
	\begin{itemize}
	\item Celts north of the Po (barbarians).
	\item Greeks settling in southern Italy.
		\begin{itemize}
		\item $6^{th}$ century BC Greeks move to the ``toe'' of Italy due to civil wars and over population.
		\end{itemize}
	\item Latium (the plains surrounding Rome)
	\item Rome built on 7 hills.  Capitoline is the highest and is where the stronghold is built.
	\end{itemize}
\end{itemize}

\section{Early Italy}
\subsection{Beginnings}
\begin{itemize}
\item Urnfield culture in 1800BC (put their dead in urns).
\item Villanovans in 1000-750BC.
\item Mostly wattle and daub houses (twigs covered with mud).
\item The center-most hill of Rome is settled (Palatine).
\item Later on the Fossa People (buried their dead in trenches).
\item Magna Graecia (southern Italy settled by Greeks).
\item No need to fight, plenty of resources to go around.
\end{itemize}

\subsection{The Etruscans}
\begin{itemize}
\item 900-800BC Etruscans.
	\begin{itemize}
	\item Herodotus says from the Near East.
	\item Lived in North-West Italy.
	\item Language unknown.
	\item Famed as town planners.
		\begin{itemize}
		\item Built towns in rectangle shape with roads crossing North/South and East/West.
		\end{itemize}
	\item Devoutly religious.
		\begin{itemize}
		\item Three gods, Jupiter, Juno and Minerva.
		\end{itemize}
	\item Mudbrick houses.
	\item \textbf{Necropolis}, burial in decorated tombs arranged,
	\item Bucchero (black pottery), sold all around the Mediterranean, excellent merchants.
	\item Fine metal worker and craftsmen in terracotta.
	\end{itemize}
\item 550BC expansion into the land around them, beginning of Hellenistic (after death of Alexander the Great, formation of Roman Empire) phase.
\item Emperor Claudius (41-54AD) was the last Etruscan speaker.
	\begin{itemize}
	\item Historian.
	\item Wrote 20 books on Etruscan History.
	\item All Etruscan literature lost.
	\end{itemize}
\end{itemize}

\section{The Monarchy (753-510BC)}
\subsection{The Founding of Rome}

\begin{itemize}
\item Founding of Rome was likely very boring, probably just farmers on a hilltop who eventually began a town and then a city.
\item However, to assert the divine creation of Rome, myth is invented.
\item So the story goes:
	\begin{itemize}
	\item A Vestal Virgin is impregnated by Mars and gives birth to \textbf{Romulus and Remus}.
	\item Her brother is upset and wants to kill the children.
	\item Like any good mother, she puts the children in a wicker basket and sends them down the Tiber.
	\item They are taken in by a \emph{She-wolf} who raises them as her own.
	\item Romulus eventually in his adulthood kills Remus.
	\item Romulus becomes the first Etruscan king of Rome (7 total - divine number).
	\item Rome officially founded April 21, 8:05 AM, \textbf{753 BCE}.
	\item \textbf{Pomoerium} is the ``sacred'' boundary.
	\item At first there is only men in Rome, so the Romans arrange a party for their neighbors and once they are drunk, steal and rape their women.
		\begin{itemize} \item Raped women love Rome so much, they stay. \end{itemize}
	\end{itemize}
\end{itemize}

\subsection{Kings}
\begin{itemize}
\item Each king (\textbf{Rex}) has two \textbf{Lictors} which are attendants of the King.  Later they become magistrates (judges).
\item The Lictors carried \textbf{Fasces}, bundles of rods and axes.  Often seen during victory parades, which happened pretty often.
\end{itemize}

\subsection{Classes}
\begin{itemize}
\item The people of early Rome had a very specific class system, broken down into two categories:
	\begin{itemize}
	\item Patricians
		\begin{itemize}
		\item Social upper class.
		\item Make up 10\% - 20\%
		\end{itemize}
	\item Plebeians
		\begin{itemize}
		\item Social lower class.
		\item Make up 80\% - 90\%
		\end{itemize}
	\end{itemize}
\item \textbf{Gentes} (the family clan) became very important.
	\begin{itemize}
	\item Your name was a compound name.
	\item Given Name + Clan Name + Family Name
	\end{itemize}
\item \textbf{Curiate Assembly} was formed, 10 for every tribe (10 x 3).  In charge of voting ``democratically'' (only Patricians could vote).
\item Each tribe provided \textbf{Centuries} for Rome.
\end{itemize}

\section{The Early Republic (509-264BC)}
\begin{itemize}
\item 510/509 BC expulsion of Etruscan kings.
\item Romans date this as 244 a.u.c. (\emph{ab urbe condita = the the foundation of the city}).
	\begin{itemize}
	\item 244 + 509 = 753BC
	\end{itemize}
\item \textbf{Res publica}, republic, for the people.
\item 2 consuls (cheif magistrates)
	\begin{itemize}
	\item Replace the Rex
	\end{itemize}
\item Dictator - 6 months maximum.  Only when issues arose and decisions had to be made, often used less than 6 months (too much power, hand it away quickly).
\item Patricians run the Senate.
\item 471BC Plebian Council
	\begin{itemize}
	\item Tribunes, representatives of the plebs.
	\end{itemize}
\item Twelve Tables
	\begin{itemize}
	\item Laws posted clearly on two bronze tablets.
	\item Speaks to the literacy level of the republic.
	\end{itemize}
\end{itemize}
\subsection{The Legend of Horatius Cocles}
\begin{itemize}
\item 509BC the Etruscan king Lars Porsemma of Clusium attacked Rome.
\item Horatius defended the pons Sublicius bridge.
	\begin{itemize}
	\item Cocles - one eyed.  Oddly, a good thing in Roman culture (blessed by gods)
	\end{itemize}
\item Defends the bridge singlehandedly as his friends cut down the bridge behind him.
\item At the last second jumps over and survives.
\end{itemize}
\subsection{A New City Defends Itself}
\begin{itemize}
\item 493BC Latin League
	\begin{itemize}
	\item Allies with Latin tribes around them to protect against the Etruscans.
	\item Rome gets between the fights of the Latin tribes, help in fights and defeat other tribes and makes them allies and eventually Roman.
	\item Slowly Rome grows and has no enemies. (Divide and conquer)
	\end{itemize}
\item 480-396BC Veii, closest Etruscan city to Rome.
	\begin{itemize}
	\item After defeating these guys though, they kill everyone.
	\end{itemize}
\item Gaul: A territory north of the Apennine mountains in modern day France.
	\begin{itemize}
	\item Taller on average, blonde or red-haired.
	\item Huge populations are armies.
	\item Heroic warfare still important.
	\item Fanatics would fight naked.
	\end{itemize}
\end{itemize}

\subsection{The Sack of Rome}
\begin{itemize}
\item 390/387 sace of Rome.  Brennus, Cheiftain of the Senones.
\item \textbf{Vae victis}.  Woe to the Vanquished.  Sucks to lose.
	\begin{itemize}
	\item 1000 pounds of gold ransom 
	\end{itemize}
\item Capitol (Capitoline Hill) is not taken.
\item Romans take advantage of the Barbarians sack of Etruscan villages on the way to Rome, following and finishing the job.
\end{itemize}

\subsection{Expanding North}
\begin{itemize}
\item Rome expands North following the retreat of the Gauls.
\item “Servian” Wall (really dates to 380's, not Servius Tullius) built
\item ager publicus ( land belonging to the state)
\item colonies (veteran settlement in captured territories)
	\begin{itemize}
	\item Keep an eye on things (well trained military)
	\end{itemize}
\end{itemize}

\subsection{Samnites}
\begin{itemize}
\item Italic herdsmen, lived in mountains.
\item Huge families, bred like rabbits, threaten to swamp Italy.
\item Mobile experts at mountain and rough ground fighting.
\item Samnite Wars (343-290BC)
\item 321BC Caudine Forks: Colossal loss for Rome.
	\begin{itemize}
	\item Pass beneath the Yolk - insult and embarrass the whole army and Romans.
	\item Refuse peace treaty, give the two generals instead.  Bad luck for the Samnites to accept the gift.
	\item Angers Rome more and more and they decided they need to work harder on defeating these guys.
	\end{itemize} 
\item Via Appia: Fortified road from Rome to Campania.
	\begin{itemize}
	\item Speed, communication and supplies.
	\end{itemize}
\item Eventually absorb the Samnites into the Roman empire.
\end{itemize}

\subsection{The Pyrrhic Wars}
\begin{itemize}
\item Tarentum
	\begin{itemize}
	\item Major Greek city state in southern Italy
	\item Threatened by Italic Sabelline trines to their north. 
	\end{itemize}
\item Tarentum calls on King Pyrrhic (Greek - Alexander the Great's Cousin) for aid.
\item Sabines call on Rome for aid.
\item 280-275BC Wars
	\begin{itemize}
	\item Pyrrhus brings 25000 pikemen and war elephants.
	\item First time Romans see Elephants.
	\item Wins three battles and leaves. Was not expecting to see the Romans, could not afford to lose key soldiers to them.
	\end{itemize}
\item 264BC Rome is the \textbf{Domina} of central and southern Italy.  Can call up to 700000 troops if needed.
\end{itemize}

\section{Government}
\begin{itemize}
\item S.P.Q.R (\emph{The Senate and Roman People})
\item Senate (Aristocratic, old Patrician families)
	\begin{itemize}
	\item Major legislation and advise consults.
	\item Foreign policy
	\item Senatus consultum (\emph{decree of the Senate})
	\item Should be of strong moral character. 
	\end{itemize}
\end{itemize}

\subsection{Three popular Asemblies}
\begin{itemize}
\item Curiate Assembly
\item Centuriate Assembly
	\begin{itemize}
	\item Contains Plebs and Patricians
	\item Majority voting power is in the patricians favor. 
	\end{itemize}
\item Tribal Assembly
	\begin{itemize}
	\item 35 tribes, 4 in Rome, 31 in country.
	\item Elect lower magistrates (Quaestor and Aediles) and the 10 Tribunes of the Plebs.
	\item Plebian Council (471BC)
	\item 287 BCE the Lex Hortensia made the \textbf{plebiscite} (decision of the plebs) law.
	\item Magistrates:
		\begin{itemize}
		\item Cursus Honorum (starts at age 30, senatorial career pattern)
		\item Quaestor (4, eventually 20), financial, inluding provincial treasurer.
		\item Aediles (4) - in charge of streets, markets, festivals and public works. 
		\end{itemize}
	\item Praetor (8)
		\begin{itemize}
		\item In charge of public law courts or governors.
		\item Held the power of a lesser Consul.
		\item Should be at least 39 years old and have serves as a Quaestor
		\end{itemize}
	\item Consul(2)
		\begin{itemize}
		\item Chief magistrate, with legal and military power.
		\item replaced the Etruscan kings
		\item commanded the armies of Rome
		\item must be at least 42 years old
		\item each could veto (meaning “I forbid”) the other
		\item 367 BCE law requires one of the Consuls to be a Plebeian
		\end{itemize}
	\item Censor (2, every 5 years for an 18 month term)
	\item Tribune (10)
		\begin{itemize}
		\item represent plebs
		\item sacrosanct man of god.  Cannot be persecuted by anyone.
		\item veto
		\end{itemize}
	\item Dictator (1)
		\begin{itemize}
		\item dictator re gerundae causa (‘dictator to do what needs to be done’)
		\item only in emergency
		\item only for 6 months maximum
		\item limitless power to safeguard the state 
		\end{itemize}
	\item Lictors (2 - same as before)
	\item Triumph
		\begin{itemize}
		\item legal wars that were won and resulted in at least 5,000 enemy dead required a Triumph
		\item Victorious General, in his best clothes and armour, with his face painted purple, was paraded through Rome
		\item accompanied by soldiers, captives and spoils of war
		\item procession ended at the Temple of Jupiter Optimus Maximus on the Capitoline Hill
		\item general made sacrifices to Jupiter for the victory
		\end{itemize}
	\item Ovation
		\begin{itemize}
		\item lesser victories (fewer enemy dead or against lesser enemies, such as slaves) received an Ovation
		\item an honour, but less impressive procession and celebration
		\end{itemize}
	\end{itemize}	
\end{itemize}

\section{Republican Ideals}
\begin{itemize}
\item mos maiorum (ancestral customs, respect and emilate ancestral traditions)
\item gravitas (seriousness - self control) 
\item pietas (respect for authority to the gods, state and family)
\item religio (being “bound” to the gods by acting the way you should)
\item virtus (manliness, courage)
\item fides (loyalty, faithfulness, honesty, integrity)
\item simplicitas (plain lifestyle)
\item clementia (calculated mercy)
\item frugalitas (frugality)
\end{itemize}

\section{Family Life}
\begin{itemize}
\item familia (family)
\item Differences between Roman and “modern” families
	\begin{itemize}
	\item extended family, including dependent children and slaves
	\item many children lost at least one parent by age 15
	\end{itemize}
\end{itemize}

\subsection{Paterfamilias}
\begin{itemize}
\item paterfamilias (male head of the family)
\item patria potestas (authority of the paterfamilias)
	\begin{itemize}
	\item can expose unwanted children, or give away/abandon to others
	\item adultery laws of 18 BCE allows father to kill daughter and seducer if “caught in the act” in his own home
	\end{itemize}
\item genius (protective spirit)
\end{itemize}

\subsection{Matrona}
\begin{itemize}
\item matrona (wife of the paterfamilias)
	\begin{itemize}
	\item virtuous and strong
	\item devoted to the education and advancement of her family
	\item self sacrificing
	\item run household and slaves
	\item make and craft with wool
	\item many wives and stepmothers due to high mortality (and divorce among Patrician class)
	\end{itemize}
\end{itemize}

\subsection{Women}
\begin{itemize}
\item bias of our evidence (written by men for men)
\item role of women:
	\begin{itemize}	
	\item biological (childbirth, sex)
	\item economic (dowry, household management, labour, wool-working)
	\item supervise slaves, children
	\end{itemize}
\item high moral standard expected (otherwise could be killed)
\item little involvement in public life (service to emperor or deity)
\item demonstration against Oppian Law on luxury (195 BC)
\item Notable women:
	\begin{itemize}
	\item Cornelia (mother of the Gracchi)
	\item Laelia, Hortensia (orators, great public speakers)
	\item Iaia of Cyzicus (painter)
	\item Theophila (philosopher-poet, compared with Sappho)
	\item Hypatia (philosopher-mathematician, in Alexandria until bishop thought was pagan and she was killed)
	\item Demo (commentator on Homer)
	\item criticism of women: Juvenal's 6th satire
	\item praise of women: Quintilian; eulogy of Turia
	\end{itemize}
\item legal dependency: male control (father, husband, guardian)
	\begin{itemize}
	\item incl. exposure, arranged marriages
	\end{itemize}
\item double standard re. adultery, citizenship
\item home bodies, or party animals? e.g. Livy vs. Ovid; Sabine women;
\item Lucretia; Good Goddess; Papirius (all role models)
\item Patrician women do not work!
\item Most Plebeian women (low class) do work
\item women in work force (jobs attested in inscriptions, reliefs)
	\begin{itemize}
	\item dress maker
	\item hair dresser
	\item fish monger
	\item farmer
	\item taberna (bar) maid
	\item cottage industrie
	\item ”comfort girl” for shepherds
	\end{itemize}
\end{itemize}

\subsection{Children}
\begin{itemize}
\item (sources: Pliny the Elder, Lucretius, Soranus, Quintilian, Martial, Cicero, Plutarch)
\item Augustus' legislation to encourage children
	\begin{itemize}
	\item 9AD law giving priority to Consul with the most children
	\item women remarry within 1 year if widowed, or 6 months if divorced
	\item financial rewards for marriage and children
	\item bachelor’s cannot inherit until they marry
	\item short engagements
	\end{itemize}
\item use of contraceptives, actually did.
\item strange ideas on mechanics of birth - didn't understand cycles - women are simply greenhouses for birth (plant the seed).
\item Miscarriages (common and due to hysteria/pressure to have children)
\item Death from childbirth common
\item abortion (e.g. Domitian's niece) (not against not having children, but against the idea of getting an abortion to prevent stretch marks)
\item exposure by paterfamilias
\item Adoption (common and often necessary to provide an heir)
\item size of families (e.g. Germanicus, Marcus Aurelius)
\item illegitimate children
\item “Posthumous” (who's the father, add Posthumous at the end of a name)
\item treatment of children
\item alimenta (relief scheme for farmers and needy children) started by the Emperor Nerva - baby bonus.
\end{itemize}

\section{Republican Literature}
\begin{itemize}
\item no Latin literature until 3rd c. BC (too busy trying to live)
\item earliest forms are just copies of Greek originals translated into Latin
\item "Captive Greece captured her rude conqueror" (Horace)
\item Romans enjoyed many and variety forms of literature
\item “Golden Age” of Roman literature begins in the 1st Century BCE
\end{itemize}

\subsection{Lucius Livius Andronicus (284-204BC)}
\begin{itemize}
\item Greek from Tarentum
\item Greco-Roman dramatist and epic poet
\item Translated many Greek works into Latin
\item \textbf{“The “Father of Latin Literature”}
\item Most famous for his plays, and translation of Homer’s Odyssey into Latin
\end{itemize}

\subsection{Quintus Ennius (239-169BC)}
\begin{itemize}
\item \textbf{“The Father of Latin Poetry”}
\item Only fragments of his work survive, but his influence is very significant 
\item The Epicharmus discusses the nature of the gods, the universe, and heavenly enlightenment. 
\item The Annals is an epic poem of the history of Rome in verse, written in 18 books, covering the period from the fall of Troy in 1180 CE, to the Censorship of Cato the Elder in 184 BC
\item Writes history as poetry.
\end{itemize}

\subsection{Polybius (203-120 BC)}
\begin{itemize}
\item \textbf{Greek Historian}, soldier, general, statesman, and political hostage of Rome
\item Wrote a prose History of Rome, The Histories, covering the period 220-146 BCE
\item A bit biased
\item Believed that Historians must write from experience
\item First person accounts.
\end{itemize}

\subsection{Titus Maccius Plautus (254-184BC)}
\begin{itemize}
\item \textbf{Roman comedic playwright}
\item 21 of 130 plays survive (high rate!)
\item Rude, crude, low class and populist comedian
\item One of the first writers of musical theatre
\end{itemize}

\subsection{Publius Terentius (Terence) Aper (195-159BC)}
\begin{itemize}
\item \textbf{Comedic playwright}
\item Subtle humor.
\item Was brought to Rome as a slave by Terentius Lucanus, a  senator, was educated by him and then freed when his talent was recognized
\item All 6 of his plays survive
\item More refined than Plautus, but less funny (more intellectual)
\item Plagiarized others?
\item \textbf{“Fortune favors the brave”}
\item \textbf{“Where there is life there is hope”}
\item \textbf{“Each man to his own opinion”}
\end{itemize}

\subsection{Marcus Porcius Cato (234-149BC)}
\begin{itemize}
\item a Roman statesman, surnamed the Censor (Censorius), the Wise (Sapiens), the Ancient (Priscus), or the Elder (Maior)
\item \textbf{“Father of Latin Prose”}
\item wrote artistic prose
\item wrote on History, politics, agriculture and technical subjects
\item disliked aristocrats
\end{itemize}

\subsection{Gaius Lucilius (160’s-103/2BE)}
\begin{itemize}
\item Roman Equestrian
\item One of the earliest Roman \textbf{satirists} (the only literary form invented by the Romans)
\item Harsh critic of people, politicians and “foreigners”
\item Few fragments survive of his work
\end{itemize}

\subsection{Titus Lucretius Carus(c. 99-55BC)}
\begin{itemize}
\item Roman poet and Epicurean philosopher
\item Only known work is the epic poem \textbf{De Rerum Natura}, (“On the Nature of Things”)
	\begin{itemize}
	\item outlines his views on Epicurean philosophy in order to free people of the fear of the supernatural and death
	\end{itemize}
\end{itemize}

\subsection{Marcus Tullius Cicero (106–43BC)}
\begin{itemize}
\item Roman Equestrian, statesman,  Consul, philosopher, lawyer, orator and constitutionalist
\item Brilliant orator and prose writer
\item \textbf{De Re Publica} (“On The Republic”) and \textbf{De Legibus} (“On The Laws”)
\item Proponent of  “rights”, based on ancient law and custom. 
\item 6 books six on rhetoric, parts of eight on philosophy, and 58 speeches survive.
\end{itemize}

\subsection{Gaius Julius Caesar (100-44BC) (all important)}
\begin{itemize}
\item Roman General and statesman 
	\begin{itemize}
	\item considered one of the best orators and writers of Latin prose
	\item historical commentaries on Gallic Wars and Civil Wars
	\end{itemize}
\end{itemize}

\subsection{Gaius Sallustius (Sallust) Crispus (86-35BC)}
\begin{itemize}
\item \textbf{historian}, politician, and Novus Homo 
	\begin{itemize}
	\item supporter of Julius Caesar and opponent of Cicero
	\end{itemize}
\item \textbf{The Jugurthine War, Catiline Conspiracy and  Histories}  (fragments)
\item tried to show the connection and meaning of events, not just record them
\end{itemize}

\subsection{Gaius Valerius Catullus (84–54BC)}
\begin{itemize}
\item A  rich Equestrian from Cisalpine Gaul
\item Alexandrian school of lyric poetry
\item very explicit style 
	\begin{itemize}
	\item very popular with some, and despised by others , for being rude and amoral
	\end{itemize}
\item Influenced Ovid, Horace and Virgil
\item Lesbia poetry
\end{itemize}

\section {The Punic Wars (264-146 BC)}
\subsection{The Founding of Carthage}
\begin{itemize}
\item The Legendary Queen Dido of Tyre founded Carthage in 814BCE
\item Named Kart-hadasht (Carthage) meaning “New City” 
\end{itemize}

\subsection{Workup to the Wars}
\begin{itemize}
\item 3 Punic Wars ("Punic" = Carthaginian or Phoenician)
\item Carthage (near Tunis) on a promontory 
\item 37 km of walls
\item Population of  700,000 (400,000 citizens plus non-citizens and slaves)
\item Military harbor holds 220 warships 
	\begin{itemize}
	\item at her peak, she had 300-350 warships
	\end{itemize}
\item Merchant harbor much bigger
\item Maritime trade empire, based on Tyrian purple dye
	\begin{itemize} 
	\item purple dye worth 15 to 20 times its weight in gold
	\item trade by sea across Mediterranean, as well as from Britain to West Africa
	\item by land trade caravans to central Africa and Persia
	\end{itemize}
\item Control up to 300 trade colonies around western Mediterranean
	\begin{itemize}
	\item trade colonies seldom have more than 1,000 inhabitants
	\end{itemize}
\item Items Commonly Traded By Carthaginian Merchants
	\begin{itemize}		
	\item finely embroidered silks
	\item dyed textiles of cotton, linen, and wool
	\item Animals (especially cattle and horses)
	\item artistic and functional pottery and ceramics
	\item incense and perfumes
	\item Items crafted from ivory, glassware, wood, alabaster, tin, bronze, brass, lead, gold, silver, and precious stones
	\item Furniture, mirrors, pillows, jewelry, arms, armour, farming implements, and household items
	\item A wide variety of foods, salted Atlantic fish and fish sauce (called “garum” by the Romans)  
	\item Goods from their trading partners across the Mediterranean 
	\end{itemize}
\item Very religious
\item Worship old Semitic gods, ie Tanit, Melqart and Ba’al
\item Sacrifice children in times of distress
\item Carthage’s Military
	\begin{itemize}
	\item Huge, elite navy for trade and protection

	\item Small citizen population

	\item Freely intermarried with local population

	\item Large mercenary army
		\begin{itemize}
		\item Carthaginian Officer
		\item Elephants, Gauls, Celts, Greeks, Africans, Italians, Sicilians, Spaniards, Numidians, etc make up army
		\item fought as Greek Hoplites, supported by lighter skirmishers and lots of cavalry
		\end{itemize}
	\end{itemize}
\end{itemize}

\subsection{1st Punic War (264-241 BCE)}
\begin{itemize}
\item Sicily divided between the Kingdom of Syracuse, Carthaginian trading cities, and independent Greek cities
\item Mamertine (“sons of Mars) take city of Messana
\item Syracuse attacks
\item Mamertines call for aid to Carthage and Rome
\item Carthage arrives and takes city
\item Mamertines call on Rome (as fellow Italians!) to get rid of Carthaginians
\item Syracuse and Carthage declare war on Rome
\item Syracuse then makes a separate peace treaty
\item Sicily: intervention of Rome and Carthage(264 – 241 BC)
\item Carthage (a sea power) is now at war with Rome (a land power)
\item Rome suffers several losses at sea
	\begin{itemize}
	\item captures a Carthaginian and copies it
	\item slowly learn to become successful sailors and mass produce a navy
	\end{itemize}
\item The Corvus (“Raven”)
\item 255 BC Regulus' expedition
	\begin{itemize}
	\item defeated by the Spartan mercenary general Xanthippus at the Battle of Bagradas River
	\end{itemize}
\item 241 Peace Treaty
	\begin{itemize}	
	\item War indemnity of 3,200 Talents of Silver over 10 years (1 Talent = 60 pounds)
	\item Sardinia, Sicily and Corsica: annexed by Rome (Rome becomes an Empire!)
	\end{itemize}
\end{itemize}

\subsection{2nd Punic War (219-201 BCE)}
\begin{itemize}
\item Hamilcar Barca (“thunderbolt”)
	\begin{itemize}
	\item 248-241 Supreme Carthaginian Commander in Sicily
	\item 236 sent to build Spanish empire for Carthage
	\item acquired vast amounts of silver, soldiers and timber
	\end{itemize}
\item Hannibal Barca of Carthage
	\begin{itemize}
	\item 221 BC Hannibal Barca, son of Hamilcar becomes general of Spain
	\item brilliant tactician
	\item loved by his soldiers
	\item oath to never be a friend to Rome
	\end{itemize}
\item The Causes of the Second Punic War (226-219 BCE)
	\begin{itemize}
	\item 226 BCE Ebro Treaty broken by Rome
	\item 219 BCE siege of Saguntum
	\item R. “I give you war or peace” H. “You choose” R. “Then it is war”
	\end{itemize}
\item Hannibal hopes to outflank the Romans by invading overland
\item 218 BC crosses the Alps (35,000 infantry, 6,000 cavalry and 60 elephants vs 700,000 Romans)
\item Half survive the journey (cold and no road), actually got there.
\item 217 BC only 1 elephant left “The Heap”
\item Has never been done before, Romans are terrified.
\item Early Victories
	\begin{itemize}
	\item \textbf{Hannibal wins two quick victories at the River Trebia (218 BCE) and Lake Trasimene (217 BCE)}
		\begin{itemize}
		\item \textbf{open northern Italy to invasion}
		\item \textbf{armies gone and leaders killed!}
		\item \textbf{attributed to “Punic Treachery!”}
		\item \textbf{Not playing fair "dirty tricks"}
		\end{itemize}
	\item Gauls and some northern Italian cities join Hannibal
	\item Some southern Italian Greek cities join Hannibal as well
	\end{itemize}
\item The Battle of Cannae 216 BCE
	\begin{itemize}
	\item 216 BCE Rome builds a massive army (80,000 men) and decides to wipe out Hannibal once and for all!
	\item Maneuvers to fight Hannibal on an open plain near the village of Cannae
		\begin{itemize}
		\item no place for ambushes or other “Punic Treachery”
		\item yet is perfect cavalry country
		\item able to use a tactic called the “double envelopment”
		\item becomes Hannibal’s greatest victory and Rome’s greatest defeat (lose 30-60,000 men)
		\item A plain the size of UWaterloo.
		\end{itemize}
	\end{itemize}
\item Results of Cannae
	\begin{itemize}
	\item 216 BC Philip V of Macedon declares war on Rome and allies to Hannibal
	\item 210 BC Rome appoints Scipio (later nicknamed Africanus for his victory over Carthage) as Commander (their own Hannibal)
	\item Carries out delaying tactics in Italy while he invades Spain
	\item Hannibal simply does not have enough troups to take over Rome.  Walk around rome for a while taking over small cities, but slowly losing forces.
	\end{itemize}
\item The Tide Turns Against Carthage
	\begin{itemize}
	\item Hannibal trapped in southern Italy
	\item 207 BCE Hasdrubal (Hannibal's brother) killed in Italy
	\item 206 BCE Scipio defeats Carthaginian armies in Spain
	\item 204 BCE Scipio invades Africa
	\item 203 BCE Hannibal recalled to Africa
	\item Zama (202 BCE): major Carthaginian defeat
	\item 201 BCE Peace Treaty
		\begin{itemize}
		\item war indemnity of 10,000 Talents over 50 years
		\item annex Spain
		\end{itemize}
	\end{itemize}
\end{itemize}
\subsection{Between the 2nd and 3rd Punic Wars}
\begin{itemize}
\item Macedonia (north of Greece)
	\begin{itemize}
	\item defeated by Rome and her allies (4 wars from 216-148 BCE)
	\end{itemize}
\item Hannibal rebuilds Carthage, but is forced to flee to Asia Minor
	\begin{itemize}
	\item 184 BCE commits suicide in Bithynia (Turkey)
	\end{itemize}
\item 183 BCE Scipio Africanus dies
\end{itemize}

\subsection{The 3rd Punic War (149-146 BCE)}
\begin{itemize}
\item Numidians (hostile neighbours of Carthage)
	\begin{itemize}
	\item become allies of Rome	
	\item King Massinissa (240-148 BC) conquers most of Carthage’s territory
	\item provokes a war and calls on Rome for protection
	\end{itemize}
\item Third Punic War (149-146 BC)
	\begin{itemize}
	\item Rome jealous of Carthage’s growing prosperity
	\item Cato: "Carthage must be destroyed!"
	\item Kill everyone after getting in, salt the ground so nothing will grow.  They take Carthage out.
	\end{itemize}
\end{itemize}

\section{Roman Social Structure}
\begin{itemize}
\item Roman society is very structured and stratified
	\begin{itemize}
	\item different rights and protections based on status
	\item all know and accept this
	\item still a society with social and economic mobility
	\end{itemize}
\item Patricians (aristocracy, upper social order) 10\% of population
	\begin{itemize}
	\item honestiores (the "more honourable" upper class)
	\end{itemize}
\item Plebeians (plebs) (commoners, lower social order) 90\% of population
	\begin{itemize}
	\item humiliores (the lower class)
	\end{itemize}
\item gentes (gens) (clans)
\item Roman Upper Classes
	\begin{itemize}
	\item Senatorials (governing body of Republican Rome)
		\begin{itemize}
		\item nobiles (nobility, patrician, senatorial class)
		\item senator: 1 million sesterces
		\end{itemize}
	\item Equites (equestrians) 
		\begin{itemize}
		\item rich plebeians (cavalry, business class) 
		\item 400,000 sesterces
		\end{itemize}
	\end{itemize}
\item Other Class Status Symbols
	\begin{itemize}
	\item novus homo ("new man", without consular ancestors)
	\item cursus honorum (career ladder, sequence of public offices)
	\item publicani (state contractors, from equestrian class)
	\item procurators, prefects (senior equestrian appointments)
	\end{itemize}
\end{itemize}

\subsection{Slaves and Freedpersons}
\begin{itemize}
\item differences between ancient and modern concepts of slavery
\item prisoners of war: cheap slaves make latifundia possible
	\begin{itemize}
	\item Caesar took over 1 million slaves (58-51 BC)
	\end{itemize}
\item servus (slave) = manpower, status symbol, wealth (chattels)
\item acquiring: purchase (slave market; dealers dishonest)
\item vernae (slaves born on the master's estate)
\item loan/rental, e.g. wet-nurses
\item Purchasing Slaves
	\begin{itemize}
	\item (prices at approximately year 1 CE)
	\item General Labourer Slave = 500-1500 Denarii
	\item Pretty Female Slave = 2000-6000 Denarii
	\item Music Girl = 4000 Denarii
	\item Skilled Vineyard Worker = 2000 Denarii
	\item By comparison, the daily wages for Roman citizens were:
		\begin{itemize}
		\item Farm labourer, with meals = 25 Denarii
		\item Baker, with meals = 50 Denarii
		\item Barber, per man = 2 Denarii
		\item Painter, walls, with meals = 75 Denarii
		\item Unskilled day labourer = 1 Denarius
		\end{itemize}
	\end{itemize}
\item Slave Jobs
	\begin{itemize}
	\item agricultural (e.g. on latifundia)
	\item industrial (manufacturing)
	\item unskilled (mines, quarries, construction, docks, galleys)
	\item domestic (household slaves: easier life, chance of freedom)
	\item clerical/administrative (civil service)
	\item gladiator (punishment for runaway/criminal slave) 
	\end{itemize}
\item slave can't be soldier, except in dire emergency, e.g. Cannae
\item job specialization, Roman's found it better to put Slaves where they are most profitable (accountants, technologists)
\item Greek slaves educated (physicians, tutors)
\item slave foreman runs rural estate for absentee master
\item master's powers of punishment unlimited
	\begin{itemize}
	\item ie Vedius Pollio (who was descended from slaves), threw his own slaves into a pond to be eaten by huge eels as punishment
	\end{itemize}
\item ergastula (prison barracks where slave were locked up)
	\begin{itemize}
	\item fugitive slaves could be branded on the face  with “FUG” (for “fugitivus”) or be forced to wear a slave collar
	\item sometimes collar has a tag attached which reads “TMQF” (tene me quia fugio) “hold me, because I run away” 
	\end{itemize}
\item distrust of slaves: evidence under torture; masters murdered
\item Slave Revolts
	\begin{itemize}
	\item 135-131 BC Sicily (60,000 slaves)
	\item 73-71 BC Spartacus (Southern Italy)
		\begin{itemize}
		\item last 6,000 crucified along Appian Way
		\end{itemize}
	\end{itemize}
\item abuses curbed by Claudius, Vespasian, Hadrian, Antoninus Pius
\item Pliny's enlightened treatment of his slaves
\item “From the master to his slave girl” (found engraved on an expensive gold bracelet on a women’s body just outside of Pompeii)
	\begin{itemize}
	\item no proof this feeling was mutual!
	\end{itemize}
\item training and wealth of some slaves
\item privileges: peculium (slave's savings)
\item contubernium (cohabitation with a fellow slave)
\item manumission (setting a slave free)
	\begin{itemize}
	\item for long or exceptional service
	\item for saving master's life
	\item in exchange for peculium
	\end{itemize}
\item methods: 
	\begin{itemize}
	\item by the rod (before praetor or governor)
	\item by testament
	\item informal (don't get Roman citizenship but become "Junian" Latins), could later say that you "ran away" and have you back.
	\end{itemize}
\item libertus (freedperson, ex-slave)
\item freed slave becomes client of former master
	\begin{itemize}
	\item owes him several days' service each year
	\end{itemize}
\item Libertus (freedman): can't hold public office (but sons can)
\item can't marry into senatorial class
\item collegia (burial/social clubs) joined by slaves, freedpersons
\item Augustus limits number of slaves that can be manumitted
\end{itemize}

\subsection{Roman Marriage}
\begin{itemize}
\item For the patricians, more of a legal and political arrangement vs romance
\item arranged marriages (matchmakers, e.g. friends, orators)
\item criteria for choosing a spouse: wealth, influence, fertility, status
\item forbidden matches: 
	\begin{itemize}
	\item senator/lower class
	\item Roman/foreigner
	\item free person/slave
	\item soldiers
	\end{itemize}
\item contubernium (cohabitation; marital union not recognized by law)
\item Republic: father's consent only; later: father's + children's consent
\item minimum age: 14 (boys), 12 (girls)
\item betrothal: minimum age gradually raised to 10
\item gifts, agreement, dowry, iron rings (on third finger of bride’s left hand), party (family lists advantages of marriage)
\item calendar restrictions
	\begin{itemize}
	\item avoid ill-omened days
	\item 2nd half of June was considered lucky
	\end{itemize}
\item The Roman Bride and Groom
	\begin{itemize}
	\item The day before the wedding bride’s dedicated their toys to their household gods
	\item Exchanged their “child’s” clothing for a wedding dress
	\item Long white dress, belted at the waist with a “Hercules knot”
	\item Flame red veil and shoes
	\item Ornate hair with ribbons, and a floral headdress
	\item The groom just wore his best toga
	\end{itemize}
\item Types of Marriages
	\begin{itemize}
	\item A) Civil wedding: 
		\begin{itemize}
		\item groom wears his best toga
		\item decoration of house (wreaths, flowers and evergreen branches)
		\item contract, sacrifice, reception, procession to new home
		\item threshold ceremony (Janus)
		\end{itemize}
	\item B) Religious wedding:
		\begin{itemize}
		\item religious wedding cakes part of ceremony
		\item hard to annul, unpopular
		\end{itemize}
	\end{itemize}
\item dowry: recoverable on divorce
\item Marriage and Divorce
	\begin{itemize}
	\item manus (husband's legal control over wife)
	\item changing attitudes ("one-man woman" vs. frequent divorce)
	\item grounds for divorce: originally, only adultery; later, any reason
	\item remarriage (not necessarily for love)
	\item divorce especially common among upper class
	\item A woman could divorce her husband by staying away from his home for 3 consecutive nights
	\item child custody: father (legally at least)
	\item punishments for adultery: death, exile, partial loss of dowry (usually women punished)
	\item 9AD enforced under Augustus’ morality laws
	\end{itemize}
\end{itemize}

\section{Agriculture}
\begin{itemize}
\item farming manuals: 
	\begin{itemize}
	\item Cato (2nd c. BC)
	\item Varro (1st c. BC)
	\item Columella (1st c. AD)
	\item Palladius (4th c. AD)
	\item Vergil, Georgics (1st c. BC)
	\end{itemize}
\item Patricians believed that owning land, livestock and farming are the most honourable way to make money
	\begin{itemize}
	\item central to the Roman economy
	\item rich often invested in crops and livestock, or purchased land and rented it out to farmers
	\end{itemize}
\item "Mediterranean triad" (wheat/grain, olives, grapes)
\item Farmers tried to diversify their crops to maximize their profits and minimize the impact of poor harvests
\item Taxes based on the harvest, and paid in cash or in kind
\end{itemize}

\subsection{Agriculture Techniques}
\begin{itemize}
\item terracing crops
\item rotation of crops
\item Fertilizer
\item intercultivation (planting cereals between rows of trees)
\item draining and irrigation
\item farm animals transhumance
\item animal husbandry also important for labour (oxen to pull wagons), food (meat), and clothing (hide, leather and wool)
\item ard (early plough -- scratched surface only)
\item sickle, scythe, flail
\item amphora (clay shipping container)
\end{itemize}

\subsection{Farms and Farm Land}
\begin{itemize}
\item ager publicus ("common" land, owned by the state)
\item latifundia (sing. latifundium) (plantations, large estates)
\item coloni (tenant farmers)
\item villa (estate owner's residence and outbuildings)
	\begin{itemize}
	\item includes"urbane", "rustic", and utility areas
	\end{itemize}
\end{itemize}

\section{The Late Republic(146-27 BC)}
\begin{itemize}
\item Problems caused by long series of wars
	\begin{itemize}
	\item decline in the number of citizen/soldier/farmers
	\item creation of Latifundia(plantations or country estates) in opposition to the Licinian-Sextian Law of 367 BC (limited to 320 acres of land)
	\end{itemize}
\item Problems caused by long series of wars
	\begin{itemize}
	\item massive influx of slaves from great victories results in the decline in the need for citizen farmers to work the Latifundia
	\item creation of "The Mob"
	\item rise of the Equestrians and their struggle for power with the Patricians
		\begin{itemize}
		\item all use “The Mob” as a political tool, using them to boo opposing politicians and cheer themselves.
		\end{itemize}
	\item Patrician Governors of new provinces often corrupt
		\begin{itemize}
		\item often tried for corruption
		\item try and amass 3 Fortunes while in office, one fortune for legal council, second fortune to bribe everyone and third fortune for yourself.
		\end{itemize}
	\item publicani/tax-farming
		\begin{itemize}
		\item breeds further corruption
		\end{itemize}
	\item a city-state government struggling to rule an "empire"
	\end{itemize}
\item Two Political Factions form in Rome
	\begin{itemize}
	\item Populares (“of the People / Popular Assembly”)
	\item Optimates (“of the Patricians / Senate / best men”)
	\end{itemize}
\end{itemize}
\subsection{Tiberius Gracchus}
\begin{itemize}
\item Tiberius Gracchus (162-133 BCE)
	\begin{itemize}
	\item Plebeian
	\item Tribune of the People 133 BCE
	\item latifundia 
	\item ager publicus
	\item revise Licinian-Sextian Law (500 acres vs 320 + 160 for each of two sons)
	\end{itemize}
\item Tiberius commits 3 great irregularities
	\begin{enumerate}
	\item Tribal Assembly vs Senate (vetoed by Tribune loyal to Senate)
		\begin{itemize}
	 	\item Took to Senate (also vetoed)
	 	\end{itemize}
	\item had opposing Tribune removed (illegal)
	\item ran for second consecutive term to save his life and legislation (legal?)
		\begin{itemize}
		\item murdered (with 300 of his supporters) by the Senate
		\item The Senate isn't seeing the results they want and officially have had to use killers to solve their problems.  Government is falling apart.
		\end{itemize}
	\end{enumerate}
\end{itemize}

\subsection{Gaius Gracchus}
\begin{itemize}
\item Gaius Gracchus(153-121 BCE)
	\begin{itemize}
	\item Plebeian and brother of Tiberius Gracchus
	\item elected Tribune 123 BCE
	\item re-enacted brother's land reforms
	\item brought many reforms
	\end{itemize}
\item Equites, not Senators, to judge corrupt Governors
\item Stabilize grain prices for the poor
\item Proposed creating new colonies outside of Italy
\item proposes to extend Roman citizenship throughout Italy
\item Tries to be elected for the 3rd term in a row!
\item 121 BCE  1st use of the Senatus Consultum Ultimum (‘the final decree of the Senate”)  which leads to his murder, and that of 3,000 of his supporters
\item Shows weakness of the Senate and how political opportunists can use the power of the Plebeian Assembly/Tribune for their own ends
	\begin{itemize}
	\item use of “The Mob” for political terrorism, kill him and 3000 of his supporters.
	\item the entire system of checks and balances is breaking down!
	\end{itemize}
\end{itemize}

\subsection{Gaius Marius(157-86 BC)}
\begin{itemize}
\item Equestrian and Populares)
	\begin{itemize}
	\item Novus Homo (non-Consular family)
	\item elected Tribune in 119 BC
	\item elected Consul in 107 BC
	\end{itemize}
\item War with Jugurtha of Numidia (111-104 BC)
	\begin{itemize}
	\item remodels army (volunteers vs "landed" citizens, better training, equipment, pay, conditions, organization) 
	\item creates a full-time professional army
	\item additional pay through looting defeated enemies
	\item loyalty to general or the Senate?
	\item cohort vs maniple legion (10 cohorts of 300 men each = 1 Legion)
	\item uses army to support/intimidate Senate.  Millionaire, so he equips them with the best possible armor and pays the soldiers.
	\item Interesting mutual relationship flowers, the Senate needs to give Gaius their blessing, and also need him to protect them.
	\end{itemize}
\item Defeats Numidians (North Africa), Cimbri and Teutons (southern France)
\item 104-99 BC defeats slave revolt in Sicily and pirates
\item Marius a Hero of the Social War (90-88 BC)
\item Made Consul 7 times in 20 years(5 times in a row!)
\item Becomes Rome’s first great “Warlord”
\end{itemize}

\subsubsection{Sulla (138-78 BCE)}
\begin{itemize}
\item Lucius Cornelius Sulla "Felix”
	\begin{itemize}
	\item Patrician and colleague of Marius (one of Marius’ junior officers)
	\item an outstanding soldier
	\end{itemize}
\end{itemize}

\subsubsection{King Mithridates of Pontus (120-63 BCE)}
\begin{itemize}
\item King Mithridates of Pontus (three wars between 88 and 63 BC)
	\begin{itemize}
	\item a rich kingdom on the Black Sea
	\item 88 BC Mithridates invades Asia and threatens Greece
	\item kills 80,000 Italians in 1 day!
	\item Dude is out for blood
	\end{itemize}
\item Both Marius and Sulla want command to attack (and plunder) Mithridates
\end{itemize}	

\subsubsection{Marius vs Sulla}
\begin{itemize}
\item 88 BC Sulla granted command against Mithridates
	\begin{itemize}
	\item Marius intimidates Senate to give him the command
	\item Sulla marches on Rome and Marius flees to Africa
	\item Sulla marches on Mithridates
	\end{itemize}
\item 86 BC Marius marches on Rome, is made Consul for the 7th time, kills Sullans, and dies a few days later
	\begin{itemize}
	\item Rostra, like a hit-list (top targets that Marius wants dead and will pay for)
	\end{itemize}
\item 83 BC Sulla returns to Rome and defeats Marians
\item 82 BC The Great Proscription (1,600 Equestrians and Senatorials killed)
	\begin{itemize}
	\item Sulla made Consul and Dictator for life
	\end{itemize}
\item 79 BC retires
\item 78 BC dies (a few months after retirement)
\end{itemize}

\subsection{The 1st Triumvirate}
\begin{itemize}
\item Political deadlock and chaos follows the death of Sulla
	\begin{itemize}
	\item remaining Marians and Sullans continue to battle in the streets of Rome
	\end{itemize}
\item Spartacus (73-71 BC)
	\begin{itemize}
	\item 79,000 slaves finally defeated by Crassus and Pompey
	\item last 6,000 prisoners crucified along Appian Way
	\end{itemize}
\item Cicero (63 BC “novus homo”) and exiled in 58 BC
	\begin{itemize}
	\item Tries to bring back some normality, no luck.
	\item eloquent speaker
	\item opposed the “Warlords” and Julius Caesar
	\end{itemize}
\item 1st Triumvirate (60 BC): Crassus, Pompey, Caesar (renewed in 56 BC)
	\begin{itemize}
	\item Crassus (wealth)
		\begin{itemize}
		\item ”The Fireman of Rome”	
		\end{itemize}
	\item Pompey (Senate) “magnus” at age 25
		\begin{itemize}
		\item called “adulescens carnifex” (“Butcher Boy”) earlier in life
		\item married Caesar’s daughter Julia to cement alliance
		\end{itemize}
	\item Julius Caesar (People)
	\end{itemize}
\end{itemize}

\subsubsection{Marcus Licinius Crassus}
\begin{itemize}
\item 60 BCE joins First Triumvirate 
\item 55 BCE Consul and Governor of Syria
\item 55-53 BCE goes to war against Parthians in Iraq (seeks military glory)
\item 53 BCE killed at the Battle of Carrhae and army destroyed by Parthians
\item Desires power and prestige more than anything.
\end{itemize}
	
\subsection{Julius Caesar}
\begin{itemize}
\item Caesar in Gaul (59-52 BC)
	\begin{itemize}
	\item kills 3 million Gauls
	\item takes 1 million slaves
	\end{itemize}
\item Caesar in Germany and Britain (55-54 BC) 
	\begin{itemize}
	\item Never been done
	\end{itemize}
\item “Commentaries” on his wars in Gaul, Britain and Germany make him a household name
	\begin{itemize}
	\item gains wealth, fame and a loyal army by these campaigns
	\item heroic to go to these “wild frontiers” vs Pompey in the East
	\end{itemize}
\item \textbf{54BC Julia dies}
\item 49 BC Caesar want to run for Consul in absentia 
	\begin{itemize}
	\item They wouldn't let him so he fucking INVADES ROME.  This guy.
	\end{itemize}
\item Rubicon (49 BC) "Alea iacta est" ("the die is cast")
\item \textbf{The Battle of Pharsalus (48 BC)}
\item \textbf{The Alexandrian War/Cleopatra (48-47 BC)}
\item 47 BC put Cleopatra on the Egyptian throne (+ Caesarion)
	\begin{itemize}
	\item Talented woman, not just manipulative as the common impression.
	\item Egypt very strong ally of Rome.
	\item Has a son who is very likely Caesar's (though Caesar never acknowledges this)
	\end{itemize}
\item Rule of Julius and Caesar
\end{itemize}
		
\subsubsection{Rule of Julius and Caesar}
\begin{itemize}
\item Wins Civil War and returns to Rome
\item Consul 48, 46 and 45 BC
\item Pontifex Maximus
	\begin{itemize}
	\item added to the Senate, founded colonies, excused debts, Julian Calendar, loans to farmers, built temples, extended citizenship, forgave enemies(!)
	\end{itemize}
\item Dictator for 10 years in 46 BC (unprecedented)
\item Ides of March (15 Mar. 44 BCE)
	\begin{itemize}
	\item In fear of Julius Caesar becoming "rex"
	\item Marcus Junius Brutus and Gaius Cassius Longinus hope to restore the Republic under the leadership of the Senate
	\item 60 Senators involved in the conspiracy
	\end{itemize}
\item Caesar to Brutus: ”et tu Brute” (Even you? - Thought he was family)
\end{itemize}

\section{Dress and Hair Styles}
\subsection{Male}
\begin{itemize}
\item men's wear: tunic (knee-length woollen shirt, with or without sleeves, tied at waist); could also be worn to bed
\item cold weather: woolen cloak with centre hole and hood (poncho!)
\item caps worn only by ex-slaves, but citizens could wear sun-hats
\item formal dress: toga (woolen wrap, secured by knot)
\item Senators wear broad purple stripe, Equites a thin one
\item boys also wear purple stripe, until reaching manhood
\item only emperor wears purple toga; purple expensive, smelly
\item leather shoes with crossed straps (coloured for senators)
\item indoors: slippers; hobnailed army boots \textbf{(caliga)}; bath clogs
	\begin{itemize}
	\item Caliga was a studded army sandal.  Almost cleat like.
	\item "The shoe that won the Roman empire"
	\end{itemize}
\item men's rings (gold for elite;  silver for Equestrians: also signet rings)
\item men: orig. long hair and beard; shaving and haircuts - 3rd c. BC (need barber)
\item beards return in 2nd c. AD, disappear in 4th
\end{itemize}

\subsection{Women}
\begin{itemize}
\item women's wear (orig. toga?): ankle-length, long-sleeved tunic 
\item stola (long garment, belted above waist, worn over tunic)
\item outdoors: cape or mantle; head scarves, coloured shoes, leggings (if very cold) 
\item handbag
\item underwear: loincloth (optional); breast-band; girdle
\item jewelry: bulla (child's amulet)
\item earrings, necklaces, brooches, bracelets, cameos etc.
\item ROMAN FEMALE HAIR STYLES
	\begin{itemize}
	\item women: simple at first (long and straight)
	\item "Octavia" look (simple bun at back of head)
	\item Flavian high coiffure
	\item wigs, hair dye, combs, mirrors
	\item German and Gallic slaves kept to grow long blonde or red-haired wigs
	\item Small girls (with small hands) are popular hair dressers
	\end{itemize}
\end{itemize}

\section{Citizenship}
\begin{itemize}
\item cives (citizens, m. or f.) CIVIS ROMANUS SUM (I AM A ROMAN CITIZEN - very important)
\item Roman citizens, Latins and Peregrines
\item Full Roman citizen's rights:
	\begin{itemize}
	\item vote and hold public office
	\item marry other citizens and make a will
	\item commerce (property, contracts, inheritance)
	\item trial before urban praetor
	\item appeal criminal case to Rome
	\item wear toga
	\item bear 3 names (tria nomina)
		\begin{itemize}
	    \item ie Gaius Julius Caesar (given/clan/family)
	    \end{itemize}
	\end{itemize}
\item Roman citizen's responsibilities:
	\begin{itemize}
	\item military service
	\item pay special taxes (e.g. inheritance)
	\end{itemize}
\item Roman citizenship by: 
	\begin{itemize}
	\item birth (parents = citizens)
	\item manumission (freed slave of citizen)
	\item military service (25 years in auxiliaries)
	\item grant from emperor (individual Or community)\
	\end{itemize}
\item Roman Citizenship Latin Citizenship
\item ius Latii ("right of Latium") = Latin (partial) citizenship
	\begin{itemize}
	\item 1st step to full Roman citizenship
	\item no vote
	\item limited political and legal rights
	\item must serve in Roman military
	\item hard to marry into a “Roman” family
	\item could do business in Rome
	\end{itemize}
\item Emperor Caracalla gives Roman citizenship to all except slaves (AD 212)
\item non-Romans, non-Latins = Peregrines 
	\begin{itemize}
	\item all provincials have this status after 90 BC
	\end{itemize}
\item Peregrines: 
	\begin{itemize}
	\item lack all rights of Roman citizens
	\item trial by peregrine praetor
	\item can marry non-citizen
	\item can manumit (but no Roman citizenship for freed slaves)
	\end{itemize}
\end{itemize}

\section {Patrons and clients}
\begin{itemize}
\item (sources: Martial, Juvenal, Pliny the Younger)
\item nature of the patron-client system
	\begin{itemize}
	\item social inferior (ie client) attaches self to a social superior (ie patron) for legal and political protection
	\item Typically the homeless.
	\item Used to boost up the reputation of the patron.
	\end{itemize}
\item Patronus
	\begin{itemize}
	\item must provide support and protection to his clients as would a Pater Familias to his family members
	\end{itemize}
\item political use of clients (e.g. Clodius' gangs)
\item salutatio (client's morning greeting to patron)
	\begin{itemize}
	\item Expected as a social inferior to greet your patron publicly as he/she wakes.  
	\end{itemize}
\item sportula ("little basket": a handout of food or money)
\item patrons' complaints about parasitic clients
\item clients' complaints: humiliation, shamelessness, double standard
\end{itemize}

\section{Roman education}
\begin{itemize}
\item(sources: Horace, Plutarch, Lucian, Quintilian, Seneca)
	\begin{itemize}
	\item “A proper education is the source and root of all goodness” (Plutarch)
	\item “Bad habits distort one’s nature” (Quintillian)
	\item “The man who knows his letters has a superior mind” (Quintillian)
	\end{itemize}
\item Education intended to shape character and achieve moral and practical goals
\item wealthy went to school (ie to go into law or politics), poor learned a trade (ie family business)
\item Teachers in Rome
\item Anyone could be a teacher
	\begin{itemize}
	\item the best were highly sought after and well paid
	\item often Greeks
	\item the worst had to bribe their students not to leave (so they could keep their jobs)
	\item firm discipline
	\item Horace had a teacher nicknamed “The Swiper” (physical abuse)
	\item the average teacher made in a year what a good charioteer made in an afternoon
	\end{itemize}
\item The school day was dawn to mid afternoon
\item Students crowded on benches, with no desks, on the sides of streets 
\item Took notes on wax writing tablets using a stylus
\item Both girls and boys went to school, but not necessarily the same school/class
\item Parents are the first teachers of their children
	\begin{enumerate}
	\item ludus (school) for ages 7-12
		\begin{itemize}
		\item learn 3 R’s and moral education
		\item little or no math or science
		\item most ended their formal education at this level
		\item paedagogus (slave who escorted and tutored children)
		\end{itemize}
	\item grammaticus (grammar teacher) for ages 12-16
		\begin{itemize}
		\item Cicero, Vergil, Livy studied
		\item Latinand Greek works/speeches memorized, recited and commented on 
		\item Patrician girls taught privately, and usually ended their education here
		\end{itemize}
	\item rhetor (teacher of rhetoric) for ages 16+
		\begin{itemize}
		\item Learn how to write speeches.
		\end{itemize}
	\end{enumerate}
\item Greek ("second language" of Roman Empire)
\item Athens (world's first "university" - Ivy league of the Greek world)
\item papyrus (a type of paper, made from an Egyptian plant)
\item volumen (scroll: a "book" on rollers)
\item palimpsest ("recycled" papyrus, with writing erased)
\item Alexandria (Library)
\item Varro (most versatile of ancient teachers)
\item Palatine Library (Latin and Greek sections)
\item Augustus, Trajan, Hadrian
\end{itemize}

\section{Roman Law}
\begin{itemize}
\item IUS (law) = Root of Justice
\item early Rome: paterfamilias, king, consul
\item legal sources:
	\begin{itemize}
	\item Twelve Tables (450 BC)
	\item Senatus Consultum
	\item plebiscite
	\item edicts of magistrates/emperors
	\item legal textbooks: Institutes (Gaius, 2nd c.),
	\item Theodosian Code (4th c.)
	\item Digest (Justinian, 6th c.) Corpus Iuris Civilis, or Justinian’s Code 
	\end{itemize}
\end{itemize}

\subsection{Justinian’s Code}
\begin{itemize}
\item Justinian I, 6th c. Emperor of the Eastern Roman Empire 
	\begin{itemize}
	\item commissioned the writing of the Corpus Iuris Civilis, (Body of Civil Law, or Justinian’s Code ) between 529 and 534 CE
	\item All the existing laws and summarizes them under one book.  Amazing feat.
	\item summarized all existing laws
	\item create a text for law students
	\item update present laws and throw out old ones
	\item show precedents for current laws
	\item include Justinian’s edicts
	\item major influence in modern Western Law
	\end{itemize}
\item 3 major precepts underlying Justinian’s Code
	\begin{enumerate}
	\item Live Honestly
	\item Injure No One
	\item Grant Each Man His Rights
	\end{enumerate}
	\begin{itemize}
	\item major influence in modern Western Legal thought
	\end{itemize}
\end{itemize}

\subsection{Kinds of Law}
\begin{itemize}
\item kinds of law: 
	\begin{itemize}
	\item public (criminal): state/citizen
	\item civil (private): citizen vs citizen
	\item law of nations: citizen/foreigner (common to all men)
	\end{itemize}
\item praetors (judicial magistrates): urban (citizens)
\item peregrine praetors(foreigners)
\item provincial governor: circuit court; edicts
\item consilium principis (emperor's council) = legal experts who advise the emperor on interpretation of laws
\end{itemize}

\subsubsection{Roman criminal law}
\begin{itemize}
\item criminal case:
	\begin{itemize}
	\item  originally heard by Centuriate/Tribal Assembly
	\item trial by jury (chaired by praetor)
		\begin{itemize}
		\item usually 50 jurors to try a Governor
		\end{itemize}
	\item jury selected from pool ("college")
	\item senatorial vs. equestrian
	\item arraignment before praetor (trial date set)
	\item jury selection
		\begin{itemize}
		\item poor have no “jury of their peers”
		\end{itemize}
	\item witnesses
	\item time limits
	\end{itemize}
\end{itemize}

\subsubsection{Roman Civil Law}
\begin{itemize}
\item civil case: preliminary hearing before praetor
	\begin{itemize}
	\item trial by judge
	\item in minor cases, out-of-court settlement by arbiter
	\item 30 days to pay penalty (but could appeal)
	\end{itemize}
\item Cicero "The Spirit of the Law versus the Letter of the Law“
\end{itemize}
\subsubsection{Lawyers}
\begin{itemize}
\item Lawyers: originally non-professional (patron/friend) and not cheap
	\begin{itemize}
	\item fee limited to 10,000 sesterces (1st c. AD)
	\item paid in cash, property or grain
	\item courtroom tricks
	\end{itemize}
\end{itemize}

\subsection{Police}
\begin{itemize}
\item In rome:
	\begin{itemize}
	\item none at first
	\item Augustus in 6 CE creates 3 urban cohorts (1000 men each) under City Prefect (senator)
	\item deal with crime, fair prices in the markets, crowd control at the Games, and control the city gate
	\item supplemented by the 3,500 men of the Vigiles (firemen)
	\item further supported by the 10,000 members of the Praetorian Guard
	\end{itemize}
\item outside Rome, local soldiers, military garrisons and “policing officials” 
	\begin{itemize}
	\item stationarii (“post-holders”) assigned to man watchtowers and strategic points along the roadways to protect against brigands
	\item also protect important economic areas (ie mines, quarries, estates, plantations, ports)
	\item help suppress crime, root out bands of brigands, and capture escaped slaves
	\end{itemize}
\end{itemize}

\subsection{Punishments}
\begin{itemize}
\item punishments:
	\begin{itemize} 
	\item change over time
	\item different punishments for honestiores and humiliores
	\item rich can always go into exile
	\item poor often beaten, scourged, burned, sent to the mines or arena, decapitated, crucified or drown
	\end{itemize}
\item both sides pay court fees, but the loser forfeits his
	\begin{itemize} 
	\item poor can’t afford fees
	\end{itemize}
\item value of damages decided by assessor 
	\begin{itemize}
	\item 30 days to pay fine/assessment
	\item enslavement for debt abolished in 4th c. BCE
	\end{itemize}
\item death penalty and imprisonment: abolished in 190's BCE (but only for Roman citizens)
	\begin{itemize}
	\item "capital punishment" = exile, loss of property
	\item wealthy can go into exile at any time during a court case and plead “no contest”
	\item lesser penalties: fines, loss of citizenship
	\item jail: not a punishment; holding cells only
	\end{itemize}
\item Parricide:
	\begin{itemize}
	\item convicted sewn up in a sack and drown in a body of water
	\end{itemize}
\item Most severe crimes punished with crucifixion and being thrown into the arena to be torn apart by wild animals (all very public!)
\item Self Defence: you could kill a thief with impunity if the thief
\item was found in your home, or
\item was a thief in the night, or
\item was armed and you called on neighbors to witness you attacking him/her in self defence
\item Slander:
	\begin{itemize}
	\item convicted is clubbed to death
	\end{itemize}
\item Bearing False Witness in Court (ie Perjury):
	\begin{itemize}
	\item death
	\end{itemize}
\item A judge who takes a bribe:
	\begin{itemize}
	\item put to death
	\end{itemize}
\item Highway Robbery
	\begin{itemize}
	\item Crucifixion along the same stretch of road as a warning to others
	\item all cases look at aggravating and mitigating circumstances
	\end{itemize}
\end{itemize}

\section{Early Roman Paganism}
\begin{itemize}
\item much based on Etruscan models
\item gods begin as spirits
\item numen (divine power)
\item later, identification with Greek gods e.g. Vulcan (fire/blacksmith), Neptune (water), Mars (spear/war)
\item numen: attached to gods, groups of people, emperor, family genius (spirit protecting emperor, family, etc.)
\item direction of numen, e.g. Terminus (boundary marker)
\end{itemize}

\subsection{Priesthoods and Sacrifices}
\begin{itemize}
\item A very religious people with many different priesthoods
	\begin{itemize}
	\item all important political and civil affairs needed to be “blessed by the gods” to be successful
	\item if not seen to be favored by the gods, they were cancelled or postponed
	\end{itemize}
\item One Pontifex Maximus (chief priest) chosen for life
	\begin{itemize}
	\item oversees religious orthodoxy and rituals
	\item chooses priests, Vestal virgins
	\item Vestal Virgins (cult of hearth-goddess Vesta; sworn to chastity)
	\end{itemize}
\item Augur
	\begin{itemize}
	\item priests who examine movements of stars, flights of birds – reading these signs called “taking the auspices”
	\end{itemize}
\item Haruspex
	\begin{itemize}
	\item priests who examine entrails of sacrificial animals (the “liver-lookers”)
	\item reading these signs are called “taking the omens”
	\end{itemize}
\item sacrifice: act of piety; worshipper hopes for favour from gods
\item food or liquid burnt on altar (preferably by priest) 
	\begin{itemize}
	\item vows suovetaurilia (sacrifice of pig, sheep and bull)
	\end{itemize}
\end{itemize}

\subsection{Deities}
\begin{itemize}
\item Early deities based on Etruscan religion, e.g. Jupiter (thunder), Ceres (grain), Janus (beginnings), Juno (wife of Jupiter and goddess of women), Minerva (goddess of female handicraft and wisdom)
\item Capitoline Triad (Jupiter, Juno, Minerva): shrine is in the temple on Capitoline Hill
	\begin{itemize}
	\item their cult combines Etruscan, Italic and Greek concepts
	\end{itemize}
\item Adapted/adopted Greek divinities over time
	\begin{itemize}
	\item Ares (god of War) becomes Mars
	\item Aphrodite (goddess of Love) becomes Venus
	\end{itemize}
\end{itemize}

\subsection{Temples}
\begin{itemize}
\item temple: originally were areas for auspices; later, permanent building
	\begin{itemize}
	\item combined Etruscan, Greek and Italic influences
	\end{itemize}
\item usually rectangular; contains cult statue; high podium with stairs
\item temple = house of god, not a place of congregation
	\begin{itemize}
	\item sometimes contained a treasury for offerings as well
	\item altar outside for public display of sacrifice
	\end{itemize}
\item Temple of Portunus,  2nd century BCE
\item Dedicated to Portunus, god of harbors and ports
\item Combines Greek, Etruscan and Roman practices
\item Small rectangular temple built on a raised platform
\item Ionic columns, both full free-standing on the porch and engaged on the exterior cella walls
\item Continual frieze on the entablature
\item One flight of stairs leads up to one front entrance
\item Entrance leads to one cella with the cult image of the god
\end{itemize}

\subsection{Odds and Ends}
\begin{itemize}
\item festivals (holidays): incl. Spectacles (provided by magistrates)
\item Saturnalia (December): slave holiday, reversal of roles, gifts
\item family religion: Lares, Penates (household gods)
\item lararium (shrine of the Lares) = niche in wall for offerings etc.
\item floor must be kept clean (evil spirits); spring "housecleaning“ (get evil out of corners of house)
\item Much superstition
\item Roman Paterfamilias carrying busts of his ancestors as part of annual religious celebration
\item Part of his role as family priest
\item Prayers and make sacrifices for family each morning and night
\end{itemize}

================================================================== \\
END OF MIDTERM ONE MATERIAL \\
================================================================== \\

\section{The Early Empire: The Age of Augustus (63 BCE - 14 AD)}
\subsection{The Death of Julius Caesar}
\begin{itemize}
\item 48 BC Wins Civil War with Pompey
\item Consul 48, 46 and 45 BC
\item Dictator for 10 years in 46 BC (unprecedented)
\item 44 BC Murdered on the Ides (15th) of March
\item Brutus and Cassius the leading conspirators (of 60)
	\begin{itemize}
	\item proclaim the death of a “tyrant” and the “restoration of the Republic”
	\end{itemize}
\item Cleopatra and Caesarion return to Egypt
\item Marc Antony momentarily holds power
\item The Rise of Octavian Caesar
\item 63 BC born a sickly(epileptic), but handsome, brave and scholarly child
\item 44 BC adopted by great uncle Julius Caesar in his will (only 18 years old!)
	\begin{itemize}
	\item Was studying in northern Greece at the time
	\item Quickly showed a shrewd and forceful personality
	\item Saw power of his uncle, and risks of claiming his inheritance
	\item Saw Marc Antony (Caesar’s friend and second in command) as his major rival to taking his inheritance
	\item Antony felt he should be Caesar’s heir
	\item Portrait Bust of the Young Octavian Caesar
	\end{itemize}
\item Changes his name from Octavian Caesar to Gaius Julius Caesar Octavianus
\item Gathers many of Caesar’s veterans to him with promises of bonuses
\item Marches on Rome and demands his inheritance and Caesar’s Consulship
\item Has Julius Caesar deified
\end{itemize}

\subsection{Octavian in Rome, 43 BC}
\begin{itemize}
\item Senate does not wish another civil war
\item Most troops won’t fight against him, and people love him as Caesar’s heir
\item Senate see ambition of Marc Antony, and wish to “use” Octavian against him
\item Marc Antony turns over inheritance and Senate grants him a Consulship
\item Both try to “use” Octavian for their own purposes
\end{itemize}

\subsection{2nd Triumvirate (43 BC)}
\begin{itemize}
\item Octavian, Lepidus, Antony create a new alliance to share power and keep the peace
\item Officially called the “Triumvirs for the Restoration of the State” (Triumviri Rei Publicae Constituendae)
	\begin{itemize}
	\item Legal alliance, ratified by the Senate
	\item purpose is to bring Caesar’s assassins to justice
	\end{itemize}
\item Lepidus and Antony hope to undermine Octavian and remove him from power
	\begin{itemize}
	\item Octavian (western and northern provinces)
	\item Marc Antony (Greece, Asia and Egypt)
	\item Lepidus (Africa)
	\item Proscriptions held to eliminate all their enemies (300 Senators, 2000 Equestrians)
	\item Cicero, who denounced Marc Antony also killed
	\end{itemize}
\item 42 BC Battle of Philippi (Brutus and Cassius and the last Republican army defeated)
	\begin{itemize}
	\item civil war ends
	\item 2nd Triumvirate begins to disintegrate
	\end{itemize}
\item 40 BC Octavian’s sister, Octavia the Younger, marries Marc Antony to cement their alliance
\end{itemize}

\subsubsection{Cleopatra VII}
\begin{itemize}
\item 41 BC Marc Antony meets her at Tarsus
	\begin{itemize}
	\item explain her role in the civil war
	\item came on a barge dressed as Venus
	\end{itemize}
\item Antony spends more time in the East
\item Marries Cleopatra and has 3 children
\item Gives most of Eastern Roman Empire to Cleopatra and her children
\item Scandalous behaviour and an insult to Octavia and Octavian
\end{itemize}

\subsubsection{The End of the 2nd Triumvirate}
\begin{itemize}
\item Octavian stays in the West and builds his powerbase in Rome
\item 36 BC Lepidus tries to invade Sicily, but army defects to Octavian
	\begin{itemize}
	\item Lepidus retires and becomes Pontifex Maximus
	\end{itemize}
\item 32 BC Marc Antony divorces Octavia
	\begin{itemize}
	\item Octavian convinces Senate to declare war on Cleopatra (clever move!)
	\end{itemize}
\end{itemize}

\subsubsection{The End of the Antony and Cleopatra}
\begin{itemize}
\item 31 BC Battle of Actium
	\begin{itemize}
	\item Antony and Cleopatra commit suicide
	\item their children raised by Octavia
	\item Caesarion murdered on Octavian’s orders (because "One Caesar is enough“)
	\end{itemize}
\item Egypt annexed as a Roman Province
\end{itemize}

\subsubsection{The Octavian’s Political Options}
\begin{itemize}
\item 27 BC had settled all issues dealing with civil war and annexation of Egypt
\item He holds sole power as Rome’s greatest warlord
\item Now what?
	\begin{itemize}
	\item a) retire and risk civil war or assassination
	\item b) wield power openly
	\end{itemize}
\item He created a third option: Create the
\item appearance of a Republic, by giving most of his
\item power back to the Senate through a slow
\item process of trial and error to see what would be
\item tolerated
	\begin{itemize}
	\item rule from the shadows, not the stage!
	\end{itemize}
\end{itemize}

\subsection{The Creation of an Emperor}
\begin{itemize}
\item 27 BC offers to give back all power to the Senate (during his 7th Consulship) but is refused
	\begin{itemize}
	\item "Restores the Republic" (age 35)
	\item Proconsular Power
	\item "Augustus" (revered one)
	\item month of August
	\item Consul 13 times
	\item Tribune 37 years in a row!
	\item Ranking Senator for 40 years
	\item Censor
	\item Pontifex Maximus
	\item Imperator 21 times
	\item 2 BC "Father of the Country"
	\item Princeps ("first citizen") vs Rex
	\end{itemize}
\end{itemize}

\subsubsection{The Principate}
\begin{itemize}
\item “The Rule of the First Citizen”
\item Façade of Republicanism
	\begin{itemize}
	\item all runs and appears as it should
	\item Augustus runs all from the shadows vs open rule
	\end{itemize}
\item Pax Romana “The Roman Peace”
\item 25 BCE Temple of Janus closed (means the Empire is at peace)
\item Aided by his wife, Livia
\item Groomed a fatherly and pious image
\item Imperial bureacracy
	\begin{itemize}
	\item slaves and freedmen in the Imperial palace
	\end{itemize}
\item Julia
	\begin{itemize}
	\item only natural child
	\item exiled for adultery
	\end{itemize}
\item Livia (Octavian’s 2nd wife)
	\begin{itemize}
	\item matronly and respectable first lady
	\item of the Claudian clan
	\end{itemize}
\item Designate heir
	\begin{itemize}
	\item all die, only left with Tiberius (Livia’s son)
	\item Senate allows him to designate a successor (creation of an imperial family line)
	\end{itemize}
\item "I found Rome a city of brick and left it a city of marble“
\item 14 CE population of all free people in the Roman Empire estimated at 45 million
\item 14 CE dies – “If I’ve played my role well, applaud”?!
\item Senate deifies him, builds temples and creates a priesthood in his honour
\end{itemize}

\subsubsection{The 4 Pillars of the Principate}
\begin{itemize}
\item Must please 4 groups in Rome to keep power:
	\begin{itemize}
	\item a) People
		\begin{itemize}
		\item police, fire dept, peace and prosperity
		\item public works, libraries, fountains
		\item forums, roads, temples, jobs
		\item schools, theatres, docks
		\item grain supply
		\end{itemize}
	\item b) Equestrians
		\begin{itemize}
		\item More money through building contracts
		\item Governors of less important and more unsettled provinces
		\end{itemize}
	\item c) Senate
		\begin{itemize}
		\item honour and respect
		\item Governors of more civilized and secure provinces
		\item Governors given a salary to encourage efficient rule
		\end{itemize}
	\item d) Army
		\begin{itemize}
		\item decrease legions from 60 to 28
		\item better pay, conditions, plus bonuses and land upon retirement
		\item 28 new colonies of retired veterans across Empire
		\item new conquests in Germany and the East
		\item Praetorian Guard (9000 soldiers) created to protect Emperor
		\end{itemize}
	\end{itemize}
\end{itemize}

\section{The Julio-Claudian Emperors (14 - 68 AD)}
\begin{itemize}
\item 14 AD Augustus dies
\item Senate so inspired they allow Augustus to designate an heir
\item Creation of a “Caesar”, not “Rex”
\item Rome accepts one man rule in the façade of Republicanism
\item In reality, an Emperor who rules the Roman Empire as his own private domain has been created
\item Augustus had no natural son, only Julia
	\begin{itemize}
	\item his first four designated heirs all died before he did
	\item Eventually, he had to designate Tiberius, Livia’s son from her previous marriage (didn’t really like him)
	\item Livia was pregnant with his brother Drusus when she left Tiberius’ father and married Octavian!
	\end{itemize}
\end{itemize}

\subsection{Tiberius (14-37 AD) The 1st Julio-Claudian Emperor}
\begin{itemize}
\item age 52 at the time of Augustus’ death
	\begin{itemize}
	\item burnt out, brooding, resentful and heavy handed, with no political finesse
	\item hesitant to take power (but had been a good general and organizer)
	\item lived in awe of Augustus
	\item upset that he was forced to divorce a wife he loved to marry Julia
	\end{itemize}
\item Started well
\item Deified Augustus
\item consult the Senate at first
	\begin{itemize}
	\item quickly becomes a paranoid resentful old man
	\item begins treason trials against perceived enemies
	\end{itemize}
\item 26 AD semi retire to Capri
	\begin{itemize}
	\item left his nephew, Caligula (by brother Germanicus) in Rome to run Empire
	\item Caligula aided by Macro (Prefect of the Praetorian Guard)
	\end{itemize}
\item 29 AD Livia dies
\item Spent most time in his "pleasure palace"(?)
	\begin{itemize}
	\item porn collection (on pottery) and “little fishes” (little boys and little girls...)
	\item abused Senators and their wives, would have sex with wives and then comment.  Could not comment on it or face treason.
	\item drinking, stripers
	\end{itemize}
\item 37 AD dies (stroke or murder?)
	\begin{itemize}
	\item no heir, but Caligula and Macro are present
	\item Praetorian Guard proclaim Caligula Emperor (precedent)
	\item “To the Tiber with Tiberius” (not deified)
	\item People didn't like this guy...
	\end{itemize}
\end{itemize}

\subsection{Caligula (37-41 AD)}
\begin{itemize}
\item 25 years old (son of Germanicus, nephew of Tiberius)
\item Named Gaius Julius Caesar Augustus Germanicus
	\begin{itemize}
	\item Nicknamed "little boots“ (Caligae)... obviously not how people addressed him.
	\item nervous youth (2 elder brothers killed by Praetorian Guard)
	\item hid a nasty and sadistic streak, calm demeanor.
	\end{itemize}
\item Greeted by the Roman people with much celebration
	\begin{itemize}
	\item Not Tiberius, not a drunkard, not a jerk.
	\end{itemize}
\item Immensely popular at first (Germanicus’ son and not Tiberius)
\item Paid bonuses to Praetorian Guard (bad precedent - this will continue on to prevent revolts)
\item Shared money of Tiberius and Livia’s will, ended treason trials
\item Liked to be at the games and among common people
\item Restored the authority of the Assemblies
\item Campaigned with the army and held their loyalty
\item October 37 AD fell ill (7 months into reign)
	\begin{itemize}
	\item fell into a coma
	\item recovered after a few months
	\item saw self as the god Jove
	\item Schizophrenia(?), we're not sure... coma could have pushed him over the edge.  He goes nuts.
	\end{itemize}
\item Druscilla (favourite sister, he wanted to marry and make a goddess)
\item Incitatus (favourite race horse)
\item Marble stable, purple robe, wanted to make him a Senator and Consul
\item Declared War on English Channel (My new favorite guy), actually got the soldiers to go to the channel and collect shells
\item Brothel in Palace (to make money - senators wife become the whores)
\item Spent 3 billion sesterce surplus of Tiberius
\item Wanted Governor of Judea to put statue of himself in Holy of Holies (temple) in Jerusalem
	\begin{itemize}
	\item Stalls him, of course he never builds it
	\end{itemize}
\item Treason trials
	\begin{itemize}
	\item Beginning of trial, grab all possessions... so good way to make money.
	\end{itemize}
\item Murdered by Praetorian Guard
\end{itemize}

\subsection{Claudius (41-54 AD)}
\begin{itemize}
\item 51 years old uncle of Caligula
\item Praetorians found him hiding in the imperial palace
\item Paraded him before the Senate and proclaimed him Caesar (another bad precedent! [The fact he was made emperor by the guard])
\item A physical wreck
	\begin{itemize}
	\item physical afflictions, drooled, twitched, stuttered, runny nose, limp (club foot)
	\item alcoholic
	\item ridiculed by Caligula (threw walnuts at him!)
	\item hidden in the palace by Augustus
	\item 10/10 would hire for boss of the empire
	\end{itemize}
\item Great scholar
	\begin{itemize}
	\item Had nothing better to do then learn
	\item last Etruscan speaker
	\item wrote on Etruscans, Carthaginians and Augustus (whom he admired)
	\end{itemize}
\item Britain conquered, public works, finished Augustus' wishes for the city.
\item Senate disliked him for lecturing them on their duties
\item No treason trials
\item unlucky in love!
	\begin{itemize}
	\item Messalina (wife of Claudius)
		\begin{itemize}
		\item descendant of Augustus
		\item bathed in donkey’s milk!
		\item bore Claudius 2 children, Ocatavia and Britannicus
		\item 48 AD executed with her lover
		\item she went on a public hunnymoon with another man... is she stupid.
		\end{itemize}
	\item Agrippina (sister of Caligula, niece of Claudius and mother of Nero from a previous marriage)
		\begin{itemize}
		\item a total terror!
		\item her fourth marriage, his third
		\item 50 AD made Claudius adopt Nero as his principal heir (4 years older than Brittanicus)
		\item 53 AD made Claudius marry Octavia to Nero
		\item evil, manipulator, vicious
		\end{itemize}
	\end{itemize}
\item 54 AD poisons Claudius’ mushrooms at a banquet and kills him
	\begin{itemize}
	\item even gets doctor to put poison on the feather used to induce the vomiting.
	\item deified!? (Kind of a joke... just the thing you do)
	\end{itemize}
\end{itemize}

\subsection{Nero (54-68 AD)}
\begin{itemize}
\item 16 years old when he is made Caesar
\item Born in 37 AD (a descendant of Augustus on both sides of his family)
\item Very popular at first (puppet of mom)
\item Guided by Agrippina, Seneca (philosopher) and Burrus (Prefect of the Praetorian Guard)
	\begin{itemize}
	\item very artsy and dressed like a charioteer in public (a bit scandalous!)
	\end{itemize}
\item Agrippina tries to control Nero and be coruler of Rome
	\begin{itemize}
	\item uses the threat of Britannicus (The "actual" heir) to keep him in check
	\item Nero rebels and wants to rule alone
	\end{itemize}
\item 55 AD divorces Octavia and later poisons Britannicus (starts breaking free of his mother's chains)
\item 59 AD has Agrippina put to deal (“Stab me where I gave birth to the Viper!”) (This is like the 10th time he's tried to kill her... she evades him every time)
	\begin{itemize}
	\item marries Poppea Sabina (ruthless, ambitious, bisexual)
	\item the right woman in his mind
	\end{itemize}
\item 62 all advisors are dead or go into retirement
	\begin{itemize}
	\item more interested in theatre singing, arts, music and horse racing than ruling
	\item performs “his works” in public
	\item street performers considered low class (down with prostitutes etc...)
	\item had to respect him fully during concert, no sleep, no peeing, even someone couldn't leave while in labor.  All considered treasonous.
	\end{itemize}
\item 64 AD Great Fire of Rome
	\begin{itemize}
	\item burns for days
	\item 3/4 of city destroyed
	\item Nero watches and composes song of the burning of Troy! (plays lyre)... hey man, when inspiration strikes.
	\item Eventually sends army to start destroying buildings to create a fire wall, citizens start turning.
	\end{itemize}
\item 64 AD beginning of Christian persecutions
	\begin{itemize}
	\item Nero uses Christians as a scapegoat to take suspicions off of him
	\item At this point, Christians were just weirdos.
	\item Burned, torn to death, stolen from...
	\end{itemize}
\item 64 AD Golden House “Domus Aurea”
	\begin{itemize}
	\item covered 100 to 300 acres along the Palatine, Esquiline and Caelian Hills
	\item Suetonius describes it as "ruinously prodigal“
	\item it included groves of trees, pastures with flocks, vineyards and an artificial lake
	\item rus in urbe, "countryside in the city“
	\item massive public amusement park and gardens, and private palace
	\item Oculus in ballroom
	\item over 300 rooms. but no bedrooms
	\end{itemize}
\item Nero commissioned the creation of a colossal 35.5 m high bronze statue of himself, called the “Colossus Neronis”
	\begin{itemize}
	\item the statue was placed just outside the main palace entrance at the terminus of the Via Appia
	\end{itemize}
\item Prowls streets at night assaulting people
	\begin{itemize}
	\item turns good men away and keeps scoundrels close by
	\end{itemize}
\item 65 AD Poppea dies
	\begin{itemize}
	\item drunk, she's pregnant, he freaks out, kicks her.
	\item one of the last checks on Nero
	\item her body was not cremated, but stuffed with spices, embalmed and put in the Mausoleum Augustus
	\item she was given a state funeral and divine honors.
	\item has her stuffed and keeps her around.
	\item good times story about finding a boy who looked like her and him becoming his boy toy.
	\end{itemize}
\item 65 AD Conspiracy of Piso
	\begin{itemize}
	\item murder plot makes him even more paranoid
	\end{itemize}
\item 66 AD Tour of Greece (1808 first prize awards - they just let him win - ego grows)
\item 68 AD Galba Governor of Spain rebels, and others turn on Nero
	\begin{itemize}
	\item Senate declares him (Nero) an outlaw (anyone can kill him on sigh, for a reward)
	\item "Quolis artifex pereo"/"What a great artist dies in me“ - last words, suicide by slave.  10/10.
	\end{itemize}
\item Last of the Julio-Claudians (no heir)
\end{itemize}

\section{Roman Entertainment}
\begin{itemize}
\item children's games:
	\begin{itemize}
	\item evens and odds
	\item pitching/stacking nuts
	\item dolls
	\item "bronze fly"
	\item "jar"
	\item toy carts (mouse chariot races!)
	\item "triangle“(a ball game), etc.
	\end{itemize}
\item Dawn to 7th hour (mid day) is the Roman working day in summer (lots of leisure time)
\item adult games:
	\begin{itemize}
	\item ball
	\item board games
	\item dice
	\item knucklebones
	\end{itemize}
\item dinner parties, banquets (show off wealth), taverns (take out food and prostitutes)
\item During the reign of Claudius there were 159 public holidays/year (not all could afford to take those days off)
\item forum; libraries; triumphs; Campus Martius (public military exercise field); festivals
\item tourism: middle-eastern tours; temples=museums; beaches
\item leaving town: villas, baths, hunting/fishing, beaches (Baiae/Capri)
	\begin{itemize}
	\item best resorts in the south
	\item only the rich could afford this
	\end{itemize}
\item thermae (baths): hot, warm, cold pools; heating system; mineral baths; exercise court; refreshments; locker room; toilets; strigil
\item relatively cheap
\item 170 Baths in Rome at the time of Augustus (clean culture)
\item 4th century 1000
\item Baths of Diocletian held 3000 peopleat a time!
	\begin{itemize}
	\item ”Baths, wine and sex ruin our bodies. But what makes life worth living, except baths, wine and sex?” (epitaph)
	\end{itemize}
\item Strigil, used to clean off a layer of skin.
\item Circus (chariot racing): long track with spine; 7 laps; "factions“
\item Circus Maximus is 550 metres long and holds 250,000
\item 24 races a day (arace is 7 laps of the track)
\item Reds, Blues, Greens and Whites (Racing Factions)
\item amphitheatre (gladiator/animal fights); oval arena e.g. Colosseum(held 50,000)
\item theatre (drama): semicircular; associated with festivals (both a religious and political function)
\item comedies popular, especially Greek “comoedia palliata” (comedy in Greek dress), made fun of famous powerful Romans through proxy of a Greek to be politically safe.
\item pantomimes; music halls (odeons)
\item gladiators=slaves; training school; diff. types of gladiator
\end{itemize}

\subsection{Gladiators}
\begin{itemize}
\item What romans enjoyed the most
\item First Gladiator games (called “Munera”) in 264 BC
	\begin{itemize}
	\item based on Etruscan fuberal rites
	\end{itemize}
\item By the 1st century CE, the schedule of events for most “games” included:
	\begin{itemize}
	\item Venationes (hunts of wild beasts) in the morning
	\item Damnati (execution of convicted criminals) follows
	\item this could also include Dwarves and cripples with blunt wooden weapons
	\item Gladiatorial combat at end of the day
	\end{itemize}
\item Usually slaves
\item "Ave, imperator, morituri te salutant"
\item "Hail, emperor, we who are about to die salute you!“
\item Trajan held 117 days of contests, with 10,000 gladiators fighting and 11,000 animals killed
\item “Thumbs down” = live
\item “Thumbs up” = killing blow to the throat
\item lol the thumb thing is something we have wrong, it's the opposite.
\item much betting on the “great shields” (defensive) or “little shield” (offensive) factions!
\item Successful gladiators became rich celebrities (could even sell sweat for perfumes and good omen), these are the guys.
\item Only fought about 2 to 4 times per year on average
\item A successful Gladiator could be granted a rudis (an engraved wooden training sword) as a symbol of his freedom
	\begin{itemize}
	\item usually granted after 5 victories
	\end{itemize}
\end{itemize}


\subsubsection{Famous Gladiators}
\begin{itemize}
\item Most died in their first fight or two
	\begin{itemize}
	\item a very few fought up to 150 bouts
	\end{itemize}
\item A Gladiator named Flamma (“Fire”) was awarded the rudis four times, but still chose to remain a gladiator
	\begin{itemize}
	\item his gravestone in Sicily includes his record
	\end{itemize}
\item "Flamma, secutor, lived 30 years, fought 34 times, won 21 times, fought to a draw 9 times, defeated 4 times, a Syrian by nationality. Delicatus made this for his deserving comrade-in-arms.”
\end{itemize}

\subsubsection{Types of Gladiators}
\begin{itemize}
\item Various styles of gladiator and fighting developed
	\begin{itemize}
	\item Samnite
		\begin{itemize}
		\item heavy, ornate body armour, with visored helmet and greaves
		\item large, oblong shield and sword or lance
		\item right arm usually armoured
		\end{itemize}
	\item Thracian
		\begin{itemize}
		\item “Light and Fast”
		\item small curved sword and small round or square shield
		\item leather bands on legs and thighs
		\end{itemize}
	\item Myrmillo “the fishman”
		\begin{itemize}
		\item Heavy helmet with a sea fish crest
		\item bare torso and legs
		\item large rectangular or oval shield
		\item dagger or short sword
		\item greave on the left leg
		\item wide leather or metal belt
		\item usually paired against the Retarius
		\end{itemize}
	\item Andabatae
		\begin{itemize}
		\item fully armoured riders on fully armoured horses(cataphracti)
		\item wore heavy helmets with no eye holes!
		\item charged blindly at each other with lances
		\item didn't really take off...
		\end{itemize}
	\item Eques
		\begin{itemize}
		\item cavalry with only a sword or spear and a round shield
		\end{itemize}
	\item Velites
		\begin{itemize}
		\item unarmoured men armed with a spear
		\item erotic, titillating, I mean, these dudes are ripped and naked.
		\end{itemize}
	\item Retiarius
		\begin{itemize}
		\item the “fisherman”
		\item armed with a trident or harpoon, a dagger and a fish net
		\item wore leg or ankle bands and leather or metal shoulder piece on the left shoulder
		\item usually paired against the Myrmillo
		\end{itemize}
	\item Essedarii
		\begin{itemize}
		\item spearmen or archers in two horse chariots
		\item fight men on foot, other chariots, and wild animals
		\end{itemize}
	\item Scissores
		\begin{itemize}
		\item the "carvers“ (little is known about them)
		\item half moon blade
		\end{itemize}
	\item Dimachaeri
		\begin{itemize}
		\item armed with two daggers or swords, and no armour
		\item Gladiatrix (female gladiators)
		\item no helmets
		\end{itemize}
	\item (Boudicea?)
	\item Sagittarii
		\begin{itemize}
		\item armed with bow and arrows
		\end{itemize}
	\end{itemize}
\end{itemize}

\subsubsection{Hermes and Charon}
\begin{itemize}
\item A character dressed as Hermes, messenger of the gods
	\begin{itemize}
	\item poke corpses with red hot wand
	\end{itemize}
\item A character dressed as Charon, ferryman of the dead
	\begin{itemize}
	\item smashed skulls of corpses with a large mallet
	\end{itemize}
\item Libitinarii (“bearers”) then drag corpses out, strip them and toss them in mass graves
\end{itemize}

\subsubsection{The Wild Beast Hunts}
\begin{itemize}
\item called “Venationes”
	\begin{itemize}
	\item animals vs animals
	\item animals vs bestiarius (“animal fighters”)
	\item animals vs wounded/crippled humans
	\item Lions, tigers, panthers, elephants, bears, wolves, bulls, etc
	\item 80 AD 5,000 wild beasts and 4,000 other animals killed in one day
	\end{itemize}
\end{itemize}

\subsubsection{Naumachiae (“Sea Battles”)}
\begin{itemize}
\item Naval battles where the Colosseum was flooded, or combat moved to a nearby lake
\item 52 AD greatest naumachiae on Lake Fucine
	\begin{itemize}
	\item 19,000 gladiators on two fleets of Galleys
	\item Emperor Claudius declared it a draw!
	\end{itemize}
\end{itemize}

\section{Roman Dining}
\begin{itemize}
\item Only real entertainment at night
\item 2 light meals (bread, water, leftovers) for breakfast and lunch
\item 1 main meal in the evening
\item differences between Roman and modern menu
	\begin{itemize}
	\item less fatty
	\item olives, cabbage, beans, leaks, little meat and no sugar
	\item honey, wine and fruit to sweeten their diet
	\item boiled sheep lips common for the poor
	\item much boiling vs baking (no ovens for the poor)
	\end{itemize}
\item All drank wine (heavily diluted)
\item porridge replaced by bread (2nd c. BC): round loaf
\item veggies and fruits (no tomatoes, potatoes or citrus fruit)
\item meat = expensive, esp. beef; poor eat chicken
\item fish and seafood
\item sauces: sweet and sour; garum (pungent black fish sauce)
\item cookbook of "Apicius“
	\begin{itemize}
	\item 450 recipes(including Numidian Chicken, and Anchovy Delight, without anchovies!)
	\end{itemize}
\item breakfast and lunch = snacks (leftovers, or purchased at snack bar)
\item cena (dinner): eaten in triclinium (dining room)
\item 9th hour is the time for major meal (10 hour day)
\item 3 couches ("U" shape) to lie on; round table for food service
\item etiquette: left elbow on cushion, feet to right; plate in left hand, eat with right (fingers, unless spoon needed; no fork or knife)
\item napkins (provided, or bring your own to wrap leftovers)
	\begin{itemize}
	\item often of rich cloth
	\item Catullus calls Marrucinus “the napkin thief”
	\end{itemize}
\item "mixed" dinner parties (women sit on chairs)
\item waiters (fancy dress and hairdo/big perm) vs. busboys
\item courses: appetizers, entrées (numerous), dessert
\item Trimalchio's dinner (in novel by Petronius)
\item Vomitarium
\item dinner from mid afternoon to ?
\item skeleton displayed: enjoy life while you can!
\end{itemize}

\subsection{Trimalchio's Feast}
\begin{itemize}
\item Trimalchio is a character in Petronius’ Satyricon
	\begin{itemize}
	\item he is a freedman who attains great wealth and power
	\item shows off his “class” by hosting a lavish “Banquet of Trimalchio”
	\item golden cups and plates
	\item wears a scarlet cloak as a sign of his wealth
	\item wears large gold rings
	\item napkins have broad purple strip (mock Senator toga!)
	\end{itemize}
\end{itemize}

\section{Death and Burial}
\begin{itemize}
\item funerary rites = show of pietas (respect, devotion)
\item dying person placed on ground; last words prophetic
\item after death: mourning; body washed, anointed, dressed (toga)
\item Mourning
\item for parents and children over 6 – 1 year
\item for children under 6 – 1 month
\item for a husband or wife – 10 months
\item for a close blood relative – 8 months
\item placed on funerary couch; hearth extinguished
	\begin{itemize}
	\item wax death mask made
	\item actors hired
	\end{itemize}
\item flowers, lamps/candles, cypress boughs placed in front of house
\item burial (for poor in mass graves) vs. cremation (for rich)
\item funeral procession to cemetery (outside the pomoerium)
	\begin{itemize}
	\item originally at night (torchlight)
	\item later by day, except poor, children
	\item musicians, torchbearers, professional mourners, clients, ancestral portraits, clowns, dancers (strange Roman sense of humour)
	\end{itemize}
\item wooden coffin (poor) vs. elaborate tomb (rich)
\item sarcophagus (stone coffin with lid, often decorated with reliefs)
\item grave goods (pottery, jewelry, coin to pay passage to underworld)
\item mausoleum (tomb building); sometimes arranged in "streets“ (necropolis)
	\begin{itemize}
	\item rich have monuments lining the road to Rome
	\item poor have ashes in boxes in niches of walls of underground chambers
	\end{itemize}
\item tombstone information
	\begin{itemize}
	\item name
	\item age
	\item origin
	\item career
	\item relatives
	\end{itemize}
\item deceased portrayed on tomb reliefs (family groups; kids with pets)
\item tombs protected by curses; often found on roadside
\item cremation: funeral pyre (possessions burned with deceased)
\item ashes placed in urn or amphora (e.g. Ostia)
\item columbarium (underground chamber with niches for urns)
\item ie Roman catacombs
\item collegium (social club providing funerals and banquets for its members; often based on a particular trade)
\end{itemize}

\section{69 AD “The Year of the Four Emperors”}
\begin{itemize}
\item 68 AD Galba Governor of Spain rebels, and others turn on Nero
\item Nero commits suicide
\item 69 AD "The Year of the Four Emperors“
\item Galba
	\begin{itemize}
	\item supported by the Spanish legions and Praetorian Guard
	\item too old, mean and cheap
	\item Augustus June 8, 68 AD to January 15, 69 AD
	\end{itemize}
\item Otho (friend of Galba) 
	\begin{itemize}
	\item supported by Portuguese legions
	\item friend and fellow debaucher of Nero as well (until his wife Poppaea was taken by Nero and he was sent away as Governor of Lusitania)
	\item thought he should succeed Nero
	\item losses Battle of Cremona to Vitellius and commits suicide
	\item Augustus from January 15 to April 17, 69 CE
	\end{itemize}
\item Vitellius
	\begin{itemize}
	\item supported by the Rhine legions
	\item declared by the legions, but not want the job
	\item Augustus from April 16 to December 22, 69 AD
	\item tried to abdicate but was beheaded by Vespasian’s troops and his head paraded around Rome
	\item “Yet, I was once your Emperor” were his final words
	\end{itemize}
\item Vespasian
	\begin{itemize}
	\item supported by the eastern legions in Syria-Judaea
	\item acclaimed by the legions and joined by the Danube legions
	\item Augustus from 69 79 AD
	\item founded Flavian Dynasty
	\end{itemize}
\end{itemize}

\subsection{Titus Flavius Vespanasius/Vespasian (69-79 AD)}
\begin{itemize}
\item Born 9 AD
\item A blunt, honest soldier
\item wished to make right the wrongs of previous Emperors
\item Reform Rome politically, socially, and economically
\item good administrator
\item appoints good men and enforces competence
\item Equestrians begin to replace freedmen in civil service/imperial bureaucracy (money and higher status in civil service)
\item Aristocrats begin to accept money in place of land for payment (a more liquid asset)
\item frugal and stabilizes economy (tax everything!)
\item pay toilets
\item extend Roman citizenship in the western provinces
\item conquer Judea and consolidate frontiers
\item began Colosseum
\item "Ut puto deus fio" "I think I'm becoming a god"
\item deified
\end{itemize}

\subsection{Titus (79-81 AD)}
\begin{itemize}
\item elder son of Vespasian
\item great general
\item Put down Jewish Revolt and destroyed the Temple of Solomon in 70 AD
\item popular, generous and efficient
\item open Colosseum (Flavian Amphitheatre vs Colossus of Nero)
\item criticized for slow response to aiding the victims of the Mt. Vesuvius eruption in 79 AD
	\begin{itemize}
	\item buried Pompeii and Herculaneum
	\end{itemize}
\item Died of disease while campaigning in the East
\item deified
\end{itemize}

\subsection{Domitian (81-96 AD)}
\begin{itemize}
\item younger brother of Titus
	\begin{itemize}
	\item lived in his brother’s shadow
	\item a nasty and murderous man
	\end{itemize}
\item tried to restore old gods
	\begin{itemize}
	\item anti-Christian and Eastern cults
	\item persecute Christians
	\end{itemize}
\item 89 AD rebellion flamed his paranoia
	\begin{itemize}
	\item saw conspiracies everywhere
	\item began treason trials against Senators
	\end{itemize}
\item 96 AD murdered by the Praetorian Guard
	\begin{itemize}
	\item damned by the Senate and all his statues torn down
	\item not deified
	\item no heir
	\end{itemize}
\item End of the Flavian line
\end{itemize}

\section{Provincial Administration}
\begin{itemize}
\item Rome had c. 48 Provinces at its peak
\item Roman control made use of local elites
	\begin{itemize}
	\item much toleration of local politics, culture and religion
	\end{itemize}
\item client kings (run internal affairs and friendship with Rome and look like status quo in effect)
	\begin{itemize}
	\item knew the people, provide troops, money, supplies, spies
	\item mostly in East; client kingdoms later become provinces (ie Judea)
	\end{itemize}
\item provincial towns run by local council
	\begin{itemize}
	\item local magistrates get citizenship, set model of Romanization (loyalty and status)
	\end{itemize}
\end{itemize}

\subsection{Types of Provinces}
\begin{itemize}
\item Republic: provinces governed by Senatorial magistrates
\item Empire: 2 main types of province (senatorial, imperial) + Emperor’s provinces
\item senatorial: governor = proconsul (imperium)
	\begin{itemize}
	\item usually a military man/Senator
	\end{itemize}
\item imperial: governor = legate (hand-picked)
	\begin{itemize}
	\item an Equestrian
	\end{itemize}
\item Emperor's provinces = on frontiers
	\begin{itemize}
	\item usually on crucial frontiers (ie Persia) or politically fragile (ie Armenia)
	\item also controls Egypt (grain supply)
	\end{itemize}
\end{itemize}

\subsection{Provincial Concerns}
\begin{itemize}
\item provincial assembly could complain to Rome (ie corrupt or harsh Governor, high taxes, forced labour, crumbling cities, etc)
\item rescript (emperor's reply)	
	\begin{itemize}
	\item ie Pliny (Gov of Bithynia-Pontus) and Emperor Trajan concerning Christians
	\end{itemize}
\item Cursus Publicus (Imperial Postal Service
	\begin{itemize}
	\item checkpoints and rest stations
	\end{itemize}
\end{itemize}

\subsection{Provincial Magistrates}
\begin{itemize}
\item Quaestor (senatorial prov.) = treasurer
	\begin{itemize}
	\item taxes collected by publicani (Republic), later by towns (Empire)
	\item pay for Governor, government and military
	\end{itemize}
\item local magistrates: Duovirs (judges, census), Aediles (works, festivals), Quaestors (finances)
\item Procurator: looks after emperor's property in imperial provinces (land, palaces and military bases)
	\begin{itemize}
	\item handles all finances (no Quaestor) and watches Governor (no corruption or mismanagement)
	\end{itemize}
\end{itemize}

\subsection{Benefits to Provinces}
\begin{itemize}
\item urbanization: introduced by Rome in western provinces (few major cities before)
	\begin{itemize}
	\item grow up around military camps
	\end{itemize}
\item town-country relations (rural food, urban merchandise)
	\begin{itemize}
	\item often see each other as inferior
	\end{itemize}
\item colonia ("colony") a settlement of Roman citizens, veterans and surplus population in captured territory)
	\begin{itemize}
	\item security and begin Romanization.
	\end{itemize}
\item municipality (native town granted Latin right)
	\begin{itemize}
	\item first step to citizenship + economic advantages
	\end{itemize}
\item all other towns = unprivileged
\item also animosity between “natives” and Roman army
\end{itemize}

\section{Roman Philosophy}
\begin{itemize}
\item from the Greek "Philo"/lover "Sophia"/of wisdom
\item guidance for life; intellectual pursuit for the leisurely upper classes
	\begin{itemize}
	\item logically suited their legal minds
	\end{itemize}
\item Lower and uneducated classes stick to traditional religion and cults
\item Many conservative Romans consider it impractical and a waste of time
\end{itemize}

\subsection{Epicureanism}
\begin{itemize}
\item founded by Epicurus (Athens, 4th c. BC)
	\begin{itemize}
	\item all knowledge based on perception of senses
	\item gods exist but don't influence human affairs
	\item see religion as a “fairy tale” vs faith
	\item open your mind to understand your world through your senses VS live in myths and misperceptions
	\item live simply since luxury and possessions cause pain and stress
	\item free self from unnecessary pain and anxiety
	\item have piece of mind vs physical pleasure
	\end{itemize}
\item gods = products of atomic system, same as us
	\begin{itemize}
	\item don't fear gods or worry about afterlife (there isn't any)
	\item enjoy life while you can; strive for happiness
	\item avoid pain (Carpe Diem “Seize the Day”)
	\end{itemize}
\item 173 BC Epicureans banned from Rome (for "pleasures")
\item Lucretius (1st c. BC) "Nature of Things": good poetry, but fails to convert Romans to Epicureanism
\item Epicureans criticized for "eat, drink and be merry" attitude, but their idea of pleasure = stress-free life, music, friends
	\begin{itemize}
	\item some confuse Epicureanism with hedonism
	\end{itemize}
\end{itemize}

\subsubsection{The Atomic Theory of Democritus}
\begin{itemize}
\item Philosophy included the atomic theory of Democritus (5th c. BC)
	\begin{itemize}
	\item the universe is made up of Atoms and Void
	\item Atoms and Void combine in patterns that create everything in the universe
	\item Atoms are eternal and indestructible
	\item Death = Atoms break apart and form something new
	\item don’t fear death since you will be reborn
	\item gods = products of atomic system, same as us
	\item don't fear gods or worry about afterlife (there isn't any)
	\item enjoy life while you can; strive for happiness
	\item avoid pain (Carpe Diem “Seize the Day”)
	\end{itemize}
\end{itemize}

\subsection{Stoicism}
\begin{itemize}
\item founder Zeno of Citium
\item tauth in Athens 4th c. BC)
\item taught in Stoa (porch)
\item Stoicism reaches Rome 2nd c. BC; appeals to Roman temperament
\item Most popular philosophy
	\begin{itemize}
	\item emotions are irrational and harmful
	\item live in harmony with nature, pursue the "good life”
	\item conscience, duty = keys to moral perfection (self-discipline, perseverance and steadfastness)
	\end{itemize}
\item universe governed by divine intellect, in which humans share
\item virtue is the only good: brings man nearer to gods
	\begin{itemize}
	\item virtue involves hardship, simple living, ignoring distractions
	\item bear up under your burdens and don’t complain (“Keep Calm and Carry On”)
	\item don’t let physical/bodily needs dictate the best action to take, use logic at rational thought
	\end{itemize}
\item if all else fails: suicide (free self from irrational constraints)
\item Cato, Brutus, Seneca: famous Stoics, all committed suicide
\item Brotherhood of Man = rationale for Empire
\item Nero and Flavians banned Stoics
\item 2nd c. AD: Marcus Aurelius (Stoic emperor), Meditations
\end{itemize}

\section{Roman Music}
\begin{itemize}
\item Central component to Roman religion, festivals and funerals
\item Much music used in Roman comedies
\item Woodwinds:
	\begin{itemize}
	\item Tibia (orig. a shin-bone, later wood)
	\item diff. Lengths, played in pairs (treble and bass)
	\item sacrifices, funerals, etc.
	\item Pan pipes (7+ tubes of diff. Length tied together), like harmonica
	\item Bagpipes; water organ (bronze pipes; played at sports events)
	\end{itemize}
\item Brass:
	\begin{itemize}
	\item tuba (straight horn without valves, flared at one end)
	\item cornu (curved horn, flared at one end, shaped like ¢)
	\item  cornu used in Roman military
	\end{itemize}
\item Strings:
	\begin{itemize}
	\item lyre (hand-held harp; tortoise shell as resonator)
	\item cithara (large wooden version of same; up to 18 strings)
	\item lute (has neck and bridge like modern guitar, but no frets)
	\item played with a fingers or wand (no bows)
	\end{itemize}
\item Percussion:
	\begin{itemize}
	\item mostly “Eastern” instruments
	\item castanets
	\item cymbals
	\item kettledrums (military use in the East)
	\item tambourines
	\item sistrum (bronze rattle), used especially in cult of Isis
	\end{itemize}
\item 115 BC - ban on foreign musical instruments
	\begin{itemize}
	\item not appealing to Roman ears
	\item usually associated with strange religious cults
	\end{itemize}
\item Vocal:
	\begin{itemize}
	\item chorus in theatre (mixed, but sang in unison) or solo
	\item poetry was "sung" or read to music
	\item voice exercises and coaches
	\item often a recitation of your own work at a dinner party
	\end{itemize}
\item Dance:
	\begin{itemize}
	\item originally leaping in triple time
	\item Greek dancing arrives 2nd c. BC but distrusted by adult males (some “orgiastic”, other have men touching men)
	\item involved moving body, singing, and playing an instrument
	\item famous dancing girls of Cadiz (ie belly dancers)
	\end{itemize}
\item “Proper” people are not wild dancers
\item “Public” dancers were seen to be low class entertainers
	\begin{itemize}
	\item under Empire, less prejudice against dancing (but no touching! NO TOUCHING! NO TOUCHING!)
	\end{itemize}
\end{itemize}

\section{The ``Five Good Emperors''(96-192AD)}
\subsection{Nerva (96-98)}
\begin{itemize}
\item Nerva (96-98)
\item 96 AD Praetorian Guard murder Domitian
\item Senate nominates him, quickly, as emperor to avoid civil war
\item elderly, childless, ideal senator, he won't be around long, he's a nice guy.
\item Prestigious and old, but no leader
\item suspected by army (not a military man)
\item reduces taxes, recalls exiles, gives land to poor, does a bang up job.
\item alimenta (scheme to help farmers and needy children)
\item 97 AD adopts Trajan (respected and popular general) as son and successor
\item dies of old age after 16 months (stroke after shouting at an officer?)
\end{itemize}

\subsection{Trajan (98-117 AD)}
\begin{itemize}
\item Experienced general of the Rhine and Danube, no civil war to become emperor.  Very popular guy.
\item from Spain (first provincial emperor), up until now, always Italian.
\item Put fiscal restraints onto overspending cities and provinces (try to balance the budgets)
\item expanded empire to its greatest extent
\item invasion of Dacia (across the Danube), extremely successful general.
\item Trajan's Column (depicts Dacian war); Dacian gold, slaves
	\begin{itemize}
	\item gives 75 denarii to each citizen of Rome from spoils (congiaria - 3 months pay)
	\end{itemize}
\item annexes Arabia
\item conquers Parthians (weak) = 3 new provinces (Iraq, Iran)
\item eastern war drains other frontiers
\item massive revolts as a result of financial strains of war (extremely expensive to go to war)
\item Dies on way home from Parthia
\end{itemize}

\subsection{Hadrian (117-138)}
\begin{itemize}
\item Another Spaniard
\item Curly hair and beard becomes fashionable
	\begin{itemize}
	\item "adopted" by Trajan (forged will?)
	\item groomed to be Trajan’s heir for 20 years
	\item Aug 8 Trajan dies
	\item Aug 9 pronouncement the Hadrian is Trajan’s adopted son and heir
	\item Aug 11 Trajan’s death proclaimed 
	\item  Mostly to legitimize what everyone wanted anyway.
	\end{itemize}
\item An excellent emperor
\item Brings peace, prosperity and order
\item Ruled with a personal, “hands on” style, one of the boys. (to avoid revolts)
\item abandons new provinces (except Dacia which was Romanized) to shorten the frontier
	\begin{itemize}
	\item empire becoming too large and expensive to control.
	\item retire borders to defensible  lines (ie Rhine and Danube rivers)
	\end{itemize}
\item visits provinces and frontiers: Hadrian's Wall (practical, but defensive policy)
\item 2nd Jewish War(132-135) destroys remain of the Old Temple of Solomon at Jerusalem
	\begin{itemize}
	\item replaced by a Temple of Jupiter (blasphemy?)
	\item begins Diaspora, Romans see this as a victory (Jews banned from Jerusalem)
	\end{itemize}
\item expands Athens
	\begin{itemize}
	\item loved Greek culture, architecture, philosophy.
	\item builds the Pantheon in Rome
	\end{itemize}
\item rebuilds treasury
\item excuse \$900 million in back taxes (stimulate the economy instead of extorting money).
\item Gives money to poor and to poor Senators (so they can remain Senators = loyalty)
\item 130 AD - boyfriend Antinuus commits suicide to ensure safety of Hadrian, interesting to note the opinions on homosexuality in the Roman empire (divine honours and city of Antinuopolis founded)
\item top civil service posts go to equites (more trust put into equites)
\item bad relations with Senate (4 senators executed; jealous of equites)
\item Empire run by Emperor and Advisors
	\begin{itemize}
	\item Senate becomes more of a “town Council”
	\end{itemize}
\item adopts Antoninus Pius; Antoninus adopts Marcus Aurelius and Lucius Verus (secure line of succession – two generations of successors to guarantee no more civil wars)
\item in old age developed a nosebleed for two years, and suffered a slow, lingering death from water accumulation in the body
	\begin{itemize}
	\item asked slaves to kill him, wouldn't because they love him so much
	\end{itemize}
\end{itemize}

\subsection{Antoninus Pius (138-161)}
\begin{itemize}
\item insists on Hadrian's deification (his adopted son)
\item excellent administrator and model of behaviour
\item maintains status quo
\item legal reforms
\item teachers to be paid by cities
\item university at Athens
\item Antonine Wall (north of Hadrian's): shorter but less secure
\item Does a good job, nothing flashy or overly notable, but great a great emperor.
\end{itemize}

\subsection{Marcus Aurelius (161-180)}
\begin{itemize}
\item reluctant emperors
	\begin{itemize}
	\item a good man at a bad time
	\end{itemize}
\item insists that his ``brother'' Lucius Verus be co-emperor (161-169)
	\begin{itemize}
	\item useless emperor
	\item dies of apoplexy
	\item Empire needs at least 2 men to run it
	\end{itemize}
\item Stoic philosopher-statesman, not soldier
\item German tribes swarm across Danube and threaten Italy
	\begin{itemize}
	\item ``barbarians'' begin to breach the Roman frontiers
	\end{itemize}
\item long wars
\item Plague ravages Italy
	\begin{itemize}
	\item beginning of the long “decline and fall of Rome”
	\end{itemize}
\item suffered great chest and stomach pains
	\begin{itemize}
	\item became a "junkie" on opium
	\item died in sleep (or murdered?)
	\end{itemize}
\end{itemize}

\subsection{Commodus (180-192)}
\begin{itemize}
\item son of Aurelius
\item lacked ability (worst ruler since Nero)
\item addiction to pleasure 
\item highly unstable personality
\item Praetorian Prefect rules so he can play
\item abandons (a) invasion across Danube (buys off barbarians!), (b) Antonine Wall
	\begin{itemize}
	\item upsets army
	\end{itemize}
\item alimenta suspended
\item appoints his favorites to high court positions
\item reign of terror (many plots and purges) especially against the Senate
\item 192 CE started demanding that he be worshipped as a living god, Hercules Romanus
	\begin{itemize}
	\item started his own priesthood (flamen Herculaneus Commodianus)
	\item has Senate deify him while still alive!
	\end{itemize}
\item assassinated (poisoned by his lover Marcia, then strangled by wrestling partner and gladiator Narcissus) 
	\begin{itemize}
	\item supported by Praetorian Prefect
	\end{itemize}
\item memory damned
\item civil war results
\end{itemize}

\subsubsection{The Gladiator}
\begin{itemize}
\item appeared as a gladiator 735 times (opponents are animals or men only allowed to use blunt wooden swords!)
\item Excellent at throwing javelins at ostriches 
\end{itemize}

\subsection{Short-lived Emperors that follow}
\begin{itemize}
\item Helvius Pertinax (193 AD):
	\begin{itemize}
	\item assassinated after 87 days
	\end{itemize}
\item Didius Julianus (193 AD): (bust at left)
	\begin{itemize}
	\item Praetorian Guard auctions off the throne
	\item does not pay up and is murdered after a few months by the Praetorian Guard
	\end{itemize}
\end{itemize}

\section{The Severan Dynasty (193-235 AD)}
\subsection{Septimius Severus (193-211)}
\begin{itemize}
\item a North  African married into a Syrian royal family
	\begin{itemize}
	\item first Black/Berber Emperor
	\item spoke Latin with a Punic accent!
	\item shows cosmopolitan nature of the Roman Empire
	\end{itemize}
\item declared Caesar by the Danube legions 
	\begin{itemize}
	\item replaces Praetorian Guard with his own troops
	\item extends power of Equestrians (run legions, provinces)
	\item alimenta restored
	\item free medical care
	\item soldiers allowed to marry (children can become heirs and inherit!)
	\item devaluation of coinage
	\item recaptures Parthian provinces
	\end{itemize}
\item Gains some peace and prosperity
\item Made sons Caracalla (198 CE) and Geta (209 CE) co-emperor with him to train them to rule
\item Dies in York, England (good Emperors are always on the move and with the army at troubled frontiers)
\item last words were “Get along with each other, pay the soldiers, and despise all the others”
\end{itemize}

\subsection{Caracalla (211-217)}
\begin{itemize}
\item son of Severus and a bad ruler
\item murders his brother Geta to gain sole power
\item raises army pay
\item increased taxes
\item citizenship to all except slaves (no incentive to join army!)
\item Tried to emulate his hero, Alexander the Great
	\begin{itemize}
	\item saw himself as a second Alexander
	\item took part of Alexander the Great's tomb from
	\end{itemize}
\item Alexandria (last mention of the tomb), and
\item wore his armour
\item Started a war in the East with the Parthians and traced Alexander the Great’s invasion route (according to the historian Cassius Dio)
	\begin{itemize}
	\item even re-equipped a Roman Legion with the long pikes of Alexander’s phalanx (heavy infantry)
	\end{itemize}
\item assassinated during war with Parthians near the site of Crassus’ defeat at Carrhae in 217CE
\end{itemize}

\subsection{Macrinus (217-218)}
\begin{itemize}
\item Praetorian Prefect of Caracalla
	\begin{itemize}
	\item part of the plot to murder Caracalla)
	\end{itemize}
\item 1st Moor (North African) Emperor
\item Bought peace with the Parthians for 200 million sesterces
\item Murdered 
\end{itemize}

\subsection{Elagabalus (218-222)}
\begin{itemize}
\item chosen by Syrian legions (after bribed)	
	\begin{itemize}
	\item claimed to be the bastard son of Caracalla
	\item 15 years old
	\end{itemize}
\item Controlled by the powerful and influential women of the Severan Dynasty
\item “puppet Emperor”
\item religious fanatic of the Eastern (sun) cult of Heliagabalus
\item Brings big black rock of cult of Elagabalus to Rome
\item imposes his religion on Rome and replaces Roman gods with the sun cult
\item travelled with a harem of "300 cute young boys and 300 cute young girls"
\item bloody, cruel, decadent and perverted
	\begin{itemize}
	\item wore pearls and lots of make-up in public
	\end{itemize}
\item Mom, Julia, and grandmother, Julia, ruled for him
\item murdered by Praetorians while hiding in a chest/privy
\end{itemize}

\subsection{Alexander Severus (222-235)}
\begin{itemize}
\item Praetorian Prefect becomes senator (Ulpian)
\item Alexander tries to rule on his own, but is a poor warrior and ruler
	\begin{itemize}
	\item defeats Persians
	\item but bribes Germans to withdraw (insult to army, who wanted the money!)
	\end{itemize}
\item army revolts, lead by Maximus the Thracian
\item “It’s Maximus or me!”
\item Alexander and Julia murdered by army
	\begin{itemize}
	\item Alexander cries to mom that its all her fault!
	\end{itemize}
\end{itemize}

\section{Medicine}
\begin{itemize}
\item real medicine developed in Greek world
	\begin{itemize}
	\item little improvement until the 17th century
	\end{itemize}
\item Greek doctors came to Rome as slaves, so medicine = servile
\item fees paid by patient (no OHIP)
	\begin{itemize}
	\item some Romans believed that you should not charge a fee for saving a life
	\end{itemize}
\item medical schools (Alexandria etc.) optional; apprenticeship usual
	\begin{itemize}
	\item you were a doctor because you said you were
	\end{itemize}
\item no licensing, or malpractice, therefore some physicians deadly
\end{itemize}

\subsection{Origins}
\begin{itemize}
\item Greek:
	\begin{itemize}
	\item Hippocrates of Kos (c. 460-370 BCE)
		\begin{itemize}
		\item basis of Greek and Roman Medicine
		\item apply philosophy to medicine to create a clinical science
		\item diet exercise and rest vs religion and magic
		\end{itemize}
	\item Hippocratic Corpus
		\begin{itemize}
		\item 70 texts attributed to him	
		\end{itemize}
	\item Hippocratic Oath
		\begin{itemize}
		\item do no harm
		\end{itemize}
	\end{itemize}	
\item Roman:
	\begin{itemize}
	\item Celsus (1st c. AD)
		\begin{itemize}
		\item Latin medical text, based on Greek models
		\item reliance on drugs, herbs, home remedies
		\item outline various medical practices and procedures
		\end{itemize}
	\item Galen (2nd c. AD)
		\begin{itemize}
		\item builds on Hippocrates and Galen
		\item much on anatomy, Physiology and pathology
		\item influence extends beyond Roman period
		\end{itemize}
	\end{itemize}
\end{itemize}

\subsection{Causes and Cures}
\begin{itemize}
\item epidemics, e.g. kissing disease (Tiberius), smallpox and bubonic plague (160's-180's)
\item Bubonic plague said to kill 2,000/day in Rome in 189 CE (c. 10\% of Roman Empire in total)
\item ignorance of hygiene
	\begin{itemize}
	\item toilet in kitchen or on shared bench
	\item live in close quarters
	\item ghettos/crowded insulae
	\item no washing of hands
	\item mice and fleas everywhere
	\end{itemize}
\item treatments: diet, rest, blood-letting, enemas, ointments, leeches and maggots
\item pharmacy: ointments in cakes with stamped directions (many herbal remedies)	
	\begin{itemize}
	\item no antibiotics, no 	anesthetic (mostly 	use strong, warm wine)
	\end{itemize}
\item Temple of Aesculapius (island in Tiber): 
	\begin{itemize}
	\item dream-cures
	\end{itemize}
\item Rome builds hospitals
	\begin{itemize}
	\item work on sanitation and fresh water
	\end{itemize}
\item spas (mineral/hot springs)
	\begin{itemize}
	\item many still in use
	\end{itemize}
\item alternative = home remedies
\item home remedies: Cato on cabbage (cure-all) and magic spells (ie chanting for a dislocated shoulder)
\item surgical tools: lancets, scalpels, probes, forceps, clamps, saws etc. (mostly of steel/iron vs stainless steel)
\end{itemize}

\subsubsection{Battlefield Surgery}
\begin{itemize}
\item A mosaic from Pompeii
\item Army surgeons operate on the field
\item Stretcher-bearers paid by the number of wounded they bring to the doctor
\item Roman Military Medics (the Medicus) was highly respected
	\begin{itemize}
	\item learned a variety of techniques to stop bleeding, set bones, close wounds, use of medicinal herbs and amputate limbs
	\item cleaning and closing a wound was crucial to a soldier’s survival and stopping infection
	\end{itemize}
\item Medicus commonly used Spider Web Bandages
	\begin{itemize}
	\item use a combination of honey, vinegar and cobwebs to bandage wounds
	\item modern research tells us the spider silk is, based on weight and tensile strength, stronger than steel and could possess antibacterial properties
	\end{itemize}
\end{itemize}	

\subsection{Dentistry}
\begin{itemize}
\item Carried out by physicians (no “dentists”)
\item Less sugar in diet, so fewer cavities, but worn teeth
\item Believe toothache caused by worm
\item Extractions, wiring, filling, bridgework, dentures
\item Crowns and bridges made from gold
\item False teeth taken from other people, animals, or even made of iron!
\item Dental patients tied down and given wine to numb the pain
\item Toothpowder: not to fight cavities, but for white teeth, clean mouth
\item Brush teeth with a finger or chewed stick
\item Some Roman patricians had special slaves whose role was to clean their teeth
\end{itemize}

\section{Technology}
\begin{itemize}
\item Very few literary sources on Greco-Roman technology have survived
	\begin{itemize}
	\item Vitruvius (On Architecture), Frontinus (Aqueducts), Hero of Alexandria (Pneumatics and Mechanics), Cato's treatises, Pliny (Natural History)
	\end{itemize}
\item Science = “knowledge”
\item scientific discoveries were made mostly by Greeks, not Romans, and borrowed from other cultures
	\begin{itemize}
	\item ie adapt drainage systems and aqueducts from the Etruscans
	\item ie adapt naval fleet technology from non-Roman Italian allies
	\item ie adapt some aspects of building construction, and surveying from the Greeks
	\end{itemize}
\item Roman technology not fully appreciated, developed or used to its greatest potential
\item Greatest successes are in civil engineering
\item Romans a labour intensive people
	\begin{itemize}
	\item use muscle (mostly slaves), water and wind power
	\item not push to develop new technology
	\end{itemize}
\item hydraulic engineering
\item water moved by siphon, pump, aqueduct (based on Etruscan designs)
\item draining of Roman Forum
\item Agrippa's and Claudius' aqueducts
	\begin{itemize}
	\item provide 445 L water/person/day to Rome
	\end{itemize}
\item Cloaca Maxima (main sewer of Rome)
	\begin{itemize}
	\item built 200-33 BCE
	\item sewage waste into the Tiber River!
	\item sewer workers paid 25 denarii/day + meals vs a teacher’s 50 denarii/student/month
	\end{itemize}
\item metallurgy: coins, statues, tools etc. (stamped, cast or forged)
\item mines use horizontal tunnels; ore flushed and filtered by water
\item smelting furnaces and pottery kilns
\item machines:
	\begin{itemize}
	\item lever, pulley, siphon
	\item water-lifting screws, treadmills, cranes, paddlewheels
	\end{itemize}
\item catapults: powered by sinew wound onto a windlass
\item hips: powered by sail (useless if wind wrong) and oars
\item animal power, e.g. to move heavy loads
\item milling:
	\begin{itemize}
	\item push mill
	\item donkey mill
	\item hand mill
	\end{itemize}
\item Romans did not have windmill, rubber, crank, big factories
\item labour saving devices unsuccessful, e.g. Gallic reaper
\end{itemize}

\subsection{Vitruvius - De Architectura “On Architecture”}
\begin{itemize}
\item Wrote during the 1st century BCE (20’s BCE) during the construction boom of Augustus
\item wrote10 volumes on engineering and architecture
\item Only surviving Roman text on the subject
\item Believed that engineers should be well-educated and well-rounded in the following:
	\begin{itemize}
	\item liberal arts
	\item surveying
	\item drafting
	\item history
	\item music
	\item some knowledge of law, writing, medicine and astronomy
	\end{itemize}
\end{itemize}

\subsection{The Haterii Family}
\begin{itemize}
\item Family of successful building contractors during the reign of the Flavians (late 1st century CE)
\item Family tomb shows an A-frame crane, using tread wheels and block and tackle
\item Crane could swivel vertically and horizontally
\item Lifting mechanism powered by men walking on a large treadmill that turned a drum at the base of the crane
\item Drum wrapped or unwrapped the rope that pulled the weight/object
\end{itemize}

\subsection{Water Technology: Hydraulic Engineering}
\begin{itemize}
\item watermills from 1st century CE onwards
\item water-powered saws to cut marble and other stone for building
\item Hydraulic mining used to move earth, sift deposits, then break up ore with hydraulic hammers
	\begin{itemize}
	\item mine stone, marble, gold, silver, copper and metals
	\end{itemize}
\item Mining of this scale not seen again in Europe until the 19th century
\end{itemize}

\subsection{Roman Concrete}
\begin{itemize}
\item First developed in late 3rd century BCE
	\begin{itemize}
	\item walls built with a mixture of mortar and small stones, called opus caementicium
	\item Vitruvius describes mortar as a mix of pozzolana (a volcanic ash from Puteoli), lime, water, and small (aggregate) stones
	\item found that it could be shaped into any form and hardened like stone
	\item no longer needed stones to fit perfectly together
	\item allowed the construction of larger, broader buildings and domes
	\item ability to set and harden under water allows the building of bridges and harbors quickly and easily
	\end{itemize}
\end{itemize}

\section{Occupations}
\begin{itemize}
\item urban plebs: idle rabble, or work-force?
	\begin{itemize}
	\item most hired as day labourers (menial work)
	\item Forum as labour pool; pay 1 denarius per day
	\item keep poor busy and happy
	\end{itemize}
\item stigma against undignified jobs (manual work for wages; factories; tax collection; retailing; food services, perfume, entertaining, fish sellers, butchers, cooks, poultry raisers, fishermen, salesmen, peddlers and porters)
\item regular wages are suitable only for slaves
\item respectable workers are paid for the item produced or the service performed
\item good jobs: art, medicine (only if you are really good at it!), architecture, teaching
\item FARMING is the most noble occupation
\item other factors in finding a job: training, money, talent, inclination
	\begin{itemize}
	\item architecture and law are expensive fields to study
	\end{itemize}
\item taberna (shop): often a family operation
	\begin{itemize}
	\item usually located at front of house
	\item family trade
	\item cottage industries
	\end{itemize}
\item small factories:
	\begin{itemize}
	\item collegia (workers' associations, social clubs)
	\item apprenticeships in some trades (ie weaving and sculpting)
	\item collegia sometimes involved in politics, e.g. firemen
	\item roots of the Medieval Guild system
	\end{itemize}
\item tombstones name professions or show them in relief
\item 200+ different jobs attested, largely at Rome (some jobs regional)
	\begin{itemize}
	\item ie local wines, pottery styles, fishermen and shepherds
	\end{itemize}
\end{itemize}

\subsection{Woman's occupations}
\begin{itemize}
\item Patrician women don’t work, but plebeians do:
	\begin{itemize}
	\item work mostly in service trades (catering, nursing, prostitution, shepherd comfort girl!)
	\item tabernae (barmaid, cook “hostess”)
	\item "feminine" crafts (weaving, laundry, crafts)
	\item often learned their trade at a young age
	\item Epitaphs for Viccentia, a 9 year old gold worker, and Pieris, a 9 year old hair dresser
	\end{itemize}
\end{itemize}

\section{Communications}
\subsection{Roman Roads}
\begin{itemize}
\item Rome builds 120,000 km of roads
	\begin{itemize}
	\item communication, trade and military uses
	\end{itemize}
\item Appian Way (road from Rome to Capua)
	\begin{itemize}
	\item first major military highway built in 312 BC
	\item 132 miles long
	\item designed for speed
	\item 2.4-7.5 metres wide
	\end{itemize}
\item Most roads built by Marius’ Mules
\item road construction techniques
\item 1 metre trench
\item 4.5 m wide, with a 120 cm foundation
\item larger stones under smaller stones and cement
\item Topped with gravel, flint and slabs
\item 1 Roman mile = 1,000 paces
\item “Mile” comes from the Latin “milia passuum” ("one thousand of paces“), which was approximately 1620 yards, 1480 meters
\item each mile marked by a 2-metre tall pillar, called a Milestone (miliarium)
\item A circular column on a solid rectangular base is a milestone or miliarium.
	\begin{itemize}
	\item set two feet into the ground
	\item 2 metres tall
	\item 20" in diameter
	\item weigh about 2 tons
	\end{itemize}
\item Base was inscribed the number of the mile relative to the road it was on
\item A panel at eye-height indicated the distance to the Roman forum
	\begin{itemize}
	\item plus officials who made or repaired the road and when
	\end{itemize}
\item Rome builds many bridges as part of the road system
	\begin{itemize}
	\item originally wooden
	\item later permanent stone bridges replace wooden bridges
	\end{itemize}
\item Roman army engineers learn to build temporary collapsible bridges and pontoon bridges
	\begin{itemize}
	\item  Trajan’s column depicts pontoon bridge across the Danube built in 104CE
	\item  Civilian and military uses
	\end{itemize}
\item curatores viarum (local officials in charge of roads)
	\begin{itemize}
	\item look after their own section of road
	\end{itemize}
\item cursus publicus (Imperial postal system)
\item mansiones (inns run by the state)
\item diploma (permit to use Imperial Post)
\item private inns and their facilities
\item cisium (two-wheeled cart)
\item A variety of carts, of all sizes, used to move goods, pulled by oxen, mules, horses and camels
\item Ostia (seaport of Rome, at mouth of Tiber)
\item Rhine, Rhône (main rivers of Gaul)
\item imports and exports
\item terra sigillata (red-gloss pottery)
\item 2.5\% tax on goods crossing provincial borders
	\begin{itemize}
	\item all get a cut of profits
	\item protect local industries
	\end{itemize}
\item precious metals leave the empire to pay for oriental goods (trade ties as far as India and China)
	\begin{itemize}
	\item much money goes out, but less comes in as Roman expansion stops
	\item creates inflation and stalled economy
	\end{itemize}
\end{itemize}	

\subsection{Transportation and communication by boat}
\begin{itemize}
\item water transport:
	\begin{itemize}
	\item much cheaper and faster than road
	\item safest to sail during the summer (April to October), vs stormy months of winter
	\item sail by sun at day and by stars at night
	\item most only sail by day
	\end{itemize}
\item Alexandria (chief port of Egypt)
	\begin{itemize}
	\item grain freighters carry 200-3,000 tons each
	\item commercial fleet (naves onerariae)
	\item annona (grain supply)
	\item amphoras (clay shipping containers)
	\end{itemize}
\item shipping hazards
\item Storm
\item Pirates
\item Shipwreck
\item Insurance fraud
\item Lighthouses
	\begin{itemize}
	\item  usually travel by day
	\item used to avoid rocks at night
	\item strategically placed along Mediterranean coastline
	\end{itemize}
\end{itemize}

\subsubsection{How Large was Rome’s Merchant Marine?}
\begin{itemize}
\item If we just consider the number of ships needed to feed Rome:
	\begin{itemize}
	\item assume the average inhabitant of Rome ate 237kg of wheat per year
	\item Rome’s population of 1 million requires 237,000 metric tons of wheat per year
	\item average Roman cargo ship holds 250 tons of wheat
	\item this requires 948 shiploads of wheat per year
	\item considering the high rate of spoilage at sea and lost ships, Rome alone would require perhaps 1,300 ships of grain/year just to provide the basic needs of whea
	\end{itemize}
\end{itemize}

\subsubsection{Travel Times}
\begin{itemize}
\item Rome to Cologne (land) - 67 days
\item Rome to Carthage (sea) - Minimum 2 days, Normal 10
\item Rome to Alexandria (sea) - Minimum 9 days, Normal 41
\item Rome to Antioch - 124 days by land + 2 days by sea
\end{itemize}

================================================================== \\
END OF MIDTERM TWO MATERIAL \\
================================================================== \\

\section{Imperial Literature}
\subsection{Augustan Writers (The golden age)}
\begin{itemize}
\item covers the reigns of Augustus, Tiberius and Caligula
\item A great deal of patronage
	\begin{itemize}
	\item wish to show cultural sophistication
	\item Maecenas was the greatest patron of the time
	\item brought scholars and artists to court
	\item friend of Augustus
	\item introduced him to Vergil
	\item supported poets who wrote on patriotic themes
	\end{itemize}
\end{itemize}

\subsubsection{LIVY (64? BCE – 17 CE)} 
\begin{itemize}
\item From northern Italy (1st c. BC)
	\begin{itemize}
	\item greatest prose writer in Rome
	\item shows great eloquence and great speeches
	\item very patriotic
	\item plays up Rome’s enemies and plays down Roman vices
	\end{itemize}
\item Uses several sources, some questionable
\item History of Rome (year by year) from early Republic to Augustus in 142 Books
	\begin{itemize}
	\item arranged by Consular year, event, theme or idea (creative format
	\item left ideas/facts out if it destroyed or confused his themes
	\item Biblical Historical Truth and Moral Truth
	\end{itemize}
\end{itemize}

\subsubsection{Virgil (70 – 19 BCE)}
\begin{itemize}
\item poet from northern Italy
\item recognized as second to Homer by his peers
\item patronized by Maecenas (a friend of Augustus)
\item Ecologues (pastoral poetry) ideal country people vs ugly city dwellers
\item Georgics (didactic farming poetry) contrast gaudy life of city dwellers vs the simple life
	\begin{itemize}
	\item celebrate the “good old days” when Rome was small, close-knit farming community (vs a huge, impersonal Empire)
	\end{itemize}
\item Aeneid (Fall of Troy to the rise of Augustus)
	\begin{itemize}
	\item a national epic/propaganda value
	\item story of Aeneas and son Iulus
	\item love affair with Dido
	\item Rome’s great ancestors from Venus to Augustus
	\item celebrates Rome’s virtues 
	\item beautiful poetry
	\item Books 1-12 // Odyssey of Homer
	\item Books 13-24 // Iliad of Homer
	\item theme = Rome’s fate/destiny (to rule the world, spare the weak, and defeat the proud)
	\item Virgil wanted it burned at his death
	\end{itemize}
\end{itemize}

\subsubsection{Horace(65 – 8 BCE)}
\begin{itemize}
\item freedman's son
\item introduced to Maecenas by Vergil
\item lyric poetry 
\item Odes (light, lyrical poetry on life, love, money, virtue, wine and beauty)
	\begin{itemize}
	\item includes many carpe diem themes
	\end{itemize}
\item Epodes (bitter, pessimistic poems)
\item Satires (makes fun of life in Rome)
	\begin{itemize}
	\item clever turns of phrase
	\item only original literary form
	\end{itemize}
\item Epistles (sermons on morals, religion and philosophy)
\item Art of Poetry (principles for writing poetry and tragedy)
	\begin{itemize}
	\item basis for Alexander Pope’s Essay on Literary Criticism in the 18th century
	\end{itemize}
\item hymn for secular games
\end{itemize}

\subsubsection{Propertius(50 – 2 BCE)}
\begin{itemize}
\item writer of elegy (his joys and pains)
\item affair with Cynthia
	\begin{itemize}
	\item a great beauty
	\item her rages, suspicions and infidelities drive him away
	\end{itemize}
\item very scholarly, but his habit of going off into tangents of obscure Greek myths (distracts from his poetry)
	\begin{itemize}
	\item poetry as a game for intellectuals
	\end{itemize}
\end{itemize}

\subsubsection{Ovid(43 BCE – 18 CE)}
\begin{itemize}
\item popular poet (most sensual and sophisticated of elegists)
\item Art of Love (pornographic handbook which explains all the known aspects of the “heterosexual experience”, from rape to incest),	published 2BCE
\item Metamorphoses (250 stories of Greek myths and creation myths, some of which were pornographic)
\item Fasti (chief religious festivals of Rome)
\item ran afoul of Augustus (involved with Julia?) and condemned for “teaching adultery”
	\begin{itemize}
	\item exiled to Black Sea(8 - 18 CE)
	\end{itemize}
\end{itemize}

\subsection{The silver age of imperial literature (1st c. CE)}
\begin{itemize}
\item covers the reigns of Claudius and Nero
	\begin{itemize}
	\item dangerous to be considered a contemporary or competitor to Nero in literature/arts
	\end{itemize}
\end{itemize}

\subsubsection{Petronius (27 – 66 CE)}
\begin{itemize}
\item “arbitrator of social graces” at Nero’s court
\item Satyricon (parody the morals of the time)
	\begin{itemize}
	\item adventures of three young (depraved) freedmen as they tour the taverns and brothels of southern Italian port towns
	\item Encolpius seeks aid of Priapus
	\item Trimalchio's dinner(T's home based on Golden House of Nero)..
	\end{itemize}
\end{itemize}

\subsubsection{Seneca (4 BCE – 65 CE)}
\begin{itemize}
\item Nero's mentor
\item millionaire in banking and politics
\item Stoic philosopher
\item letters, essays, Natural Questions, satires, and Stoic plays
\item pithy sentiments and clever turns of phrase
\item Moral guidance for life
\end{itemize}

\subsubsection{Lucan(39 – 65 CE)}
\begin{itemize}
\item Seneca’s nephew
\item epic poet
\item wrote on the Civil War (the Pharsallia)
	\begin{itemize}
	\item violent and pessimistic epic poem of Caesar and Pompey
	\item very pro-Republic and hostile to Caesar
	\item an “anti-Aeneid”
	\end{itemize}
\item put to death for opposing Nero
	\begin{itemize}
	\item following the Conspiracy of Piso in 65 AD, Petronius, Seneca and Lucan were forced to commit suicide(slit wrists) by Nero in 66 AD
	\end{itemize}
\end{itemize}

\subsubsection{Martial(40 AD – 102 AD)}
\begin{itemize}
\item Spanish poet and satirist
\item Spectacles (attacked the shams and vices of people from all walks of life)
\item Epigrams (sharp and, often, indecent short poems) 
\item his sharp and biting wit made him a popular source of entertainment at dinner parties
\item good commentary on daily life
\item Pliny stated that his poems “reflected life like a mirror”
\end{itemize}

\subsection{2nd century CE Silver Age Authors}
\subsubsection{Pliny the younger (63 – 113 CE)}
\begin{itemize}
\item Senator and Governor of Bithynia
\item great letter writer
	\begin{itemize}
	\item wrote letters to be published(short and polished style, covering one topic)
	\end{itemize}
\item Panegyric (praise of Trajan)
	\begin{itemize}
	\item uses all of the rhetorical tricks of the trade to contrast Domitian with Trajan
	\end{itemize}
\item Letters (correspondence as Governor of Bithynia)
	\begin{itemize}
	\item letters to Trajan concerning problems/concerns
	\item shows his nobility and sharp eye for detail
	\end{itemize}
\end{itemize}

\subsubsection{Tacitus (56 – 117 CE)}
\begin{itemize}
\item Roman Senator and outstanding prose historian
\item Lived through the reigns of Nero, Galba, Otho, Vitellius, Vespasian, Titus, Domitian, Nerva and Trajan
\item Writings cover the period from 14 AD – 96 AD
\item Very perceptive, based on his own practical experience
	\begin{itemize}
	\item wished to  show the dignity and moral events of history
	\item he did not chronicle petty events, rumours or gossip
	\end{itemize}
\item lived through Domitian’s purges, so his works are more pessimistic as compared to Livy’s optimism
\item Histories (Flavian period)
\item Agricola (biography of a famous general)
	\begin{itemize}
	\item married his daughter in 77 AD
	\item wrote of Agricola’s campaigns in Britain
	\end{itemize}
\item Annals (Augustus - Nero); perceptive historian
\item Germania (contrasts nobility of barbarians to the corruption of Rome)
\item also sees the honesty and nobility of the common man vs. the corruption and cowardice of Rome’s leaders
	\begin{itemize}
	\item take moral lessons from our enemies
	\item very biased against the Dynastic system
	\end{itemize}
\end{itemize}
	
\subsubsection{Suetonius(71 - 135 CE)}
\begin{itemize}
\item professional scholar and civil servant (secretary) under Hadrian
\item access to Imperial records
\item Dismissed by Hadrian for making improper comments about the Empress
\item Biographies of the 12 Caesars (from Julius Caesar to Domitian)
	\begin{itemize}
	\item smut (simple, gossipy and to the point, with lots of room for rumours and little time for analysis of inconsistencies) 
	\end{itemize}
\item however, he is the first “Historian” to quote sources, phrases and passages directly in both Latin and Greek
\end{itemize}

\subsubsection{Decimus Iunius Iuvenalis “Juvenal”(55 – 127 CE)}
\begin{itemize}
\item last great Roman satirist
	\begin{itemize}
	\item great master of vocabulary, hexameter and clever phrases
	\item moralist, but often lost in his bitterness
	\end{itemize}
\item Satires (attacks on nearly everything!)
	\begin{itemize}
	\item cannot help to write satires of corruption in Rome
	\end{itemize}
\item Many of his sayings have become part of modern speech, such as “bread and circuses” and “who will guard the guards themselves?”
\item only make fun of the dead, since too dangerous to make fun of the living
	\begin{itemize}
	\item forgot his own rules and later exiled by Domitian for satirizing a court favourite
	\end{itemize}
\end{itemize}

\subsubsection{Common Themes of Silver Age Authors}
\begin{itemize}
\item Complain of  a lack of leadership, individuality, morals and manners
\item Celebrate the simpler “good old days” of the early Republic
\item Compare the good rule of the Senate/Republic with the bad rule of some of the Emperors
\item Nowadays fewer noble and virtuous men, and more political opportunists
\item It was the virtues of our ancestors that made us great
\item Our noble past has many lessons to teach us
\end{itemize}
\subsubsection{Philogelos (“The Laughter-Lover”)}
\begin{itemize}
\item The oldest existing collection of jokes
	\begin{itemize}
	\item written in Greek
	\item attributed to Hierokles and Philagrios
	\item written in 4th century CE
	\item contains 265 jokes categorized into subjects such as Drunkards, Intellectuals, Gluttons and Fools
	\end{itemize}
\end{itemize}

\section{The Roman Calendar}
\subsection{Roman Time Keeping}
\begin{itemize}
\item Day: sunrise to sunset, divided into 12 hours
	\begin{itemize}
	\item midday in summer is the 7th hour
	\item the 8th hour is 1:30 pm
	\end{itemize}
\item Night: sunset to sunrise, divided into 4 watches
\item hours and watches varied in length at different times of year
\item timekeeping: originally by observing sun and moon
	\begin{itemize}
	\item sundial (3rd c. BC)
	\item water clock used in law courts (limit time to state your case)
	\end{itemize}
\end{itemize}

\subsection{The Roman Calendar}
\begin{itemize}
\item Republican calendar: lunar month (29 days); no weeks
	\begin{itemize}
	\item 1st day = Kalends, 5th = Nones, 13th = Ides . -Exceptions:
	\end{itemize}
\item "In March, July, October, May The Ides are on the 15th day,	The Nones the 7th; but all besides Have 2 days less for Nones and Ides“
\item lunar year 11 days too short, so extra months inserted by priests (based on agrarian calendar)
\end{itemize}

\subsection{The Roman Julian Calendar}
\begin{itemize}
\item Caesar's calendar: year of 365 1 /4 days devised by the Egyptian astronomer Sosigenes (still used today)
	\begin{itemize}
	\item only slightly modified by Pope Gregory XIII in 1582
	\end{itemize}
\item calendar once began in March (December = 10th month)
	\begin{itemize}
	\item 153 BC year begins in January to allow consuls to get to provinces
	\item July=Julius Caesar (was once Quintilis, the “Fifth”)
	\item August+Augustus (was once Sextilis, the “Sixth”)
	\item other months kept old names
	\end{itemize}
\end{itemize}
\begin{tabular}{ | l | l | l | p{4cm} | }
  \hline                       
  \textbf{Name of Month} & \textbf{Named After?} & \textbf{What Does it Mean?} & \textbf{Why This Name?} \\
  \hline
  January & Janus & God of Doors & This month starts the year \\
  \hline
  February & Februo & Purity & Roman month of sacrifices and purification \\
  \hline
  March & Mars & God of War & Star of year for soldiers (no fighting during the winter) \\
  \hline
  April & Aperire & Open & The is the month where trees open their leaves \\
  \hline
  May & Maia & Goddess of Growth & Month when plants start to grow/sprout \\
  \hline
  June & Juno & 	& Queen of the Gods \\
  \hline
  July & Julius Caesar & Ruler of Rome & He reorganized the Roman/Julian Calendar \\
  \hline
  August & Augustus Caesar & 1$^{st}$ emperor of Rome & Thought to be as important as Julius Caesar \\
  \hline
  September & septem & Seven & 7$^{th}$ month (counting from March) \\
  \hline
  October & octo & Eight & 8$^{th}$ month (counting from March) \\
  \hline
  November & novem & Nine & 9$^{th}$ month (counting from March) \\
  \hline
  December & decem & Ten & 10$^{th}$ month (counting from March) \\
  \hline
\end{tabular}
\begin{itemize}
\item unlucky days, e.g. anniversaries of disasters; no business
\item market day every 8 days (different day in each town)
\item 2nd c. AD: introduction of week days named after planets (Saturn, Sun, Moon Mars, Mercury, Jupiter and Venus)
\item years named after consuls, or numbered from foundation of Rome in 753 BC (AUC = ab urbe condita)
\end{itemize}

\section{The Imperial Cult}
\begin{itemize}
\item A new element of the Roman State Religion (in addition to traditional deities)
\item Imperial cult worshipped Emperors and members of their families as gods
\item Emperor seen as successor to Alexander and other god-kings
	\begin{itemize}
	\item in west, only deceased and deified rulers worshipped
	\item more likely to be worshipped seriously in the east where a strong god-king tradition had existed (Persian and Egyptian tradion)
	\end{itemize}
\item Augustus organizes Imperial Cult for worship of Julius Caesar (temples, festivals, priesthoods)
	\begin{itemize}
	\item On Caesar’s death, he was officially recognized as a god
	\item the Divine (“Divus”) Julius Caesar
	\item 29 BCE Augustus permits the Greek cities of Asia Minor to set up temples to the Divine Julius
	\item begins the tradition of Emperor-worship throughout the Empire
	\item a focal point to unify the peoples of a vast and diverse Empire
	\end{itemize}
\item Living emperor is not worshipped, only his genius (exceptions: Caligula, Domitian, Commodus)
\item apotheosis (transformation into gods) on their death
\item Freedmen and Provincials active in the cults
	\begin{itemize}
	\item unifying influence on the Empire
	\item public display of piety, devotion and loyalty to Rome/Empire/State
	\end{itemize}
\item Tradition to deify deceased emperors, but not always carried out
	\begin{itemize}
	\item Claudius and Hadrian deified only because successor insisted
	\item Apocolocyntosis ("pumpkinification" of Claudius): by Seneca?
	\item "bad" emperors not deified after death (Tiberius, Nero, etc.)
	\item others damned (ie Domitian and Commodus) by Senatorial decree and their memory erased (Damnatio memoriae) 
	\end{itemize}
\item Empresses can also be deified (e.g. Livia/Augustus, Faustina/Antoninus Pius)
\end{itemize}

\section{Mystery Cult Religions}
\begin{itemize}
\item Traditional State religions and Imperial cult too impersonal to satisfy needs of the individual
\item Common people turn to exciting Eastern religions
	\begin{itemize}
	\item upper classes turn to philosophy
	\end{itemize}
\item Mystery religions offered purification, monotheism,  communion, life after death
	\begin{itemize}
	\item secret rites and levels of initiation
    -People want to be a part of something, this is a good way.	
    \end{itemize}
\item Build a relationships with a single god.
\item Cults come from eastern Mediterranean (exotic); often involve ecstasy (dancing, intoxication, worse?)
\end{itemize}

\subsection{Greek/Hellenistic Cults}
\subsubsection{Cybele}
\begin{itemize}
\item Cybele (Asian mother goddess): brings boyfriend Attis back to life
	\begin{itemize}
	\item fertility goddess, protectress (wears walled city as crown), wild animals (lion attendants), cures and oracles
	\item wild ecstatic state (feel no pain)
	\item cult reaches Rome in 2nd Punic war
	\item later patronized by Claudius (priesthoods opened)
	\end{itemize}
\item Priests (orig. eastern) could be Roman but must castrate selves, run through Rome and toss genitals at a house (counter to value system – no heirs)
\item festival: fasting, purification, taurobolium (bath in bull's blood) and carry around bull’s genitals
\item popular with women (female dominated)
\end{itemize}

\subsection{The Cult of Isis (turrists)}
\begin{itemize}
\item Isis (Egyptian saviour goddess)
	\begin{itemize}
	\item puts husband Osiris back together
	\end{itemize}
\item Protecting and motherly.  Trend among female goddesses.
\item priests = Egyptian
	\begin{itemize}
	\item use Nile water, street parade, drama, penitents, festivals, banquets, interpretation of dreams
	\end{itemize}
\item ideal mother (nurses son Horus)
	\begin{itemize}
	\item cult popular among women
	\end{itemize}
\item Apuleius’ Golden Ass (2nd c.) describes initiation (ecstacy and flagellation)
\item Serapis another similar god (sky/healing god)
	\begin{itemize}
	\item state-of-the-art temple at Alexandria
	\end{itemize}
\end{itemize}

\subsection{The Cult of Mithras}
\begin{itemize}
\item Mithras (Asian/Persian god of light, truth, and good god)
	\begin{itemize}
	\item battles forces of darkness
	\item known as “Lord of Light”, God of Truth”, “Saviour from Death”, “Giver of Bliss”, “Warrior” and “Victorious”
	\end{itemize}
\item This is a god who does all the good stuff.
\item comes to Rome in the later half of the 1st c. AD
\item followers must be tough, disciplined; popular with army, merchants and all social classes
\item bull-slaying scene, reproduced in underground shrines: meaning? (we don't know much about this cult)
	\begin{itemize}
	\item link to astronomy
	\end{itemize}
\item temples built in caves or built to look like caves (bull slaying happened in a cave)
\item belief in prosperity and an afterlife
\item Wore a pointed cap (came from the middle east)
\end{itemize}

\subsection{Sol Invictus Cult} 
\begin{itemize}
\item Sol Invictus(“Sun Unconquered”)
	\begin{itemize}
	\item Sun God cult
	\end{itemize}
\item conical black stone is Syrian cult image
\item Romans didn't like this cult very much due to the strangeness, dancing, noise, screaming.
\item Very secretive, so heavily speculated about by the Romans and a common topic of conversation.
\item weird rites (perversions?)
	\begin{itemize}
	\item drums, cymbols and anthems sung by women
	\item rites include baptism and ceremonial meal
	\item rites often linked to Mithraism (Sol and Mithras... pretty damn similar)
	\item Same hat, killing bulls, astronomy.
	\end{itemize}
\end{itemize}

\section{Christianity}
\begin{itemize}
\item Christianity (Jewish Palestinian carpenter's son, claimed to be son of Yahweh)
\item 30 AD crucified by Pontius Pilate (prefect)
	\begin{itemize}
	\item came back to life after death (reign of Tiberius)
	\end{itemize}
\item promote love, forgiveness of sins, equality, and everlasting life, purification, community, communion 
	\begin{itemize}
	\item cult was open to everyone (and so unpopular with mainstream Judaism of the time, you were either Jew or Gentile, you don't change...)
	\item attracted the poor, slaves, cripples, women, and society’s outcasts at first
	\item spread by travels of early disciples
	\end{itemize}
\end{itemize}

\subsection{Persecutions}
\begin{itemize}
\item Christianity denied emperor's divinity, thus was treasonous (Caesar burning in hell right now – btw, don't say this in 50AD Rome)
\item Rites misunderstood
	\begin{itemize}
	\item secret meetings in catacombs, bird and fish secret symbols, murder (eat his flesh and drink his blood), cannibalism, incest (“brothers and sisters in Christ”)
	\end{itemize}
\item Everything taken in the literal by mainstream society.
\item various persecutions beginning in 64 AD under Nero
\item Christians blamed for plagues, famines, infertility, crop failures, defeat in war, etc
\item Public persecutions and executions (burning, crucifixion, gored to death in the arena by bulls, etc) used rally support for the State Religion and deter converts
\end{itemize}

\subsection{Constantine's Role in Christianity}
\begin{itemize}
\item 312 AD wins Battle of Miilvian Bridge with divine aid.
\item Constantine first to use “Chi-Rho” symbols.
\item Christianity legalized by emperor Constantine (313) in the “Edict of Milan”. (slowly makes the world more Christian, notices that if he wants to switch the religion, needs to do so without ticking off the other two days)
\item 321 AD Sunday a legal holiday (brilliant move! - Makes three religions happy with him (sun-day))
\item 325 AD Council of Nicaea
\item Laws passed to support  Christian ideals (ie banned tattooing of a slave’s face since it was made in the image of God)
\item Bishops, priests (and later Jewish rabbis) exempted from magistracies and other expensive community services
\end{itemize}

\subsection{The triumph of Christianity}
\begin{itemize}
\item 380 AD Emperor Theodosias the Great makes it the official State religion and bans paganism
	\begin{itemize}
	\item Christians then begin to persecute pagans!
	\item take what is familiar (and pagan) and make it Christian to explain their theology
	\end{itemize}
\item Christianity combined the strengths of
	\begin{itemize}
	\item Greek Philosophy (use what is familiar to explain key parts of Christianity)
	\item Roman Administration (Swap pagan gods with saints, rename festivals)
	\item and the Jewish faith (religious lifestyle)
	\end{itemize}
\end{itemize}

\section{Imperial Art and Architecture}
\begin{itemize}
\item Combine Etruscan, Greek and Oriental ideas (eclectic in all things – put things together and call it their own)
\item Reflects the values and ideals of a culture
\item Also a way to show off wealth and power
\end{itemize}

\subsection{Sculpture}
\begin{itemize}
\item intimately linked with Roman funerary practice
	\begin{itemize}
	\item busts often displayed in homes or at funerals
	\item portraiture both idealistic and realistic (should show them as they are)
	\end{itemize}
\item Glorification of the Emperor begins with Augustus
\item Begins to decay in late 2nd century
	\begin{itemize}
	\item anatomy not as well done
	\item expressions more serious and troubled (art imitating life?)
	\end{itemize}
\item can be huge (head of statue of Constantine is 2.5 m tall alone!)
	\begin{itemize}
	\item reflect Eastern god-like awe
	\end{itemize}
\item Romans could not work the stone as well as the Greeks
	\begin{itemize}
	\item Roman copies of Greek originals are supported by a tree stump, flowing robes or other small attachment at the base of the legs
	\end{itemize}
\item Not nearly as good at this as the Greek
\item Cameos common and popular
	\begin{itemize}
	\item Beautiful art form held onto in the later years.
	\end{itemize}
\end{itemize}

\subsection{Painting}
\subsubsection{Mural Painting}
\begin{itemize}
\item Found on walls, canvas and ceilings
\item Roman artists were renowned for their renditions of social or leisure events, mythological themes, and nature scenes and landscapes
\item “Pompeiian” style most popular
\item House of the Mysteries in Pompeii
	\begin{itemize}
	\item distance, space, light, expression and perspective are shown in great detail
	\item painting material was very colourful and realistic
	\end{itemize}
\item Often paint outdoor scenes as if looking through a window
\item Most interior Roman painting has maintained its colours
\end{itemize}

\subsubsection{Portrait Painting}
\begin{itemize}
\item Found on walls, canvas and ceilings
\item Roman artists were renowned for their renditions of social or leisure events, mythological themes, and nature scenes and landscapes
\item “Pompeiian” style most popular
\item House of the Mysteries in Pompeii
	\begin{itemize}
	\item distance, space, light, expression and perspective are shown in great detail
	\item painting material was very colourful and realistic
	\end{itemize}
\item Often paint outdoor scenes as if looking through a window
\item Most interior Roman painting has maintained its colours
\end{itemize}

\subsubsection{Mosaics}
\begin{itemize}
\item Located mostly on floors (massive)
\item Labour intensive and expensive
\item Geometric or abstract patterns with a two-dimensional design
	\begin{itemize}
	\item commonly used materials were marble, glass paste and naturals shells
	\item pebbles used were mostly black and white
	\item black silhouettes with white outlining was popular
	\item usually contained figures of humans, animals,  and mythological figures, all contained within a floral-type border
	\end{itemize}
\item Were placed in houses like carpets, for example, in the middle of rooms or near doors
	\begin{itemize}
	\item Caveat Canum (“Beware of Dog) in front of door in Pompeii
	\end{itemize}
\end{itemize}

\subsection{Places}
\subsubsection{The Roman Forum}
\begin{itemize}
\item Centre of city life (Greek agora)
\item Shops, statues, meeting places, public speaking podiums
\item Various additions built over the centuries
	\begin{itemize}
	\item Forum of Caesar (46 BC)
	\item Forum of Augustus (2 AD)
	\end{itemize}
\end{itemize}

\subsubsection{Basilica}
\begin{itemize}
\item Rectangular, covered building
\item Courts, businesses and social gatherings
\item Built to be seen from the inside (opposite to Greek designs)
\item Early Christian churches were converted basilicas, or based on their design
\item “Cross”-shaped Christian church or cathedral
	\begin{itemize}
	\item Evolved out of the Roman basilica shape
	\item Transept added to make cross-shape
	\item Main tower usually at intersection of transept and nave
	\end{itemize}
\end{itemize}

\subsubsection{Temples}
\begin{itemize}
\item Follow Etruscan, and later Greek, lines
	\begin{itemize}
	\item built on a huge podium
	\item cella, columned porch, and triangular pediment
	\end{itemize}
\item Quite simple design
\item Romans experimented freely with it
\item Temple of Castor and Pollux in Rome on a huge, concrete, landscaped podium 
\end{itemize}

\subsubsection{Baths}
\begin{itemize}
\item Combined bath, library, gymnasium and community centre
	\begin{itemize}
	\item enclosed gardens
	\item about a penny to enter them
	\end{itemize}
\item Culture status to have a bath, both for home and town.
\item Rooms heated through the flow of warm air through the flues in the wall
	\begin{itemize}
	\item frigidarium (cold rooms)
	\item tepidarium (warm heat)
	\item laconia (sweat baths)
	\end{itemize}
\item Pompeii’s bath (75 BC) an early example
\item Baths of Caracalla (217 AD) in Rome had libraries, lecture halls, gymnasiums, pools, lounges and vast vaulted public spaces decorated in statues, mosaics, stuccos and paintings  
	\begin{itemize} 
	\item held 1,600 bathers in marble-lined pools (largest Roman bath)
	\end{itemize}
\item By the middle of the 4th century AD there were 952 operating bath facilities in Rome
\end{itemize}

\subsubsection{Circuses}
\begin{itemize}
\item Huge race tracks
	\begin{itemize}
	\item elongated rectangles, curved at one end
	\item spina (spine) runs down the middle of the track (horses run around)
	\end{itemize}
\item Rome’s Circus Maximus is 2,000 feet long and hold ½ million spectators (Rome is only a million people).
\end{itemize}

\subsubsection{The Roman Amphitheatre}
\begin{itemize}
\item Semicircular
\item Much use of arch and concrete
\item Colosseum, in Rome, begun by the Emperor Vespasian and opened by his son, the Emperor Titus, in 80 AD
	\begin{itemize}
	\item tiers of seats surround arena
	\item arena (“sand”) in the center, measuring 500 x 620 feet
	\item rooms, passageways, and elevators beneath arena floor
	\end{itemize}
\item 4 stories tall, decorated in statues on the outside
	\begin{itemize}
	\item partial retractable canopy on top
	\item hold 45,000 spectators
	\item buy a seat
	\item can be emptied in 10 minutes
	\end{itemize}
\end{itemize}

\subsubsection{Theatres}
\begin{itemize}
\item Oval amphitheatre with a semi-circular stage
\item Built on Greek models from southern Italy an across the eastern Empire
\item Intricate backdrops were contributed under the Empire
\item Stage and seating area
\item Theatre of Pompey (55 BC)
\item Theatre at Bosra could seat 15,000 and add 6,000 standing
\end{itemize}

\subsubsection{Arches}
\begin{itemize}
\item Victorious sculpture
\item Arch of Titus in Rome commemorates the capture of Jerusalem after the Jewish Revolt in 71 AD
\item Arch of Constantine (315 AD) covered in sculpture from earlier monuments of Trajan, Hadrian and Marcus Aurelius
	\begin{itemize}
	\item round medallions on it from Hadrian’s reign
	\item earlier sculpture re-worked to look like Constantine
	\end{itemize}
\end{itemize}

\subsubsection{Columns}
\begin{itemize}
\item Popular commemorative sculpture
\item Trajan’s Column (113 AD)
	\begin{itemize}
	\item tells the story of the Dacian War in sculpture
	\item spiral band of relief winds up the column (3 and a half feet tall and 800 feet long)
	\item reads like a scroll
	\item 2,500 figures on it
	\item Finer sculpted figures near the bottom
	\item painted, gilded and metal work
	\end{itemize}
\end{itemize}

\subsubsection{Pantheon}
\begin{itemize}
\item Built by the Emperor Hadrian (117-138 AD)
	\begin{itemize}
	\item “House of all Gods”
	\end{itemize}
\item Cylindrical, drum-shaped building, capped by huge dome and entered through a deep porch with Corinthian columns, incredible acoustics
\item Top of dome is 110 feet tall (= to the distance of the diameter of the drum)
	\begin{itemize}
	\item many square recesses in inside of roof to hold statues of deities (represents the heavens)
	\item small oculus (eye) opening in top of dome to let in light (represents the sun)
	\end{itemize}
\item Originally roof gilded in gold!
\item Floor slightly raised, so would self drain.
\end{itemize}

\subsection{Aqueducts}
\begin{itemize}
\item Water from mountain streams could be carried from 40 miles away using gravitational flow to reservoirs near cities
\item Cities die without water.
\item They do the math to figure out the angle and all the work, very smart.
\item Channels in arches were lined with hard, water-proof cement
\item Often built into bridges and supported by arches
\item Water then went through smaller pipes made of lead, wood or terra cotta, into fountains, houses or public baths.
\item 1st century AD Rome had a population of 1 million
	\begin{itemize}
	\item aqueducts provided 455 L of fresh water/person/day
	\end{itemize}
\item Rich have running water on main floor, in kitchen and lavatory
\item Many public fountains (some still used today)
\end{itemize}

\section{Roman Life}
\subsection{Roman House and Furniture}
\subsubsection{Housing}
\begin{itemize}
\item The Roman house (domus):
	\begin{itemize}
	\item compluvium (opening in roof to let in light and rain)
	\item impluvium (collects rainwater from compluvium)
	\item atrium (central living room surrounded by other rooms)
	\item hortus (small garden furthest away from the front door)
	\item lararium (shrine to household gods)
	\item fauces (entrance passage)
	\item tablinum (passage room)
	\item triclinium (dining room)
	\item cubiculum (bedroom)
	\item peristylium (elaborate colonnaded garden)
	\item hypocaust (heating system beneath floor)
	\item rich could have running water
	\item tabernae (shops) in front rooms facing the street (in wealthier homes)
	\end{itemize}
\item No washroom
\item Very bright and airy homes
\item All decoration on the inside
	\begin{itemize}
	\item façade is plain and whitewashed
	\end{itemize}
\item Insulae/”islands” (apartment blocks) 
	\begin{itemize}
	\item built in grid blocks
	\end{itemize}
\item So boring looking
\item Stone main floor (shops or expensive apartments)
	\begin{itemize}
	\item wood and stone “filler” upper floors
	\end{itemize}
\item Could have central courtyard (air and light)
\item No limit to occupancy (issues with smell, weight)
\item Cook on open braziers (fire hazard – these burn all the time)
\item Problem with disrepair
\item Augustus limits height to 21 m (69 feet)
\item 350 AD 1,790 private homes and 46, 602 insulae
\end{itemize}

\subsubsection{Furniture}
\begin{itemize}
\item Few pieces of furniture
\item sella (backless arm-chair)
\item cathedra (high-backed ladies' chair)
\item Cubile (bed) of simple frame, leather webbing and thin mattress stuffed with straw or wool
\item Arca (chest for blankets or clothes)
\item Lasanum (chamber pot – there's the bathroom – toss out  the window)
	\begin{itemize}
	\item busts, statues, decorations, lamps, tables (tables are expensive and often ornate – not too much use)
	\end{itemize}
\end{itemize}

\subsection{Life in Rome}
\begin{itemize}
\item Martial describes Rome as noisy, smelly, crowded, hot and foul!
\item Flooding of the Tiber (smell, mess, health and safety hazard)
\item Campus Martius ("field of Mars") People go to work out in the middle of the city
\item health: Cloaca Maxima (central sewer)
\item air pollution and fire hazards (cooking over open fire, fires, candles, bad air days, smog)
\item overcrowding: insulae (apartment blocks) often collapse, huge issues with plagues
\item unsafe streets: contrast Pompeii, Antioch
\item Forum (market and business centre)
\item Basilica (meeting hall, law court)
\end{itemize}

\section{Finance}
\begin{itemize}
\item early Rome: wealth based on property
\item coinage begins 3rd century BCE (no paper money)
\item The As is the smallest coin in value
	\begin{itemize}
	\item 1 Dupondius = 2 Asses (singular: As)
	\item 1 Sesterce = 2 Dupondii
	\item 1 Denarius = 4 Sestercii
	\item 1 aureus = 25 Denarii
	\end{itemize}
\item As and Dupondius are made of bronze (later copper)
	\begin{itemize}
	\item Sesterce and Denarius is made of silver (later brass)
	\item Aureus is made of gold 
	\end{itemize}
\item Values change over time based on inflation and devaluation
\item Coins are a valuable way to spread news, ideas and propaganda across the Empire
\item We have millions, great circulation and huge world of finance
\item Coins usually had the Emperor’s likeness on one side (and sometimes members of his family )
\item Most information is contained on the reverse side of the coin
	\begin{itemize}
	\item deities often portrayed if the Emperor wished to be associated with the deity (ie Minerva Pacifera “the bringer of peace”)
	\item animal images also used to show strength or courage, or are connected to a deity (ie Augustus used a Capricorn on one coin because it represents his birth sign
	\item buildings and military buildings may also appear
	\item the letters “SC” are also quite common, meaning “Senatus Consulto” (“by the authority, or decree, of the Senate”)
	\end{itemize}
\end{itemize}

\subsection{Controlling Finance}
\begin{itemize}
\item Temple of Juno Moneta: mint (board of three)
\item Temple of Saturn: holds aerarium (state treasury)
	\begin{itemize}
	\item controlled by quaestors (financial magistrates)
	\item under Empire, fiscus (fund controlled by emperor) is the state treasury and mint
	\end{itemize}
\item coin dies: anvil (heads), punch (tails)
\item coins spread by government, army, and money-changers (also test coins)
\item money-lenders (give credit at auctions; accept deposits)
\end{itemize}

\subsection{Gathering Revenues}
\begin{itemize}
\item revenues: from rentals, mines, booty
	\begin{itemize}
	\item many small cottage industries to tax
	\end{itemize}
\item 51 BCE Senate limits the interest on loans to 12\%
\item census every 5 years (tax registration)
\item provincial taxes (Italy exempt): fixed sum or part of harvest
\item Republic: publicani (holders of state contracts) collect Provincial taxes
\item Empire: taxes become city responsibility 
\item Portoria  (customs duties) on goods crossing provincial or imperial borders
	\begin{itemize}
	\item Rome’s trade ties at least as far as India 
	\end{itemize}
\item taxes on inheritance(5\%), manumission, slave sales, auctions, provincial tributum (poll tax)
	\begin{itemize}
	\item estimated total population of all free men, women and children in the Roman Empire at the time of Augustus’ death (14 CE) is 45 million 
	\end{itemize}
\item tax collectors hated!(some things never change…)
\end{itemize}

\section{Roman Sexual Life}
\begin{itemize}
\item Roman views towards sex and sexuality very different than ours
	\begin{itemize}
	\item sex for men outside marriage (with non-citizen women) was acceptable
	\item prostitution legal
	\item adultery more concerning than pederasty
	\item a man accused of being effeminate could defend his manhood by claiming to have had sex with his accuser’s son
	\item images of penises and sexual intercourse everywhere
	\item sex and sexuality a gift and nothing to be ashamed of
	\end{itemize}
\end{itemize}

\subsection{Roman Sexual Imagery}
\begin{itemize}
\item Roman viewed images of sex and sexuality in common, public places daily
	\begin{itemize}
	\item literature, law codes and public speeches speak of it
	\item shown on walls, paintings, vases, lamps, amulets and mosaics (one way for the elite to show off their wealth and success)
	\item found in baths, forums, the streets and homes
	\item statues have water flowing through the penis into a basin
	\item images of the penis everywhere (few images of female genitals)
	\item sexual graffiti extremely common and graphic
	\end{itemize}
\item Imagery and discussion of the phallus quite common
	\begin{itemize}
	\item a source of amusement and pride
	\end{itemize}
\item Phallus also considered an apotropaic object (it had protective powers)
	\begin{itemize}
	\item demons and humans could direct the “Evil Eye” on individuals to harm them
	\item phallus believed to be able to attract and deflect this (by the power of the divine phallus deity Fascinus)
	\item images of the phallus placed in dangerous spots in the community or home
	\end{itemize}
\end{itemize}

\subsubsection{The Fascina}
\begin{itemize}
\item Phallic imagery often placed on jewelry or amulets, called fascina , or on rings for Infants or those considered most vulnerable
	\begin{itemize}
	\item one more symbol of a male-dominated society
	\end{itemize}
\item Phallic symbol on the wall of a home to protect it from the “Evil Eye”
	\begin{itemize}
	\item found in Pompeii
	\end{itemize}
\end{itemize}

\subsection{Roman Sexual Life}
\begin{itemize}
\item In Early Rome, sex,  by tradition, only in marriage
	\begin{itemize}
	\item morality, dignity
	\item marriage contract, legitimate heirs
	\item economic concerns
	\item political relations
	\item ”not plow in another man’s field” (still      old, conservative, agrarian values)
	\item ”affairs” risk offending Pater Familias
	\item yet still had prostitutes	
	\end{itemize}
\item 2nd c. BC: influx of Greek values, foreign slaves
	\begin{itemize}
	\item much more liberal ideas
	\end{itemize}
\item immorality in the writings of Catullus, and Ovid’s  Art of Love
	\begin{itemize}
	\item use of “dirty words” by Catullus and Martial
	\item ie Mentula (“prick”) and “the big asparagus”
	\item loved to gossip about sexual escapades, affairs and mishaps
	\end{itemize}
\item Sexual rules: (1) only with spouse, (2) only at night, (3) woman mustn’t enjoy (only prostitues and low-class women do!)
\item double standard on adultery
\item Phallus is an important image of strength and power
	\begin{itemize}
	\item active partner is dominant and masculine
	\item passive partner is emasculated
	\end{itemize}
\item Oral sex: OK between men, and performed by women on men, but not performed by men on women
	\begin{itemize}
	\item risk getting the “Evil Eye” or being viewed as the subservient partner or perverse
	\end{itemize}
\item Upper classes felt that falling in love considered shameful, since marriage wasn’t for love
	\begin{itemize}
	\item turned men into women’s slaves
	\item insane (“madly” in love)
	\item Roman literature makes fun of men who fall in love with prostitutes and stalk them
	\item lower classes probably felt differently about falling in love
	\end{itemize}
\item Ovid writes in a “matter of fact” manner about the best “lovemaking positions” and how to achieve simultaneous orgasm in his Art of Love
\end{itemize}

\subsubsection{The Ideal Roman Woman}
\begin{itemize}
\item Ideal Roman woman was soft and smooth-skinned, with small firm breasts, a little “chubby”, wide round hips, and no body hair
	\begin{itemize}
	\item large breasts viewed as “barbaric”
	\item Martial writes of a breastband which read “I fear big-breasted women”
\item In order to conceive, one had to have an orgasm, which meant  one had enjoyed the sex act
	\item this had legal implications in cases of sexual assault
	\end{itemize}
\end{itemize}

\subsection{Alternative Lifestyles}
\begin{itemize}
\item Bisexual and homosexual “acts” are socially acceptable, but being exclusively homosexual is not
\item Being a homosexual did not limit ones social or political career, but:
	\begin{itemize}
	\item one had to be discreet
	\item not be a slave to one’s affections
	\item not be outrageous in public
	\item ie Julius Caesar, Mark Antony
	\end{itemize}
\item bisexual males: man must not be passive (only male prostitutes and slaves may be passive)
	\begin{itemize}
	\item shame in being the passive partner
	\end{itemize}
\item “Cinaedi” are effeminate, male transvestites
	\begin{itemize}
	\item believed to lack sexual self-control
	\item live counter to all the “manly norms of
	society” (curled hair, perfumed, bright colored
	clothes, provocative dancing) 
	\item many bi-sexual
	\item society rejects their “alternative sense of 
	shame”
	\end{itemize}
\item Juvenal’s 2nd satire – on gay men
	\begin{itemize}
	\item act and dress as women
	\item take passive role in sex act
	\item some born with this “disease”, others choose it (disgusting!)
	\end{itemize}
\item Female homosexuality: rarely mentioned
\item “Tribades” = aggressive lesbians who enjoy being the active partner
	\begin{itemize}
	\item “Women eager to lie with women then men and in fact pursue women with almost masculine jealousy....they rejoice in the abuse of their sexuality” -Doctor Caelius Aurelianus in “On Chronic Diseases”
	\item Caelius sees this as a mental illness
	\item hard to cure
	\item with age this condition causes “a hideous and ever increasing lust”
	\end{itemize}
\end{itemize}

\subsubsection{Pederasty}
\begin{itemize}
\item Pederasty, pimping and pornography legal
	\begin{itemize}
	\item Pederasty socially acceptable, within limits (a common Greek practice)
	\end{itemize}
\item Elite classes do not seduce elite classes, only lower classes
	\begin{itemize}
	\item penalty for seducing freeborn boys is exile
	\end{itemize}
\item Many have an attraction to young, smooth-skinned, beautiful pubescent and pre-pubescent boys (but no anal penetration)
	\begin{itemize}
	\item high prices paid for these boys (only elite could afford this)
	\end{itemize}
\item Poor and lower classes resort to engaging in these acts with slaves and “rough hewn people” in more public places
\item Elite feel that their more discrete, private and expensive encounters elevated the act to one of “artistic taste and style”, and a higher plane of “artistic appreciation”
	\begin{itemize}
	\item often kept a slave boy as you would a mistress
	\end{itemize}
\end{itemize}

\subsubsection{Concubines}
\begin{itemize}
\item Concubine: non-slave woman kept by unmarried man
	\begin{itemize}
	\item children free, but illegitimate
	\end{itemize}
\item After marriage, the wealthy often gave up concubines, but kept a “pretty cup-bearer”
\item Horace “Do you need a gold cup for your thirst?”
\end{itemize}

\subsubsection{Prostitution}
\begin{itemize}
\item Common across the Empire
	\begin{itemize}
	\item slaves, freemen, women and men
	\end{itemize}
\item prostitution cheap and readily available, so no need for risking “free sex” from an affair
	\begin{itemize}
	\item they “helped” to avoid offending a Pater Familias or Matrona by relieving the “needs” of men
	\item common “hobby” of single males
	\end{itemize}
\item Lower class prostitutes often found in taverns, Inns and public places (ie cemeteries, alleys, under arches, the Forum, etc) or in  brothels 
	\begin{itemize}
	\item ie Lupanar (brothel) at Pompeii
	\item elite looked down on this because it was too public (and noisy!) 
	\item elite used their homes and private rooms to be more discreet
	\end{itemize}
\item prostitutes’ cells for roadside sex (cubicle with a curtain)
	\begin{itemize}
	\item 9 in Pompeii
	\end{itemize}
\item Over the door of each cell was a tablet (titulus) upon which was the name of the occupant and her price for services rendered
	\begin{itemize}
	\item the reverse bore the word "occupata" 
	\item when the prostitute had a customer in the cell the tablet was turned so that this word was out.
	\item the cell usually contained a lamp of bronze or, in the lower dens, of clay, a cot of some sort, over which was spread a blanket or patch-work quilt
	\end{itemize}
\item Prostitutes must be registered with the local Aedile, (give their name, age, prices and “professional name”), and wear a toga (usually light blue)
\item Women wore a light blue male toga as a symbol of their trade, and to symbolize their more active, public role and “male” (ie active) identity
	\begin{itemize}
	\item registered with, and monitored by, the city Aediles (married women could be prostitutes!)
	\item taxed, as any other profession!
	\item great source of State revenue!
	\end{itemize}
\item Prostitution provided a socially accepted way to escape the Roman household and norms of behaviour
	\begin{itemize}
	\item many needed the money to survive
	\end{itemize}
\item Elite condemned them as “base women who….ought to appear in public at their best behaviour, but who actually misbehave the most in the streets” 
	\begin{itemize}
	\item Dio Chrysostom
	\end{itemize}
\end{itemize}

\subsection{Sex and Religion}
\begin{itemize}
\item sex and religion (fertility): phalluses
\item tintinnabula (phalluses with bells)
	\begin{itemize}
	\item phallic bronze windchimes!
	\end{itemize}
\item Priapus: fertility god with huge phallus(or huge phallus with misshapen body!)
	\begin{itemize}
	\item child of Dionysus and Aphrodite (or a nymph)
	\item ass sacrificed (embodiment of lust and stupidity!)
	\item not taken very serious
	\item wooden statues put in garden as a combination guardian and scarecrow
	\end{itemize}
\item Masturbation acceptable (but only indoors!), until the rise of Christianity
\end{itemize}

\subsection{Sexuality and Roman Law}
\begin{itemize}
\item Despite Rome’s liberal views towards sex and sexuality, certain lines were never crossed
\item Stuprum covered all forms of sexual misconduct
	\begin{itemize}
	\item in the early Republic a father could kill a daughter for this
	\item under Augustus’s legal/moral reforms, stuprum covered sex with unmarried women, a divorcee, a widow, a virgin or a freeborn boy
	\item a charge could be brought against both partners and punishments ranged from fines to forced marriage to divorce to death
	\item maximum penalty for rape was death
	\item those who committed murder while trying to stop a rape were given lesser punishments
	\end{itemize}
\item Adultery was a major concern and both parties were often punished
	\begin{itemize}
	\item guilty women had to be divorced by their husbands
	\item she lost half her dowry and one-third of her property
	\item the male lost half of his property
	\item it also limited their ability to inherit
	\item in some cases they were exiled to separate islands
	\end{itemize}
\item Augustus’ reforms also allowed a father to kill both his daughter and her lover if “caught in the act” in his own home
\end{itemize}

\section{Roman Imperialism}
\subsection{The Early Republic}
\begin{itemize}
\item Rome did not at first have imperial ambitions
	\begin{itemize}
	\item acquires territory haphazardly as result of success in wars
	\item struggle against neighbours, in which Roman army usually wins (“Divide and Rule”)
	\item leads to occupation of Italy; colonies set up for security and to settle surplus population
	\end{itemize}
\item 3rd c. BC: Rome, now in control of Italy
	\begin{itemize}
	\item comes into conflict with Carthage (1st Punic war)
	\item drives Carthaginians out of Sicily
	\item treaty gives Sicily to Rome (first Roman province)
	\end{itemize}
\item Provincia: originally = "task", "responsibility" of a magistrate
	\begin{itemize}
	\item comes to mean "province" (territory for which he is responsible)
	\end{itemize}
\end{itemize}

\subsection{The Middle Republic}
\begin{itemize}
\item 2nd Punic war: Rome captures Spain
	\begin{itemize}
	\item turns it into a province to keep Carthaginians from returning
	\item also rich silver mines, timber and soldiers
	\item also overruns Macedonia which had supported Hannibal
	\end{itemize}
\item 3rd Punic war: Africa itself becomes a Roman province
\item by 1st c. BC, Rome is now actively expanding its borders
	\begin{itemize}
	\item deliberate rather than accidental imperialism
	\item ally to a weak state and become its “protector” against its stronger neighbors (a “just” war “bellum iustum”)
	\item diplomatic continuation of the concept of the Patron-Client relationship
	\item often unscrupulous and meant to give Rome the image of being “the good guy”/justified
	\end{itemize}
\end{itemize}

\subsection{The Late Republic}
\begin{itemize}
\item Pompey's conquests in the Eastern Mediterranean (grid-plan conquest)
\item Caesar's conquest of Gaul and raids on Britain
	\begin{itemize}
	\item use of client kingdoms as buffer states
	\end{itemize}
\item Under Augustus, Rome establishes "natural" boundaries
	\begin{itemize}
	\item Atlantic Ocean on west
	\item Rhine-Danube on north
	\item Sahara and Syrian deserts on south and east
	\item August urges Tiberius not to expand 
	\end{itemize}
\end{itemize}

\subsection{The Empire}
\begin{itemize}
\item Changing concept of “imperium”
	\begin{itemize}
	\item originally command or authority
	\item later power, dominion ("empire")
	\end{itemize}
\item Rome needs conquests for money, slaves, raw materials, and political stability
	\begin{itemize}
	\item Romans brought up on a diet of conquest
	\end{itemize}
\item Augustus has greater imperium than all other provincial governors
	\begin{itemize}
	\item policy expressed in Vergil's Aeneid: "to rule nations by imperium, to add civilization to peace, to spare the submissive and 'war down' the proud”
	\end{itemize}
\item pax Romana – “The Roman Peace”
	\begin{itemize}
	\item peace and equality, but on Rome’s terms
	\end{itemize}
\item Empire reaches its greatest limits under Trajan
\item Hadrian retreats to defensive boundaries (Ie Rhine, Danube, and builds Hadrian’s Wall in Britain)
\end{itemize}

\section{The Roman Army}
\begin{itemize}
\item Strong army/Legion = secret of Rome's success
	\begin{itemize}
	\item reflects Roman virtues
	\end{itemize}
\item In the early Republic it was composed of citizens
	\begin{itemize}
	\item property requirement scrapped by Marius
	\item later, provincials join
	\end{itemize}
\item Legionaries must be citizens of good character
	\begin{itemize}
	\item present self to recruiting officers (for an interview) with a letter of introduction from father, local official, or family’s patron
	\item title for interview was “probatio”
	\item usually age 16-18
	\item check citizenship, health and height requirements (flexible!)
	\item numeracy and literacy also desirable assets
	\end{itemize}
\item A Legionary’s pay starts at 900 sesterces a year (before deductions)
	\begin{itemize}
	\item pay raises with promotion
	\item also bonuses; 12,000 sesterces on discharge
	\item monetary rewards for bravery
	\end{itemize}
\item As soldiers became posted on the frontiers, tented camps give way to wooden, then stone forts
	\begin{itemize}
	\item civilian settlements spring up outside the camp gates (London, Paris, Strasbourg)
	\item soldiers in private business
	\item many become part-time farmers to supplement their rations and income
	\item story of Corbulo in Syria, 58 AD
	\end{itemize}
\end{itemize}
	

\subsection{The New Recruit}
\begin{itemize}
\item Vegetius (Historian) tells of preference for recruits from certain professions (ie smiths, wagon-makers, butchers and huntsmen) vs others (those associated with women's occupations, like weavers, confectioners or even fishermen) 
	\begin{itemize}
	\item some potential recruits cut off some of their fingers in order to avoid service
	\item authorities decided to accept two mutilated men in place of one healthy one.
	\item early Empire, recruits swore an oath (the sacramentum)
 	and joined their unit
	\item later Empire, could be tattooed, or even branded 
	\item enlist for a minimum 20 years service
	\item until the 3rd century, could not marry
	\end{itemize}
\end{itemize}

\subsection{1st Century A.D. Legion Organization}
\begin{itemize}
\item Legion was divided into 10 cohorts 
	\begin{itemize}
	\item each cohort was divided into 6 centuries (each of 80 men, for a total of 480 men/cohort)
	\item each century divided into 10 contubernium (“tent parties”) of 8 men each (who shared a tent, millstone, a mule and a cooking pot)
	\item in practice, the First Cohort was double size
	\end{itemize}
\item 120 cavalrymen (divided into 4 Turma of 30 men each) attached to each legion 
\item With the addition of messengers, administrative and clerical staff, a legion consisted of about 5,400 men
	\begin{itemize}
	\item like most military units throughout history, it was rarely up to full strength
	\end{itemize}
\end{itemize}

\subsection{Legion Command Structure}
\begin{itemize}
\item Lead by example and strong leadership skills
	\begin{itemize}
	\item often military and political men out to make a name for themselves
	\end{itemize}
\item Legate commands each legion 
\item 6 military tribunes (staff officers) were sent by Rome to assist the Legate
\item Senior Centurion (like today’s Sergeant-Major) was the second-in-command to the Legate
	\begin{itemize}
	\item called the Primus Pilus (“first spear”) and commanded the First Cohort 
	\end{itemize}
\item Aquilifer carried the legion’s eagle standard
\item Imagifer carried the image of the Emperor
\item Centurion commands each century 
	\begin{itemize}
	\item aided by a Cornicen (musician), a Signifer (the century’s standard bearer), an Optio (the second-in-command of the century), and a Tesserarius (the third-in-command of the century)
	\end{itemize}
\item Praefectus Castrorum in charge of organising the legion’s camp
\item Decurion was in command of a Turma of 30 cavalrymen of the Equites Legionis (Legion Cavalry)
\end{itemize}

\subsection{Artillery and Auxiliary Troops}
\begin{itemize}
\item One Scorpio / “Scorpion” (arrow shooting artillery piece) was attached to each century (60 per legion!) 
\item One Onager / “Wild Ass” (stone throwing artillery piece) was attached to each cohort (10 per legion) 
\item Auxilia is Latin for “helpers”
\item auxiliaries (non-citizens) served as slingers, javelin throwers, archers, scouts and cavalry
\item many were “barbarians” who fought in their “native fashion”(ie Balaeric slingers, Cretan archers, Numidian light cavalry)
\item In many campaigns, auxiliary troops could outnumber the legions
\item auxiliaries served for  25 years, then received  citizenship on discharge
\end{itemize}

\subsection{The size}
\begin{itemize}
\item 28 legions in 1st c. AD (later 35)
	\begin{itemize}
	\item about half on Rhine-Danube (German Frontier)
	\end{itemize}
\item by the time of Constantine the Great, in the early 4th century, Rome had over 500,000 soldiers on active service
\item conscription rarely needed in Early Empire
\item by the mid fourth century, mercenaries and barbarians were recruited in huge numbers to make up for manpower shortages
	\begin{itemize}
	\item many Romans “buy” way out of service
	\item Legions shrink in size to 500-1,500 men
	\item far greater numbers of armoured cavalry, horse archers and foot archers in use
	\end{itemize}
\end{itemize}

\subsection{Army Life}
\begin{itemize}
\item Iron discipline is the key to success
	\begin{itemize}
	\item Centurion in Britain nick-named “give me another”
	\end{itemize}
\item train with double weight weapons
\item bribe centurion to get out of worse duties (ie guard duty, latrine cleaning)
\item severe punishments (ie Decimation)
\item long marches (24 miles/day)
	\begin{itemize}
	\item building camps (includes barracks, HQ, CO's house, hospital, granary) each night
	\item road building and agricultural work (when needed)
	\end{itemize}
\item Spartan diet
\item No wife
\item Learn a trade (ie accountant, doctor, vet, cobbler)
\end{itemize}

\subsection{Punishments}
\begin{itemize}
\item General had power of life and death over his soldiers
\item A series of punishments and rewards were used to instill discipline and loyalty
\item Severe discipline and intense training also enabled the soldiers to hold their ground at times when other armies would have broken ranks and fled the field
\item One of the key factors in the success of the Roman Army over 1,000 years
\end{itemize}

\subsubsection{Minor Punishments}
\begin{itemize}
\item Ascastigato
	\begin{itemize}
	\item It was not uncommon for a Roman Centurion to hit a soldier with his vitis (vine staff) for sloppy training, poor work or insubordination. The vitis was a swagger stick about three feet long and originally made of grape vine.
	\end{itemize} 
\item Pecunario Multo
	\begin{itemize}
	\item Loss of pay for damaging public property or losing their equipment.
	\end{itemize}
\item Munerum Indictio
	\begin{itemize}
	\item Giving a soldier extra duty (ie night time guard duty) or really unpleasant work (such as cleaning the latrines.
	\end{itemize}
\item Fustuarium
	\begin{itemize}
	\item For falling asleep on guard duty, dereliction of duty or desertion, a soldier could be beaten, flogged or stoned to death by his comrades (usually of his contubernium) whom he had put in danger.
	\item Polybius states that the fustuarium is "also inflicted on those who steal anything from the camp; on those who give false evidence; on young men who have abused their persons; and finally on anyone who has been punished thrice for the same fault."
	\end{itemize}
\end{itemize}

\subsubsection{More Serious Punishments}
\begin{itemize}
\item Animadversio Fustium
	\begin{itemize}
	\item Severe flogging/beating in front of the entire  unit to serve as an example to others. This was usualy the punishment for disobeying orders.
	\end{itemize}
\item Militiae Mutatio
	\begin{itemize}
	\item A soldier could either lose rank or his long-service privileges/bonuses, or both.
	\end{itemize}
\end{itemize}

\subsubsection{Collective Punishments}
\begin{itemize}
\item Decimation
	\begin{itemize}
	\item This was the worst punishment of all, and usually applied to an entire Cohort for such cowardly asks as fleeing before the enemy, mutiny or disobeying orders. Soldiers were divided into groups of ten and drew lots. The soldier that drew the unfortunate lot was killed by his remaining 9 comrades (either by stoning or clubbing). 
	\end{itemize}
\item Frumentum Mutatum
	\begin{itemize}
	\item Punishing a unit by switching their diet from wheat to barley (and so feeding them like pack animals).
	\end{itemize}
\item Extra Muros
	\begin{itemize}
	\item Ostracizing a unit by making it pitch its tent outside of the safety of the walls of the legionary camp.
	\end{itemize}
\item Misso Ignominosa
	\begin{itemize}
	\item An entire unit could be disbanded, with the loss of all pensions and bonuses.
	\end{itemize}
\end{itemize}

\subsection{Rewards}
\begin{itemize}	
\item Many rewards to inspire acts of bravery
	\begin{itemize}
	\item monetary bonuses
	\item booty and spoils from victory (including slaves)
	\item promotion in rank and/or pay
	\item Missio Honesta (honorary Discharge)
	\item Military Diploma (an official copy of an original bronze document issued by the emperor in Rome, granting an honorary discharge from military service, and Roman citizenship, to foreign veterans who had served for 25 years or more in the Roman auxiliary forces or navy.
	\end{itemize}
\end{itemize}

\subsubsection{Awards of Crowns for Bravery}
\begin{itemize}
\item The Grass Crown 
	\begin{itemize}
	\item the “siege crown” (corona obsidionalis) was the highest military award, awarded to the officer responsible for delivering a besieged
	\end{itemize}
\item The Civic Crown
	\begin{itemize}
	\item the “civic crown” of oak leaves (corona civica) was Rome’s second highest award, and was given for an act of bravery that saved the life of a citizen
	\end{itemize}
\item The Naval Crown
	\begin{itemize}
	\item the “naval crown” (corona navalis) was a gold crown decorated with a ship’s prow, awarded to the first man to board an enemy ship during a naval battle
	\end{itemize}
\item The Gold Crown 
	\begin{itemize}
	\item the “gold crown” (corona aurea) was awarded to Centurions and senior soldiers for killing the enemy in single combat 
	\end{itemize}
\item The Mural Crown
	\begin{itemize}
	\item the “mural crown” (corona muralis), was given to the first man over the walls of a besieged city
	\end{itemize}
\item The Camp Crown
	\begin{itemize}
	\item The “camp crown” (corona castrensis), was a golden crown awarded to the first man over the palisades of an enemy camp
	\end{itemize}
\end{itemize}

\subsubsection{Decorations and Medals}
\begin{itemize}
\item Following a battle, a general may present awards for bravery to individuals who have distinguished themselves. These awards include:
	\begin{itemize}
	\item Torques (gold necklet)
	\item Armilla (gold armband) 
	\item Phalerae (gold, silver, or bronze sculpted disks worn on the breastplate during parades)
	\item Hasta Pura (a ceremonial silver spear awarded to "the man who has wounded an enemy”
	\item Cup of silver or gold for a variety of acts of bravery
	\end{itemize}
\end{itemize}
\subsubsection{Military Honours} 
\begin{itemize}
\item Triumph
	\begin{itemize}
	\item legal wars that were won and resulted in at least 5,000 enemy dead required a Triumph
	\item Victorious General, in his best clothes and armour, with his face painted purple, was paraded through Rome
	\item accompanied by soldiers, captives and spoils of war
	\item procession ended at the Temple of Jupiter Optimus Maximus on the Capitoline Hill
	\item general made sacrifices to Jupiter for the victory
	\end{itemize}
\item Ovation
	\begin{itemize}
	\item lesser victories (fewer enemy dead or against lesser enemies, such as slaves) received an Ovation
	\item an honour, but less impressive procession and celebration
	\end{itemize}
\end{itemize}

\subsection{A Soldier’s Equipment} 
\begin{itemize}
\item Legionary carries 2 6-foot long javelins with soft points (Pilum singulat, Pila plural)
\item Short thrusting sword (Gladius)
\item Curved rectangular shield (wood bound in leather) called a Scutum
\item Metal helmet and armour
\item Bare legs (except in cold climates)
\item Thick-soled hob-nailed military sandals (Caligae)
\item Carry entrenching and road construction tools (“Marius’ Mules”)
	\begin{itemize}
	\item weight of all equipment approximately 90 lbs
	\end{itemize}
\item Auxilia usually carry lighter weapons and armour
\item Tortoise formation (shield interlocked over soldiers' heads)
\item Siege warfare: artillery, battering rams, catapults, onagers
\end{itemize}

\subsection{The Praetorian Guard}
\begin{itemize}
\item Praetorian Guard
	\begin{itemize}
	\item 9 cohorts x 1000 men (all Italians)
	\item commanded by Praetorian Prefect
	\item escort emperor, guard palace
	\item inner bodyguard are usually Germans
	\item only soldiers in Rome
	\end{itemize}
\item Paramilitary Forces in Rome
	\begin{itemize}
	\item urban cohorts (Rome’s city police force)
		\begin{itemize}
		\item 3 x 1000 men, under the City Prefect
		\end{itemize}
	\item vigiles (watchmen, firemen)
		\begin{itemize}
		\item 7 x 1000 men (all ex-slaves)
		\end{itemize}
	\end{itemize}
\end{itemize} 

\section{The Roman Navy}
\begin{itemize}
\item First developed a navy during 1st Punic War
	\begin{itemize}
	\item used captured Carthaginian ships as models
	\end{itemize}
\item 1st c AD the Mediterranean becomes a “Roman lake”
\item Divided into Classes (“Fleets”), each commanded by a Praefectus Classis (“Fleet Commander”)
\item Each ship was commanded by a Trierarchus (“Captain”), who held the rank of a Centurion
\item Their role was troop transport, escort of grain ships, policing the sea/major rivers, fighting pirates
\item Two main naval bases (Misenum, Ravenna)
	\begin{itemize}
	\item  other squadrons elsewhere (Rhine, Danube, etc)
	\end{itemize}
\end{itemize}

\subsection{Roman Sailors}
\begin{itemize}
\item Sailors can be non-citizens
\item Egyptians usually recruited/drafted as sailors
\item Classiarii (“marines”) equipped with lighter equipment, like the auxilia
\item Sailor’s tunics and ship’s sails were pale blue in colour (camouflage?)
\end{itemize}

\section{The Late Empire - The Decline and Fall of Rome (AD 235-476)}
\subsection{Barracks Emperors (235-284)}
\begin{itemize}
\item This period marked the beginning of the end for Rome
	\begin{itemize}
	\item most Emperors were short-lived and died violently
	\item financial and military troubles
	\item high, unfair taxes (rich get richer)
	\item rampant inflation
	\item cities go bankrupt, literally close cities.
	\item robber bands on the rise
	\item constant civil wars and plots to take power
	\item plagues and famines
	\item barbarian invasions
	\end{itemize}
\end{itemize}

\subsubsection{Maximinus the Thracian (235-238)(First of the “Barracks Emperors”)}
\begin{itemize}
\item Proclaimed by the Pannonian legions
\item Murdered Emperor Alexander Severus and his mother
\item An ignorant peasant of tremendous size and strength
	\begin{itemize}
	\item reportedly drank 46 pints of wine and ate 40 pounds of meat daily!
	\end{itemize}
\item 1st “barbarian” (a Goth) Emperor, and 1st Emperor to never set foot in Rome
\item Eventually lynched by his own troops when he was unable to pay them
\end{itemize}

\subsubsection{Pupienus, Balbinus and Gordian III}
\begin{itemize}
\item Pupienus and Balbinus (238)
	\begin{itemize}
	\item elderly Senators
 	\item Seek to control them.
	\item murdered by troops after two months
	\end{itemize}
\item Gordian III (238-244)
	\begin{itemize}
	\item 13 year old co-emperor with Pupienus and Balbinus (they are forced into adopting him)
	\item Praetorian Prefect (Philip the Arab) acts as regent
	\item Goths and Persians invade
	\item murdered when he gave troops choice of either “Philip or me!”
	\end{itemize}
\end{itemize}

\subsubsection{Philip the Arab (244-249)}
\begin{itemize}
\item Arab sheik from Jordan
	\begin{itemize}
	\item was Gordian III’s advisor, commander-in-chief and Praetorian Prefect
	\item fought Goths and Persians
	\item 248 celebrates 1,000th year since founding of Rome
	\item murdered in civil war against Decius
	\end{itemize}
\end{itemize}

\subsubsection{Decius, Hostilianus and Gallus}
\begin{itemize}
\item Decius (249-251)
	\begin{itemize}
	\item a good man who could have been a great emperor
	\item killed in battle with the Goths (unusual, first non-coward emperor in a long time)
	\end{itemize}
\item Hostilianus (June-July 251)
	\begin{itemize}
	\item son of Decius
	\item Adopted son of Gallus
	\item died of plague
	\end{itemize}
\item Gallus (251-253)
	\begin{itemize}
	\item proclaimed by troops of Lower Moesia
	\item murdered by mutinous troops
	\end{itemize}
\end{itemize}

\subsubsection{Aemilianus and Valerian I}
\begin{itemize}
\item Aemilianus (253)
	\begin{itemize}
	\item murdered by mutinous troops
	\end{itemize}
\item Valerian I (253-260)
	\begin{itemize}
	\item ran the Eastern Roman Empire
	\item captured by Persian King Shapur I
	\item used as a human mounting-block for his horse (used as a stepping stool)
	\item on his death, the skin was flayed from his body, dyed with vermilion, and hung in a Persian temple!  For future reference, great way to intimidate! 
	\end{itemize}
\end{itemize}

\subsubsection{Gallienus and Claudius II}
\begin{itemize}
\item Gallienus (253-268)
	\begin{itemize}
	\item son and co-Emperor with Valerian I
	\item ran the Western Roman Empire
	\item faced invasions by the Franks, Goths, Saxons, Jutes and Persians (Lots of battle, hard to balance troops)
	\item fought off 18 rebellions against him!
	\item created a mobile, elite, central reserve army to rush to trouble spots
	\item murdered by jealous staff officers
	\end{itemize}
\item Claudius II (268-270)
	\begin{itemize}
	\item staff officer of Gallienus
	\item defeated Germans and Goths
	\item died of plague
	\item invasions by Franks, Goths, Persians, etc.
	\end{itemize}
\end{itemize}

\subsubsection{Lucius Domitius Aurelianus (“Aurelian”)}
\begin{itemize}
\item Aurelian (270-275)
	\begin{itemize}
	\item Illyrian Emperor from the Balkans (the former Yugoslavia)
	\item very harsh disciplinarian, whose nick-name was Manu ad ferrum (“Hand on Steel”)
	\end{itemize}
\item Gets the army back in line.
	\begin{itemize}
	\item both Gaul and Palmyra separate from the Empire, he got them back.
	\item both defeated, as were the Goths
	\item builds a new defensive wall around Rome
	\item murdered by mistake by Praetorian Guard Officers(thought he had a “hit list”), they messed up.
	\end{itemize}
\end{itemize}

\subsubsection{Marcus Claudius Tacitus}
\begin{itemize}
\item Claudius Tacitus (275-276)
	\begin{itemize}
	\item a Senator in his mid-seventies!
	\item murdered after six months
	\end{itemize}
\end{itemize}

\subsubsection{Marcus Annius Florianus}
\begin{itemize}
\item Florianus (276)
	\begin{itemize}
	\item maternal half-brother of Tacitus and his Praetorian Prefect
	\item assumed the throne on Tacitus’ desth
	\item murdered by the army
	\item ruled only 88 days!
	\end{itemize}
\end{itemize}

\subsubsection{Marcus Aurelius Probus}
\begin{itemize}
\item Probus (276-282)
	\begin{itemize}
	\item another excellent Illyrian General
	\item defeats the Franks, Germans, Burgundians and Vandals
	\item murdered by mutinous troops, who supported his Praetorian Prefect, Marcus Aurelius Carus 
	\end{itemize}
\end{itemize}

\subsubsection{Marcus Aurelius Carus}
\begin{itemize}
\item Aurelius Carus (282-283)
	\begin{itemize}
	\item an Equestrian from Gaul
	\item fought the Germans, Sarmations and Persians
	\item died in Persia from disease, a lightning bolt strike, or possibly the dagger of his Praetorian Prefect(!)
	\end{itemize}
\end{itemize}

\subsubsection{Numerianus (283-284)}
\begin{itemize}
\item Son of Carus
	\begin{itemize}
	\item had many vices (swam in cold water, and in bathes of apples and melons, deflowered virgins and officers wives, and took revenge on old childhood friends who were mean to him) 
	\item stabbed or poisoned by Praetorian Prefect, Aper,  in Persia
	\item kept death secret and body carried through Asia Minor in a covered litter until a “smell” was noticed
	\end{itemize}
\end{itemize}

\subsection{The Tetrarchy of Diocletian}
\begin{itemize}
\item Diocletian rules jointly with Maximian
\item Tetrarchy:
	\begin{itemize}
	\item Empire divided officially into East and West
	\item joint rule of 2 Augusti(Diocletian , Maximian) + 2 Caesars (Galerius, Constantius)
	\item provinces regrouped into prefectures and dioceses (under vicar)
	\item now grow from 50 to 100+ provinces, to distribute size and power better.  Prevents a large, powerful province to have a successful revolt.
	\end{itemize}
\item Emperor worshipped like god
\item Becomes a figurehead instead of a physically present emperor, for fear of assassination.
\item Persecution of Christians
\item Economy: real gold (“Solidus”) and silver coins
	\begin{itemize}
	\item Edict on Maximum Prices
	\end{itemize}
\item People get around these rules in typical fashion, black markets, tax avoidance.
	\begin{itemize}
	\item occupations begin to become hereditary (this doesn't always work out...)
	\item some not rich or talented enough to be city magistrates, but now forced into positions
	\item annual taxes vs sporadic
	\end{itemize}
\item Rome no longer a major city (capital and mint travel with the Emperor)
\item Emotional and spiritual capital of the empire, but the real capital becomes where the emperor is.
\item Legions grow in number from 39 to 65
	\begin{itemize}
	\item legions drop in size from 5,500 to 1,000
	\item more cavalry, mobility and missile troops (to react faster and to the present threat of light armoured, high mobility foes)
	\item more barbarians and mercenaries hired 
	\item money in lieu of enlistment in army
	\item “Dukes” and “Counts” control/defend territories
	\item all very expensive	
	\end{itemize}
\item May 1st, 305 abdicates and retires to his fortress palace at Split on the Adriatic.  He thinks he did a good job setting everything up and he decided it was time to call it a day.
	\begin{itemize}
	\item spent the rest of his days weeding turnips and cabbages in his garden
	\item died in his bed in 313 AD
	\end{itemize}
\end{itemize}

\subsubsection{End of the Tetrarchy}
\begin{itemize}
\item Tetrarchy breaks down when Constantine (Caesar of the West) proclaimed emperor
\item Civil War of 306-324 AD results in the deaths of the Tetrarchs and claimants to the throne 
	\begin{itemize}
	\item Galerius (plague crotch leprosy?)
	\item Maxentius (drown  in Tiber)
	\item Maximinus Daia (plague/poison self?), gathers all his friends together and then poisons himself, but does a botched job and takes days to die from it.
	\item Licinius (executed for treason)
	\item Maximian (hanged himself?)
	\end{itemize}
\end{itemize}

\subsection{Constantine I , The Great}
\begin{itemize}
\item Rules jointly with Licinius, Augustus of the East (until 324)
\item Battle of Milvian Bridge 312 AD
	\begin{itemize}
	\item ”IN HOC SIGNO VINCAS” (With this sign you will conquer)
	\item the first Christian Emperor! (Doesn't convert right away, because Christians can't kill.  Travels with band of priests to be baptised just before death)
	\item Edict of Milan (313) legalizes Christianity
	\item Council of Nicaea (325): bishops assemble
	\item Christians appointed to high positions
	\item some privileges taken from pagan cults
	\end{itemize}
\item Does this slowly, as to not freak out everyone.
\item Some people convert out of perceived opportunity.
\item “Chi-Rho” symbol (for “Christ”) used on standards
\item “New Rome" at Byzantium (renamed Constantinople) in 324 (also protect the east and Danube frontiers)
\item 326 AD death of son Crispus and second wife Fausta
\item Occupations continue to be hereditary, in theory
	\begin{itemize}
	\item occupations tattooed on people!
	\item growth of guilds (like coop!)
	\end{itemize}
\item Baptized on his deathbed!
\item On his death, his three sons, Constans, Constantius and Constantine II (all by Fausta) ruled the Empire until 360
\item Sons kill each other off.
\item In Diocletian and Constantine, we see the birth of the Middle Ages
\end{itemize}

\subsubsection{Military Reforms}
\begin{itemize}
\item Military reforms carried out
	\begin{itemize}
	\item creates a two-pronged military force
	\item Comitatenses are the elite, mobile field armies of the Emperor
	\item better trained and equipped troops
	\item more armoured cavalry
	\item troops moved back into mobile reserves to rush to trouble spots along the frontier
	\item Limitanei are the less well-equipped frontier armies
	\item patrol and guard the frontier and call on Comitatenses when needed
	\item creates a “defense in depth”
	\item barbarians enrolled at all levels of the military
	\end{itemize}
\end{itemize}

\subsection{A Few More Emperors}
\subsubsection{Julian the Apostate (360-363)}
\begin{itemize}
\item 357 wins Battle of Strasbourg vs Alemanni
	\begin{itemize}
	\item cavalry forced to parade in women’s clothing for running away
	\end{itemize}
\item Tried to re-introduce “Patriotic Paganism” and Ares (the God of War) into Roman worship in opposition to Christianity (failed)
\item Hit in the groin with a javelin in Persia
	\begin{itemize}
	\item dieing words “Take your fill, Nazarene!” (Thinks god was out to get him – may have been made up in the future by Christian historians)
	\end{itemize}
\end{itemize}

\subsubsection{Valentinian I and Valens (364-378)}
\begin{itemize}
\item Valentinian I (364-375)
	\begin{itemize}
	\item Emperor of the West
	\item made brother Valens co-emperor and Emperor of the East
	\item fought in Gaul and Germania
	\item died from a stroke after yelling at German envoys (the usual)
	\end{itemize}
\item Valens (364-378)
	\begin{itemize}
	\item war with Goths and Persians
	\end{itemize}
\item 378 CE Killed by Goths at the Battle of Adrianople
\end{itemize}

\subsubsection{Theodosius I, The Great (379-395)}
\begin{itemize}
\item Last Emperor to rule a united Empire
	\begin{itemize}
	\item Empire officially Christian
	\item all paganism banned (time to persecute! Because people have short memories)
	\item lets Goths settle within the Empire
	\item Germanization of the Roman army
	\item some armies disappear when wages can’t be paid
	\end{itemize}
\item On his death, Empire is partitioned (East and West)
\end{itemize}

\section{The “Barbarian” Invasions}
\begin{itemize}
\item Nations on the march
	\begin{itemize}
	\item lesser tribes combine in the 1-3rd centuries and create “super tribes”
	\end{itemize}
\item Collectively “finish off” a decaying Empire
\item Jutes, Angles and Saxons invade Britain
\item Franks and Burgundians invade Gaul and Germany
\item Ostrogoths (East Goths) invade Crimea, Turkey and Greece
\item Visigoths (West Goths) invade Italy and Spain
\item Vandals invade Italy Spain and North Africa
\item Huns drive all before them
	\begin{itemize}
	\item attack Eastern Empire, Italy and Gaul
	\end{itemize}
\end{itemize}

\subsection{The Decline of the West}
\begin{itemize}
\item 409 AD Rome pulls her troops out of Britain to defend Gaul
	\begin{itemize}
	\item beginning of Arthurian legends?
	\end{itemize}
\item 410 Goths sack Rome
\item 450's Attila the Hun ravages Italy
	\begin{itemize}
	\item Battle of Chalons (451 AD)
	\item Death of Attila (453 AD)
	\end{itemize}
\end{itemize}

\subsection{The Last Caesar}
\begin{itemize}
\item Vandals seize Africa and sack Rome in 455 AD (trash and burn everything)
\item 476 Romulus Augustulus (last emperor)
	\begin{itemize}
	\item ruled for eleven months
	\item only 14 years old
	\item captured by a mutinous Roman (German!) army
	\item given the choice of death, or to abdicate and go into comfortable retirement on the Bay of Naples, with an annual pension of six thousand gold pieces.
	\end{itemize}
\item Replaced by Odoacer, a German “barbarian” king, who declared himself King of Italy
	\begin{itemize}
	\item begins the “Dark Ages” in the West (West empire is done)
	\end{itemize}
\end{itemize}

\section{Did Rome Fall or Evolve?}
\begin{itemize}
\item Why did Rome fall? (“Multiple Causation Theory”)
	\begin{itemize}
	\item Was it too old and corrupt to survive?
	\item Did plagues too greatly reduce the population to sustain itself? 
	\item Did civil wars weaken the Empire and leave it vulnerable to foreign invasion? 
	\item Did the army’s lack of discipline make it an enemy within the Empire itself? 
	\item Did the Romans become too decadent to hold the Empire together? 
	\item Did the Imperial Civil Service/bureaucracy become too top heavy and inefficient, eventually causing the empire to collapse upon itself? 
	\item Did the Roman Patrician class become too sterile (plague, disease, in-breeding, lead poisoning) to produce outstanding leaders?
	\item Did Christianity create a population more concerned with Heaven and not Earth? 
	\item Did it fall as the result of barbarian invasions? 
	\item Did the empire spend too much of its resources on the poor, thus drawing away precious funds from the empire? 
	\item Was the Roman Empire just too big, making a collapse inevitable? 
	\end{itemize}
\item or did it evolve?
\item Byzantine (Eastern Roman) Empire, based at Constantinople, survives
	\begin{itemize}
	\item called selves “Romans”
	\item tried to re-take the West several times
	\item in decline from the 7th century onwards
	\item lasts until 1453 (falls to the Ottoman Turks)
	\end{itemize}
\end{itemize}

\section{The Roman Legacy}
\subsection{4th-7th Centuries}
\begin{itemize}
\item Collapse of Western Roman Empire leads to the “Dark Ages”
	\begin{itemize}
	\item former Roman provinces, now in the hands of rival barbarian groups,   disintegrate into barbarian kingdoms
	\item Merovingian Franks in Gaul (France)
	\item Visigoths in Spain	
	\item Ostrogoths in Italy
	\item Vandals in North Africa
	\item Angles and Saxons in Britain
	\end{itemize}
\item Results:
	\begin{itemize}
	\item Financial collapse
	\item Virtual end of long-distance trade
	\item De-urbanization (or, re-ruralization…)
	\item Collapse of central political and military control
	\item Several capitals, several kingdoms, several kings competing for those kingdoms.
	\item Loss of “high”, complex culture (too busy trying to survive)
	\item “Romans” still there, and learn to work with their new overlords
	\end{itemize}
\item “Barbarians” settle and become “civilized” by copying Roman styles and institutions (become “Dukes” and “Counts”)
	\begin{itemize}
	\item Germanic Warlords wish to be seen as Kings, so create law codes/coins under Roman influence
	\item Churchmen provide literate scribes to illiterate kings (Latin preserved)
	\item monasteries preserve the “wisdom of the ancients”
	\end{itemize}
\item Co-existence of Latin written and Germanic oral traditions
\item Last masterpiece from Antiquity: Boethius’ Consolation of Philosophy (c. 524 AD)
\item Decline of education, e.g. Historian  Gregory of Tours illiterate
	\begin{itemize}
	\item Roma Aeterna (Eternal Rome) praised by late Latin poets
	\item “ideal” of Rome continues through the ages
	\end{itemize}
\end{itemize}

\subsection{8th-10th Centuries}
\begin{itemize}
\item Rise of Islam: conquest of Africa, Spain (622-750 AD)
	\begin{itemize}
	\item by 750 conquer half of the old Roman Empire 
	\end{itemize}
\item Latin replaced by German, Romance languages, Arabic
\item 8th c.: pilgrims seek books in Rome
\item Many in the West look to the Roman Church for leadership
\item Medieval scholarship based on Latin texts and classical literature (the Trivium: Grammar, Rhetoric, Logic, and the Quadrivium: Arithmetic, Geometry, Music, Astronomy) 
	\begin{itemize}
	\item texts copied by monks, thus survives the Dark Ages
	\item monasteries foster Latin as the universal language of the Church, Court and educated society
	\end{itemize}
\end{itemize}

\subsection{11th-13th Centuries}
\begin{itemize}
\item 1070 Justinian’s Code (Corpus Juris Civilis) rediscovered in the West (Bologna) and becomes basis of Western law
\item Popes gain great political power and call  Crusades (1095-1291)
	\begin{itemize}
	\item Pope Urban II calls First Crusade
	\item ”Christendom” becomes the new Roman Empire
	\end{itemize}
\end{itemize}

\subsection{14th-16th Centuries}
\begin{itemize}
\item "lost" works rediscovered in monasteries
\item Gutenberg’s printing press makes Latin literature widely accessible
	\begin{itemize}
	\item rebirth of classical culture, art, architecture, styles
	\item Renaissance, reaches height in 15th century
	\end{itemize}
\end{itemize}

\subsection{17th - 18th Centuries}
\begin{itemize}
\item The Age of Enlightenment of the 18th century reflected many of the ideals of the Greco-Roman legal, intellectual and cultural traditions
\item This is also reflected in some of ideals of the American Declaration of Independence, the United States Bill of Rights, and the French Declaration of the Rights of Man
\item The concept of the authority of a Caesar is reflected in the title and role of the German Kaiser and the Russian Czar
\end{itemize}

\subsection{19th - 21st Centuries}
\begin{itemize}
\item 1837 – 1901 Reign of Queen Victoria
	\begin{itemize}
	\item Victorian England copies much of Roman culture
	\item heirs of the Roman Empire
	\end{itemize}
\item 20th century
	\begin{itemize}
	\item Church Latin is universal until the 1960’s
	\item switch to Vulgate
	\end{itemize}
\end{itemize}

\subsubsection{The survival of Rome}
\begin{itemize}
\item Imperial boundaries
\item Pontifex Maximus / Pope
\item Roman numerals
\item Latin language
\item Julian calendar / Months / Days (names)
\item Alphabet
\item Legal system
\item Coinage
\item English vocabulary
\item Imperialism
\item Literary tradition
\item Technology/Engineering
\item Town planning
\item Christianity
\item etc...
\end{itemize}


\end{document}