\documentclass[12pt, twoside, exarticle]{article}
\usepackage[left=28mm, top=24mm, right=28mm, bottom=24mm, asymmetric, reversemarginpar]{geometry}
\usepackage{titlesec}
\usepackage{marginnote}
\usepackage{listings}
\usepackage{color}
\usepackage{xkeyval}
\usepackage{varwidth}
\usepackage{microtype}
\usepackage{hyperref}
\usepackage{enumerate}
\usepackage{amsmath}
\title{\textbf{Math239 Review Notes}}
% Code box.
\lstnewenvironment{code}[1][]
	{\begingroup
		%\vfil\penalty-9999\vfilneg\lstset{language=#1}
		\lstset{language=#1}
	}
	{\endgroup}

% Definition box.
\newcommand{\defnbox}[2] {
	\setlength{\fboxsep}{8pt}
	\marginpar {
		\vspace{0.9em}
		\begin{center}
		\footnotesize{\textbf{\color{brown}DEFINITION}}
		\footnotesize{\textbf{#1}}
		\end{center}
	}
	\colorbox{lightyellow}{
		\begin{minipage}{\dimexpr\linewidth-2\fboxsep}
		#2
		\end{minipage}
	}
	~\\
}

% Example box.
\newcommand{\exbox}[2] {
	\setlength{\fboxsep}{8pt}
	\marginpar {
		\vspace{0.9em}
		\footnotesize{\textbf{\color{darkpurple}EXAMPLE #1}}
	}
	\colorbox{lightpurple}{
		\begin{minipage}{\dimexpr\linewidth-2\fboxsep}
		#2
		\end{minipage}
	}
	~\\
}

% Exercise box.
\newcommand{\exerbox}[1] {
	\setlength{\fboxsep}{8pt}
	\marginpar {
		\vspace{0.9em}
		\footnotesize{\textbf{\color{darkred}EXERCISE}}
	}
	\colorbox{lightred}{
		\begin{minipage}{\dimexpr\linewidth-2\fboxsep}
		#1
		\end{minipage}
	}
	~\\
}

% Used on the side for definitions.
\definecolor{brown}{RGB}{101, 91, 71}
\definecolor{lightyellow}{RGB}{228, 224, 128}

% Used for the code block itself.
\definecolor{codebg}{RGB}{255, 255, 238}
\definecolor{codeborder}{RGB}{243, 242, 222}

% Used for exercises.
\definecolor{darkred}{RGB}{203, 20, 20}
\definecolor{lightred}{RGB}{229, 130, 130}

% Used for examples.
\definecolor{darkpurple}{RGB}{76, 60, 189}
\definecolor{lightpurple}{RGB}{184, 183, 255}

% Used for C and Lisp Syntax.
\definecolor{purple}{RGB}{174, 19, 198}
\definecolor{darkblue}{RGB}{0, 0, 102}
\definecolor{lightblue}{RGB}{50, 155, 171}
\definecolor{lightgreen}{RGB}{29, 131, 43}

% Document formatting for headings.
\pagestyle{myheadings}
\setcounter{secnumdepth}{4} % 4 being sub sections.

% Removes indentation of paragraphs.
\setlength{\parindent}{0cm}

% Sets page numbering to roman.
\pagenumbering{roman}

% Declaring the default listing style.
\lstdefinestyle{default_style} {
	backgroundcolor=\color{codebg},
	rulecolor=\color{codeborder},
	stringstyle=\color{purple},
	keywordstyle=\color{darkblue},
	identifierstyle=\color{lightblue},
	commentstyle=\color{lightgreen},
	basicstyle=\footnotesize\sffamily,
	xleftmargin=10pt,
	xrightmargin=10pt,
	belowcaptionskip=10pt,
	belowskip=20pt,
	framesep=10pt,
	frame=single,
	%numbers=left,
	%numbersep=8pt,
	showspaces=false,
	showstringspaces=false,
	tabsize=2
}

% Sets the default style for all code blocks.
\lstset {
	style=default_style
}

% Module section shortcut commands.
\newcommand{\newpagesection}[1] {
	\clearpage
	\section{#1}
}

\newcommand{\newpagesubsection}[1] {
	\clearpage
	\subsection{#1}
}
\begin{document}
\makeatletter
\hfil\parbox[t]{0.7\textwidth}{\centering\LARGE\bfseries\@title}\par
\kern0.5cm \hrule\kern0.5cm
\makeatother

% Table of contents
\renewcommand{\contentsname}{Table of Contents}
\tableofcontents
\clearpage

% Content
\pagenumbering{arabic}
\setlength{\oddsidemargin}{1.6cm}
\setlength{\evensidemargin}{\oddsidemargin}
\setlength{\marginparwidth}{2.6cm}
\setlength{\marginparsep}{0.25cm}

\section{Combinatorial Proofs}

This course will begin with combinatorial proofs.  However, to begin these proofs, we must first review the definition of a function. \\

\subsection{Functions}

\defnbox{Function}{A function $f: A \to B$  is:
\begin{enumerate}
\item \textbf{Injective} if $f(x) = f(y) \to x=y$.
\item \textbf{Surjective} if $\forall b \in B, \exists a \in A$ such that $f(a) = b$.
\item \textbf{Bijective} if \textbf{Injective} and \textbf{Surjective}.
\end{enumerate}
}

A bijective function is called a \textbf{bijection}.  To show that a function $A$ is in \textbf{bijection} with $B$ we must follow these steps: \\

 \defnbox{Algorithm to Show Bijection}{
 \begin{enumerate}
 \item Define a function $f$ on $A$. Make it as clear as possible.
 \item Show (if it is not evident) that $f(a) \in B, \forall a \in A$.
 \item Define an inverse of $f$, $f^{-1}$.
 \item Prove that $f \circ f^{-1} = 1$ and $f^{-1} \circ f = 1$.
 \end{enumerate}
 }

 Now that we have the notions of functions and bijections, let's see an example of a combinatorial proof. \\

 \exbox{1}{Show $\binom{n}{k} = \binom{n}{n-k}$}

First, let's do the algebraic proof: \\

$\binom{n}{k} = \frac{n!}{k!(n-k)!} = \frac{n!}{(n-k)! k!} = \binom{n}{n-k}$ \\

Now, the combinatorial proof. \\

$\binom{n}{k}$ is the number of subsets of $k$ elements in ${1, \dots, n}$.  In order to simplify this statement we'll use the following notation: $[n]$ denotes ${1, \dots, n}$.  Also, we will let $P(s)$ define the power set, that is all subsets of a set $s$. \\

Now, let $P_k(s) = \{T \subseteq S \: | \: |T| = k\}$ then we know that $\binom{n}{k} = | P_k([n])$. \\

Note now that there is a \textbf{Bijection} $P_k([n]) \to P_{k-n}([n])$. We can then conclude:\\

$\binom{n}{k} = |P_k([n])| = |P_{n-k}([n])| = \binom{n}{n-k}$. As desired. \\ \\

\exbox{2}{A binary string of length n is a word $s_1s_2,\dots,s_n$ with n letters where $s_i \in {0,1}$.  Let $B_n$ be the set of all binary strings of length n.  I claim that $P([n]) \cong B_n$ where $\cong$ means that $A\cong B$ is a \textbf{bijection}}

Why? \\

Consider the function:\\
$P([n]) \to B_n$ \\
$S \mapsto a_1,\dots,a_n$ where \\ 
\[ a_i = \left\{ 
  \begin{array}{l l}
    0 & \quad \text{if i $\notin$ S}\\
    1 & \quad \text{if i $\in$ S}
  \end{array} \right.\] \\

And its obvious inverse: \\
$B_n \to P([n])$ \\
$ a_1,\dots,a_n\mapsto {i\in [n] | a_i = 1}$ \\

The consequence of this, as per bijection is that $|P([n])| = |B_n| = 2^n$ \\

A few concluding questions: \\

\exbox{3}{How many binary strings of length n contain exactly $k$ $1$'s'.}

\exbox{4}{How many binary strings of length n do not contain an odd length maximal substring of $0$'s'.}

\exbox{5}{How many binary strings of length n do not contain $0101$ as a substring.}














\end{document}